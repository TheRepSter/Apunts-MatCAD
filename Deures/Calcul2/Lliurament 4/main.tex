\documentclass[a4paper, 12pt]{article}
\usepackage{deuresutils}

\title{Lliurament 4}
\asignatura{Càlcul en Diverses Variables}
\author{Eduardo Pérez Motato}
\niu{NIU: 1709992}
\date{04/06/2024}

\begin{document}
    \makeheader

    \begin{exercici}
        \begin{enumerate}[label=\alph*)]
            \item Determineu els exterms relatius i els punts de sella de la funció $f\left(x,y\right)=x^3+y^3-6xy$\\
            \begin{solucio}
                Per trobar els extrems relatius d'aquesta funció i els punts de sella hem de trobar
                quan s'anula el gradient de la funció:
                \begin{displaymath}
                    \begin{split}
                        \left(0,0\right) &= \nabla f(x,y)\\
                        &= \left(3x^2-6y, 3y^2-6x\right)\\
                    \end{split}
                \end{displaymath}
                Llavors, amb aixó tenim el seguent:
                \begin{displaymath}
                    \begin{cases}
                        3x^2-6y = 0\\
                        3y^2-6x = 0
                    \end{cases}
                \end{displaymath}
                Llavors, veiem que $y = \frac{x^2}{2}$ i llavors $$3 \frac{x^4}{4}-6x = 0 \Rightarrow \begin{cases}
                    x = 0, y = 0\\
                    x = 2, y = 2
                \end{cases}$$
                Amx aixó, sabem que els extrems relatius són aquells.
            \end{solucio}
            \item Trobeu els màxims i mínims de la funció \( f(x, y) = x - y + z \) sota la condició \( x^2 + y^2 + z^2 = 2 \).\\
            \begin{solucio}
                (Suposo que $f\left(x,y,z\right) = x-y+z$)\\
                Tenim que $g(x,y,z) = x^2+y^2+z^2 - 2$, llavors apliquem el teorema de multiplicadors
                de Lagrange:
                \begin{displaymath}
                    \begin{split}
                        \left(x,y,z\right) &= \nabla\left(f(x,y,z)-\lambda g(x,y,z)\right)\\
                        &= \nabla\left(x-y+z-\lambda\left(x^2+y^2+z^2-2\right)\right)\\
                        &= \left(1-2x\lambda,1-2y\lambda,1-2z\lambda\right)
                    \end{split}
                \end{displaymath}
                Amb aixó, tenim les seguents equacions:
                \begin{displaymath}
                    \begin{cases}
                        1-2x\lambda = 0\\
                        1-2y\lambda = 0\\
                        1-2z\lambda = 0\\
                        x^2+y^2+z^2 = 2
                    \end{cases}
                \end{displaymath}
                Veiem que $x = \frac{1}{2\lambda}$ (analogament amb $y\text{ i }z$), llavors
                remplaçem a l'última equació, tal que
                $$3\left(\frac{1}{2\lambda}\right)^2 = 2 \Rightarrow \frac{1}{\lambda^2} = \frac{8}{3} \Rightarrow \lambda = \frac{\sqrt{6}}{4}$$
                I amb aixó, $x = y = z = \frac{\sqrt{6}}{3}$ és un màxim o un mínim.
            \end{solucio}
            \item Trobeu el màxim de \( \sum_{k=1}^n x_k \) sabent que \( \sum_{k=1}^n x_k^2 \leq 1 \).\\
            \begin{solucio}
                La desigualtat de Cauchy-Schwarz ens diu que per qualsevol conjunt de nombres reals \( x_1, x_2, \ldots, x_n \):
                \[
                \left( \sum_{k=1}^n x_k \right)^2 \leq n \sum_{k=1}^n x_k^2
                \]
                Usant aquesta desigualtat amb la restricció donada \( \sum_{k=1}^n x_k^2 \leq 1 \), obtenim:
                \[
                \left( \sum_{k=1}^n x_k \right)^2 \leq n \cdot 1 = n
                \]
                Per tant:
                \[
                \sum_{k=1}^n x_k \leq \sqrt{n}
                \]
            \end{solucio}
            \item Per quins valors d'$\alpha$ el camp \( \mathbf{F}(x, y) = (4xy^2 - 3y^2, 4x^2y - \alpha xy - 4y) \) és conservatiu? En aquests casos, calculeu la funció potencial.\\
            \begin{solucio}
                Sabem que per fer que un camp sigui conservatiu s'ha de cumplir que $\frac{\partial p}{\partial y} = \frac{\partial q}{\partial x}$
                on $p(x,y) = 4xy^2 - 3y^2\,q(x,y) = 4x^2y-\alpha xy - 4y$.
                Llavors, apliquem aquest teorema
                $$
                8xy-6y = 8xy-\alpha y \Rightarrow \alpha = 6. 
                $$
                En aquest cas, calculem la funció potencial
                $$
                \int 4xy^2-3y^2\;dx = y^2\int 4x-3\;dx = 2x^2y^2-3xy^2+y^2C_x
                $$
                $$
                \int 4x^2y-6xy-4y\;dy = 2x^2y^2-3xy^2-2y^2+C_y
                $$
                Llavors, tenim el seguent
                $$
                2x^2y^2-3xy^2+y^2C_x = 2x^2y^2-3xy^2-2y^2+C_y
                $$
                I aixó resulta en $C_x = -2\text{ i }C_y = 0$.
                Llavors la funció potencial és $f(x,y) = 2x^2y^2-3xy^2-2y^2$
            \end{solucio}
        \end{enumerate}
    \end{exercici}
    \newpage
    \begin{exercici}
        \begin{enumerate}[label=\alph*)]
            \item Apliqueu el Teorema de Fubini per calcular \[ \int_0^1 \int_{y^{1/3}}^y e^{x^2} \, dx \, dy \].
            \begin{solucio}
                Primer veiem que \[ \int_0^1 \int_{y^{1/3}}^y e^{x^2} \, dx \, dy = \int_0^1 \int_{x^3}^x e^{x^2} \, dy \, dx \]
                Ara, calculem aixó que és més fàcil
                $$\int_0^1 \int_{x}^{x^3} e^{x^2} \, dy \, dx = \int_0^1 e^{x^2} \int_{x}^{x^3} 1\, dy \, dx = \int_0^1 e^{x^2} \left(x^3-x\right)$$
                I aquesta integral es soluciona fàcilment amb $t = x^2$
                $$\int_0^1 e^{t}x \left(t-1\right)\, \frac{1}{2x}dt = \frac{1}{2}\int_0^1 e^{t} \left(t-1\right)\, dt = \frac{1}{2} e^{t} \left(t-2\right)\bigg\rvert_{t=0}^{t=1} = 1-\frac{e}{2}$$
            \end{solucio}
            \item Sigui \( \Omega = \{ (x, y) \in \mathbb{R}^2 : x^2 + y^2 \leq 1, x > 0, -x \leq y \leq x \} \). Calculeu \[ \int_\Omega \sin(x^2 + y^2) \, dx \, dy \]
            \begin{solucio}
                Aquest problema és més fàcil si es fa un canvi de variable tal que $\Omega = \left\{\left(r, \theta\right): r \leq 1, -\frac{\pi}{4} \leq \theta \leq \frac{\pi}{4}\right\}$. Llavors tenim
                $$\int_{0}^{1} \int_{-\frac{\pi}{4}}^{\frac{\pi}{4}} \sin{\left(r^2\right)}r\, d\theta \, dr = \frac{\pi}{2}\int_{0}^{1} \sin{\left(r^2\right)}r \, dr$$
                I aquesta integral al resoldre dona
                $$\frac{1}{2}\pi\sin^2\left(\frac{1}{2}\right)$$
            \end{solucio}
            \item Calculeu el volum de \( \{ (x, y, z) \in \mathbb{R}^3 : x^2 + y^2 + z^2 \leq 4, x^2 + y^2 > z^2 \} \).\\
            \begin{solucio}
                Aquesta integral es resol més fàcilment si fem $\Omega = \left\{\left(r, \theta, z\right): r^2+z^2 \leq 4, r^2 > z^2\right\}$
                \begin{displaymath}
                    \iiint_\Omega r dr\,dz\,d\theta = \int_{0}^{2\pi} \int_{-2}^2 \int_{\left\lvert z\right\rvert}^{\sqrt{4-z^2}} 1\,dr\, dz\, d\theta
                \end{displaymath}
                Llavors, calculem
                \begin{displaymath}
                    \int_{0}^{2\pi} \int_{-2}^2 \int_{\left\lvert z\right\rvert}^{\sqrt{4-z^2}} 1\,dr\, dz\, d\theta = \int_{0}^{2\pi} \int_{-2}^2 \sqrt{4-z^2} - \left\lvert z\right\rvert\, dz\, d\theta = \int_{0}^{2\pi} 2\left(\pi-2\right)\, d\theta = 4\pi\left(\pi-2\right)
                \end{displaymath}
            \end{solucio}
            \item Enuncieu el Teorema de Green. Utilitzant el camp \( \mathbf{F}(x, y) = (-y, x) \) deduïu una fórmula per trobar l'àrea de l'interior d'una corba tancada simple a \( \mathbb{R}^2 \).\\
            \begin{solucio}
                El teorema de Green diu que si tenim una corva tancada simple orientada positivament ($\gamma$)
                i un camp $\mathbf{X} = \left(P, Q\right)$, aleshores
                $\int_\gamma \mathbf{X} = \iint_\Omega \left(\frac{\partial Q}{\partial x} - \frac{\partial P}{\partial y}\right)\;dx\;dy$, on $\Omega$ és l'area interior de $\gamma$.\\
                Fent servir $\mathbf{F}$, deduim que $\int_{\gamma} \left(-y, x\right) = \iint_\Omega 2 \;dx\;dy \Rightarrow \frac{1}{2}\int_{\gamma} \left(-y, x\right) = \iint_\Omega 1 \;dx\;dy$
            \end{solucio}
        \end{enumerate}
    \end{exercici}
\end{document}