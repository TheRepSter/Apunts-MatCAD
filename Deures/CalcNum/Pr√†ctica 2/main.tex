\documentclass[a4paper, 12pt]{article}
\usepackage{deuresutils}

\title{Pràctica 2}
\asignatura{Càlcul numèric}
\author{Eduardo Pérez Motato}
\niu{NIU: 1709992}
\date{01/03/2024}

\begin{document}
    \makeheader

    \begin{exercici}
        Considerar l'equació polinòmica
        \begin{equation}
            x^3 = x + 40
            \label{eq:1}
        \end{equation}
        i la fórmula para el càlcul de la seva arrel real (que s'obté a partir de les fórmules de
        Cardano)
        \begin{displaymath}
            \alpha = \sqrt[3]{20 + \frac{1}{9}\sqrt{32397}} + \sqrt[3]{20-\frac{1}{9}\sqrt{32397}}
        \end{displaymath}
        \begin{enumerate}[label=\alph*)]
            \item Comprovar que es produeix error de cancel·lació al evaluar en precisió simple i
            doble precisió l'expressió de l'arrel real de l'equació anterior.\\
            \begin{solucio}
                Podem veure com en fer el càlcul de l'arrel mitjançants les fórmules de Cardano, a
                \verb|Pr2Ex1a.c| en ambdues precisions dona prop de $0$, però no prou per a aquestes
                precisions. 
            \end{solucio}
            \item Aplicar el mètode de Newton a la funció
            \begin{displaymath}
                f\left(x\right) = x^3-x-40 
            \end{displaymath}
            començant amb $x_0 = 2$ fent servir precisió simple i doble. Obtenir una aproximació de
            $8$ i $15$ decimals correctes respectivament.\\
            \begin{solucio}
                En aplicar Newton per ambdues precisions a \verb|Pr2Ex1b.c|, obtenim precisions tal
                que és prou a prop de $0$ per les precisions, ja que tenen $8$ i $15$ decimals
                correctes per a precisió simple i doble respectivament.
            \end{solucio}
            \item Considerar l'equació polinòmica, $x^3=x+400$\\
            Obtenir una fórmula de Cardano per al càlcul de l'arrel real, $\beta$. Comprovar que
            aquesta arrel compleix
            \begin{displaymath}
                2 \leq \beta \leq 8
            \end{displaymath}
            Comprovar l'error de cancel·lació calculant la fórmula explicita en precisió doble.\\
            \begin{solucio}
                Tenint $x^3 -x -400 = 0$, agafem la fórmula de Cardano\footnote[1]{Agafada de \href{https://www.uv.es/~ivorra/Libros/Ecuaciones.pdf}{aquest PDF}.}
                i al remplaçar tenim
                \begin{displaymath}
                    \sqrt[3]{-\frac{-400}{2}+\sqrt{\left(\frac{-400}{2}\right)^2+\left(\frac{-1}{3}\right)^3}}+\sqrt[3]{-\frac{-400}{2}-\sqrt{\left(\frac{-400}{2}\right)^2+\left(\frac{-1}{3}\right)^3}}
                \end{displaymath}
                simplifiquem i tenim
                \begin{displaymath}
                    \sqrt[3]{200+\frac{1}{3}\sqrt{3239997}}+\sqrt[3]{200-\frac{1}{3}\sqrt{3239997}}
                \end{displaymath}
            \end{solucio}
            Si això ho possem en \verb|c|, en aquest cas al programa \verb|Pr2Ex1c.c|, ens dona $\approx 7.413$,
            un valor que, efectivament, està entre $2$ i $8$. Això no obstant, tè un error molt alt.
            Aplicar els següents mètodes iteratius per obtenir $15$ decimals correctes de l'arrel.
            \begin{enumerate}
                \item Mètode de la bisecció partint de l'interval $\left[2,8\right]$\\
                \begin{solucio}
                    En aquest cas arriba en $54$ iteracions a la soluciò, i tè un error suficientment
                    baix.
                \end{solucio}
                \item Mètode de Newton partint del pivot $x_0 = 2$.\\
                \begin{solucio}
                    En aquest cas arriba en $11$ iteracions a la soluciò, arribant al mateix valor
                    que amb el metode de la bisecció.
                \end{solucio}
            \end{enumerate}
            Comparar l'ordre de convergència numèrica i determina una estrategia per calcular les
            arrels d'aquest tipus d'equacions.\\
            \begin{solucio}
                Com ja ben savem, la bisecció tè un ordre de convergència de com minim $1$ i Newton
                com minim $2$. Això es veu reflectit en què pel metode de bisecció és necesiten $54$
                iteracions i en el de Newton $11$. Una estrategia simple per calcular les arrels
                d'aquest itpus d'equacions és obtenir una aproximació mitjançants Cardano i aquest
                resultat fer-ho servir com a llavor per Newton. 
            \end{solucio}
        \end{enumerate}
    \end{exercici}
    \newpage
    \begin{exercici}
        Sigui l'equació $f\left(x\right) = 0$ amb $f\left(x\right)$ contínuament derivable, $x^*$
        una arrel simple, $f\left(x^*\right) = 0$, amb $f'\left(x\right) = 0$ en un entorn de $x^*$.
        Considerar la iteració
        \begin{displaymath}
            x_{k+1} = x_k - b_kf\left(x_k\right) 
        \end{displaymath}
        on
        \begin{displaymath}
            b_{k+1} = b_k\left(2-f'\left(x_{k+1}\right)b_k\right)
        \end{displaymath}
        partint d'un pivot $x_0$ suficientment pròxim a $x^*$ amb $b_0 = \frac{1}{f'\left(x_0\right) }$.
        \begin{enumerate}[label=\alph*)]
            \item Aplicar la iteració a l'equació \eqref{eq:1}, tomant $b_0 = \frac{1}{3x^2_0-1}$.\\
            Estudiar l'ordre de convergència numèric: suggeriment, calcular $e_k = \left\lvert x_k-x_{k+1}\right\rvert$
            i compara els cocients $\frac{e_k}{e_{k-1}}, \frac{e_k}{\left(e_{k-1}\right)^2}, \dots$.\\
            \begin{solucio}
                A \verb|Pr2Ex2a.c| ho tenim aplicat. Podem veure que ho troba en tan sols $5$
                iteracions en agafar $4$ com a llavor. Si provem de tindre $2$ com a llavor, veiem
                que no ho trobarà, això perquè $2$ no és suficientment proxim a $4$. Aquest metode
                dona un valor que al evaluar-ho dona exactament $0$. A falta de proves, podem
                veure que l'ordre de convergència és proxim a $2$, ja que en fer $\frac{\left\lvert x_k-x_{k+1}\right\rvert }{\left\lvert x_{k-1}-x_k\right\rvert ^p}$
                tindrem que
                \begin{displaymath}
                    \text{Si }p =
                    \begin{cases}
                        1 & \Rightarrow\text{tendeix a }0.\\
                        2 & \Rightarrow\text{no sembla tendir a res, falten iteracions}.\\
                        3 & \Rightarrow\text{tendeix a } \infty.\\
                    \end{cases}
                \end{displaymath}
            \end{solucio}
        \end{enumerate}
    \end{exercici}
    \newpage
    \begin{exercici}
        \textbf{(OPCIONAL)}\\
        Sigui l'equació $f\left(x\right) = 0$ amb $f\left(x\right)$ completament derivable, $x^*$
        una arrel simple, $f\left(x^*\right) = 0$, i $f'\left(x\right) \neq 0$ en un entorn de $x^*$.
        Considerar la iteració (conocida com mètode de Halley),
        \begin{displaymath}
            x_{k+1} = x_k - \frac{2f\left(x_k\right)f'\left(x_k\right)}{2\left(f'\left(x_k\right)\right)^2-f\left(x_k\right)f''\left(x_k\right)}
        \end{displaymath}
        \begin{enumerate}[label=\alph*)]
            \item Aplicar la iteració a l'equació polinòmica del Problema 1, $x^3=x+400$.\\
            \begin{solucio}
                Fet al \verb|Pr2Ex3a.c|. Veiem que troba la solució en $6$ iteracions, considerablement més rapid quant a
                nombre d'iteracions. També retorna el mateix nombre que en bisecció i Newton.
            \end{solucio}
            \item Comprovar numericament que la convergència és d'ordre 3.\\
            \begin{solucio}
                Fet al \verb|Pr2Ex3b.c|, veiem que fent $\frac{\left\lvert x_k-x_{k+1}\right\rvert }{\left\lvert x_{k-1}-x_k\right\rvert ^p}$,
                com ja vam fer a l'exercici 2a tenim que quan $p = 4$ tendeix a $\infty$, mentre que
                $p = 3$ sembla tendir a un nombre $0.01$ i $p = 2$ tendeix a $0$, per allò és d'ordre
                de convergència $3$. 
            \end{solucio}
        \end{enumerate}
    \end{exercici}
\end{document}