\documentclass[a4paper, 12pt]{article}
\usepackage{deuresutils}

\title{Pràctica 2}
\asignatura{Càlcul numèric}
\author{Eduardo Pérez Motato}
\niu{NIU: 1709992}
\date{01/03/2024}

\begin{document}
    \makeheader

    \begin{exercici}
        Considerar l'equació polinòmica
        \begin{equation}
            x^3 = x + 40
            \label{eq:1}
        \end{equation}
        i la fórmula para el càlcul de la seva arrel real (que s'obté a partir de les fórmules de
        Cardano)
        \begin{displaymath}
            \alpha = \sqrt[3]{20 + \frac{1}{9}\sqrt{32397}} + \sqrt[3]{20-\frac{1}{9}\sqrt{32397}}
        \end{displaymath}
        \begin{enumerate}[label=\alph*)]
            \item Comprovar que es produeix error de cancel·lació al evaluar en precisió simple i
            doble precisió l'expressió de l'arrel real de l'equació anterior.\\
            \begin{solucio}
                skibidi
            \end{solucio}
            \item Aplicar el mètode de Newton a la funció
            \begin{displaymath}
                f\left(x\right) = x^3-x-40 
            \end{displaymath}
            començant amb $x_0 = 2$ fent servir precisió simple i doble. Obtenir una aproximació de
            $8$ i $15$ decimals correctes respectivament.\\
            \begin{solucio}
                crotolamo
            \end{solucio}
            \item Considerar l'equació polinòmica, $x^3=x+400$\\
            Obtenir una fórmula de Cardano per al càlcul de l'arrel real, $\beta$. Comprovar que
            aquesta arrel compleix
            \begin{displaymath}
                2 \leq \beta \leq 8
            \end{displaymath}
            \begin{solucio}
                makeheader
            \end{solucio}
            Comprovar l'error de cancel·lació calculant la fórmula explicita en precisió doble.\\
            Aplicar els següents mètodes iteratius per obtenir $15$ decimals correctes de l'arrel.
            \begin{enumerate}
                \item Mètode de la bisecció partint de l'interval $\left[2,8\right]$\\
                \begin{solucio}
                    La revolución industrial
                \end{solucio}
                \item Mètode de Newton partint del pivot $x_0 = 2$.\\
                \begin{solucio}
                    Y sus consecuencias
                \end{solucio}
            \end{enumerate}
            Comparar l'ordre de convergència numèrica i determina una estrategia per calcular les
            arrels d'aquest tipus d'equacions.\\
            \begin{solucio}
                Desastre para la humanidad
            \end{solucio}
        \end{enumerate}
    \end{exercici}
    \newpage
    \begin{exercici}
        Sigui l'equació $f\left(x\right) = 0$ amb $f\left(x\right)$ contínuament derivable, $x^*$
        una arrel simple, $f\left(x^*\right) = 0$, amb $f'\left(x\right) = 0$ en un entorn de $x^*$.
        Considerar la iteració
        \begin{displaymath}
            x_{k+1} = x_k - b_kf\left(x_k\right) 
        \end{displaymath}
        on
        \begin{displaymath}
            b_{k+1} = b_k\left(2-f'\left(x_{k+1}\right)b_k\right)
        \end{displaymath}
        partint d'un pivot $x_0$ suficientment pròxim a $x^*$ amb $b_0 = \frac{1}{f'\left(x_0\right) }$.
        \begin{enumerate}[label=\alph*)]
            \item Aplicar la iteració a l'equació \eqref{eq:1}, tomant $b_0 = \frac{1}{3x^2_0-1}$.\\
            Estudiar l'ordre de convergència numèric: suggeriment, calcular $e_k = \left\lvert x_k-x_{k+1}\right\rvert$
            i compara els cocients $\frac{e_k}{e_{k-1}}, \frac{e_k}{\left(e_{k-1}\right)^2}, \dots$.\\
            \begin{solucio}
                A mi ke me dices jaja salu2
            \end{solucio}
        \end{enumerate}
    \end{exercici}
    \newpage
    \begin{exercici}
        \textbf{(OPCIONAL)}\\
        Sigui l'equació $f\left(x\right) = 0$ amb $f\left(x\right)$ completament derivable, $x^*$
        una arrel simple, $f\left(x^*\right) = 0$, i $f'\left(x\right) \neq 0$ 
    \end{exercici}
    \begin{solucio}
        Lenin tenía razón.
    \end{solucio}
\end{document}