\documentclass[a4paper, 12pt]{article}
\usepackage{deuresutils}

\title{Pràctica 1}
\asignatura{Càlcul numèric}
\author{Eduardo Pérez Motato}
\niu{NIU: 1709992}
\date{31/03/2024}

\begin{document}
    \makeheader

    \begin{exercici}
        Considerar la funció
        \begin{equation}
            f\left(x\right) = 
            \begin{cases}
                \frac{1-\cos{x}}{x^2} & \text{si } x \neq 0\\
                \frac{1}{2} & \text{si } x = 0
            \end{cases}
            \label{eq:1}
        \end{equation}
        Volem avaluar $f(x_0)$ per al valor $x_0 = 1.2 \times 10^{-5}$.
        \begin{enumerate}[label=\alph*)]
            \item Escriure dos \textbf{programes en C}, un en \textbf{precisió simple} i un altre en
            \textbf{precisió doble} que avaluïn la funció $f\left(x\right)$.\\
            Calcular per cadascun dels programes el valor $f\left(x_0\right)$.\\
            Comparar i comentar els resultats.\\
            \begin{solucio}
                A \verb|Pr1Ex1a.c| creem dues funcions, \verb|fsimp| amb precisió float i \verb|fdoble|
                amb precisió doble. En avaluar $x_0$, en el cas del simple retorna $0$, i en el cas
                del doble retorna $\approx 0.4999997$, el que s'assembla més al valor real (ja que
                $\lim\limits_{x \to 0} f\left(x\right) = \frac{1}{2}$).
            \end{solucio}
            \item Reescriure la funció $f\left(x\right)$ fent servir fórmules trigonomètriques de
            forma que es redueixi l'error que es produeix fent servir la expressiò \eqref{eq:1}.\\
            \begin{solucio}
                A \verb|Pr1Ex1b.c| reemplacem $1-\cos{x}$ per $2\sin^2{\left(\frac{x}{2}\right)}$ en
                ambdues funcions, llavors, podem veure com les dues funcions abans creades tenen
                millor precisió.
            \end{solucio}
            \item Discutir l'observat en aquest exercici.\\
            \begin{solucio}
                Podem veure com, en aquest cas, $1-\cos{x}$ perd precisió quan $x$ tendeix a $0$.
                Per allò hem de fer servir una substitució trigonomètrica que tindrà millor precisió.
            \end{solucio}
        \end{enumerate}
    \end{exercici}
    \newpage

    \begin{exercici}
        Equació quadràtica.\\
        La solució d'una equació quadràtica amb coeficients reals,
        \begin{displaymath}
            ax^2+bx+c=0 \hspace{5em} a \neq 0
        \end{displaymath}
        s'obté a partir de la expressiò
        \begin{equation}
            x = \frac{-b\pm\sqrt{b^2-4ac}}{2a}
            \label{eq:2}
        \end{equation}
        Suposant que $a > 0$ i $b^2 > 4ac$.
        \begin{enumerate}[label=\alph*)]
            \item Escriure dos \textbf{programes en C}, un en \textbf{precisió simple} i un altre en
            \textbf{precisió doble} que calculin la solució d'una equació quadràtica mitjançant \eqref{eq:2}.\\
            \begin{solucio}
                A \verb|Pr1Ex2a.c| estan escrites dues funcions, \verb|fquad| en precisió float i \verb|quad|
                en precisió doble. Al \verb|main|, hi ha un exemple amb $a=1, b=2$ i $c=1$, donant
                el resultat correcte de $-1$ com a solució doble. 
            \end{solucio}
            \item Comprovar que si $b^2 >> 4ac$ una de les dues fórmules per al càlcul de les arrels
            amb \eqref{eq:2} produeix resultats contaminats amb error de cancel·lació.\\
            \begin{solucio}
                Com podem veure a \verb|Pr1Ex2b.c|, on calculem les arrels de $a=1, b=40000$ i $c=1$.
                En aquest cas, tenim $b^2 = 40000^2$ que és, òbviament, molt més gran que $4ac = 4$.
                En calcular, veiem que la funció amb precisió float dona $0$ i $-40000$ com a
                solució, mentre en doble dona $\approx-2.5\times10^{-5}$ i $ -39999.999975$, un
                resultat força més creïble. 
            \end{solucio}
            \item Proposa un procediment alternatiu per al càlcul de les arrels que eviti l'error de
            cancel·lació.\\
            \begin{solucio}
                Sabem que \eqref{eq:2} és equivalent a
                \begin{displaymath}
                    \frac{-b\pm\sqrt{b^2-4ac}}{2a} \underbrace{\frac{-b \mp \sqrt{b^2-4ac}}{-b \mp \sqrt{b^2-4ac}}}_{=1} = \frac{b^2-\left(b^2-4ac\right)}{2a\left(-b\mp\sqrt{b^2-4ac}\right)}
                \end{displaymath}
                i si continuem simplificant, tenim
                \begin{equation}
                    \frac{2c}{-b \mp \sqrt{b^2-4ac}}
                    \label{eq:3}
                \end{equation}
                Ara, amb \eqref{eq:3}, canviem el programa anterior i fem els canvis a \verb|Pr1Ex2c.c|
                per al cas on l'arrel és positiva, ja que és allí on hi ha error de cancel·lació. En
                cas de posar-ho també quan l'arrel es negativa crearia un altre error de 
                cancel·lació fatal que produirà $-\infty$. Amb els canvis fets, al evaluar-ho dona
                un resultat més correcte en cadascun. 
            \end{solucio}
            \item Construir exemples numèrics on el càlcul de les arrels en simple i doble precisió
            proporcionin diferències significatives en exactitud fent servir \eqref{eq:2} i el
            procediment que has proposat.\\
            \begin{solucio}
                Un exemple numèric molt bo i il·lustratiu és el que hi ha a \verb|Pr1Ex2b.c|.
                En el cas $a = 1, b = 40000, c = 1$, podem observar com \eqref{eq:2} (implementat a
                \verb|Pr1Ex2a.c| i \verb|Pr1Ex2b.c|) dona, en el cas float, $0$ i $-40000$. En el programa
                fet amb \eqref{eq:3}, en cas float, dona $-2.4999999 \times 10^{-5}$
                i $-40000$. 
            \end{solucio}
        \end{enumerate}
    \end{exercici}
    
    \begin{exercici}
        Càlcul de la variància mostral.\\
        En estadística la variància mostral de $n$ nombres es defineix com
        \begin{equation}
            s^2_n = \frac{1}{n-1}\sum\limits_{i=1}^n\left(x_i-\overline{x}\right)^2,\hbox{ on } \overline{x} = \frac{1}{n}\sum\limits_{i=1}^n x_i
            \label{eq:4}
        \end{equation}
        Una fórmula alternativa equivalent que fa servir un nombre d'operacions similars és
        \begin{equation}
            s^2_n = \frac{1}{n-1}\left(\sum\limits_{i=1}^{n} x^2_i - \frac{1}{n}\left(\sum\limits_{i=1}^{n}x_i\right)^2\right)
            \label{eq:5}
        \end{equation}
        Aquesta última fórmula pot sofrir error de cancel·lació!
        \begin{enumerate}[label=\alph*)]
            \item Escriure \textbf{programes en C en simple i doble precisió} que calculin la
            variància mostral amb les mateixes fórmules on l'input sigui un vector de nombres reals
            i l'output sigui la variància mostral.\\
            \begin{solucio}
                A \verb|Pr1Ex3a.c| es creen 4 funcions, dues amb \eqref{eq:4} on una és simple i
                l'altre és double; i altres dues amb \eqref{eq:5}, una en simple i altre en double.
                Aquest programa demana el nombre de components del vector i després demana cadascun 
                dels components. Finalment, calcula la variància mostral i mostra en cadascun dels
                casos l'output de la funció en qüestió.
            \end{solucio}
            \item Considerar el vector $x = \left(10000, 10001, 10002\right)$ i calcular la
            variància amb els programes generats. Analitzar les discrepàncies.\\
            \begin{solucio}
                Amb aquest vector, totes les funcions donen $1$ menys la simple amb \eqref{eq:5},
                que dona $0$. Això és degut al fet que en aquest cas hi ha un error de cancel·lació.
            \end{solucio}
            \item Construir dos exemples de vectors de dimensió gran (almenys 100 components) on
            aquestes discrepàncies siguin més evidents.\\
            \begin{solucio}
                A \verb|Pr1Ex3c.c| he posat els exemples dels vectors $\left(10000, 10001, \dots, 10099\right)$ i $(\underbrace{10000}_\text{33 vegades}, \underbrace{10001}_\text{34 vegades},\underbrace{10002}_\text{33 vegades})$.
                En el primer cas podem veure un error de cancel·lació de $\Delta x \approx 16.8384$
                mentre al segón hi ha $\Delta x \approx 40.7070$, amb un error relatiu $\approx 61.0606$.
            \end{solucio}
            \item Discutir les diferències en els resultats.\\
            \begin{solucio}
                Podem veure que, en el cas de la fórmula amb possible cancel·lació \eqref{eq:5} en
                simple hi ha una alta probabilitat de tindre un error de cancel·lació, mentre 
                ambdues fórmules són aproximadament igual d'eficients, llavors no dona benefici
                fer-la servir.
            \end{solucio}
        \end{enumerate}
    \end{exercici}
    \newpage

    \begin{exercici}
        Suma d'una sèrie.\\
        És conegut que la sèrie dels recíprocs dels quadrats dels nombres naturals convergeix i la
        seva suma és $\frac{\pi^2}{6}$
        \begin{equation}
            S = \sum\limits_{k=1}^\infty\frac{1}{k^2} = \frac{\pi^2}{6} \approx 1.644934066848226\dots
            \label{eq:6}
        \end{equation}
        Volem calcular aproximadament la suma $S$ sumant termes (sumes parcials) de la sèrie i
        establirem dues estratègies per programar-les en C en \textbf{simple i doble precisió.}\\
        \begin{enumerate}[label=\alph*)]
            \item Escriure programes C que calculin la suma dels termes de la sèrie $S$ en ordre
            creixent fins a un terme màxim $\left(5000, 10000, \dots\right)$ on les dades siguin el
            nombre de termes a sumar.\\
            \begin{solucio}
                \underline{Suposant que quan es refereix en ordre creixent es refereix a $k$ creixent.}\\
                A \verb|Pr1Ex4a.c| es creen dues funcions, una en simple i l'altra en doble, que
                calculen en ordre creixent \eqref{eq:6}, sent més eficient la doble.
            \end{solucio}
            \item Escriure programes en C (doble i simple precisió) que sumin els termes de la sèrie
            en ordre decreixent.\\
            \begin{solucio}
                A \verb|Pr1Ex4b.c| es creen dues funcions, una en simple i l'altra en doble, que
                calculen en ordre decreixent \eqref{eq:6}, sent més o menys igual d'eficients però
                la simple amb menor precisió.
            \end{solucio}
            \item Comparar els resultats anteriors amb el valor exacte i justifica els diferents
            resultats.\\
            \begin{solucio}
                Podem veure que el cas d'ordre decreixent \underline{de $k$} millora el càlcul.
                Això a causa de les propietats dels punts flotants en fer la suma els bits més
                petits del nombre inferior es perden, en aquest cas quan $k$ decreix, $\frac{1}{k^2}$
                creix i això fa que al primer sumar els nombres més petits hi hagi menys error.  
            \end{solucio}
            \item Proporcionar una fórmula alternativa que es comporti millor que \eqref{eq:6}.\\
            \begin{solucio}
                Buscant per internet vaig trobar la seguent igualtat
                \begin{equation}
                    \frac{\pi}{6} = \sum\limits_{k=1}^{\infty} \frac{-120+329k+568k^2}{k(1+k)(1+2k)(1+4k)(3+4k)(5+4k)}
                    \label{eq:7}
                \end{equation}
                llavors, si fem servir això podem arrivar fàcilment a $\frac{\pi^2}{6}$ mitjançant
                \eqref{eq:7} de la següent manera
                \begin{equation}
                    \frac{\pi^2}{6} = 6\left(\sum\limits_{k=1}^{\infty} \frac{-120+329k+568k^2}{k(1+k)(1+2k)(1+4k)(3+4k)(5+4k)}\right)^2
                    \label{eq:8}
                \end{equation}
                això programat en \verb|Pr1Ex4d.c|, podem veure que en el cas doble (perqué en el
                simple no es pot demanar més precisió) l'error és de $-2.22\times10^{-16}$, és a dir,
                un error molt petit. 
            \end{solucio}
        \end{enumerate}
    \end{exercici}
    \newpage

    \begin{exercici}
        Escriure conclusions sobre l'observat i après en aquesta pràctica. Extensió màxima de mitja
        pàgina. 
    \end{exercici}
    \begin{solucio}
        A l'hora de fer càlculs hem de tindre molta cura amb la precisió de les funcions que
        utilitzem, tant en el tipus de dada rebuda i retornada, com els limits de precisió de la
        propia funció (com que per exemple si fem $1.2 \times 10^{-5}$ el cosinus no dona un valor
        correcte i en canvi el sinus si). L'ordre de les operacions també afecta i és millor sumar o
        restar sempre, a ser possible, en el mateix ordre de magnitud. 
    \end{solucio}
\end{document}