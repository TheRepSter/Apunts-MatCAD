\documentclass[a4paper, 12pt]{article}
\usepackage{deuresutils}
\usepackage{lipsum}

\title{Lliurament 3}
\asignatura{Àlgebra Lineal}
\author{Eduardo Pérez Motato}
\niu{NIU: 1709992}
\date{12/01/2024}

\begin{document}
    \makeheader
    \begin{center}
        En el meu cas, $B := 5$, ja que la unitat del meu NIU és 2.
    \end{center}
    \section{Buscar bases de subespais vectorials reals finits generats.}
    \subsection{Considera l'espai vectorial $\mathbb{R}^4$. Sigui $F=<(B, B, B-1, 0), (3, 2, 1, 0), (1, 2, 3, 0), (4, 3, 2, 0)>$
    i $G = \{(x, y, z, t) \in \mathbb{R}^4\text{ }|\text{ }x + 2y - 3z + 4t = 0\}$.}
    \begin{exercici}
        Trobeu una base de $F$ i la dimensió de $F$. Comproveu que el vector $v = (2022, 2022, 0, 0)$
        és de $F$ i doneu les coordenades del vector $v$ en la base triada de $F$.
    \end{exercici}
    \begin{solucio}
        Per trobar una base de $F$, posem el seu sistema generador en forma matricial i reduïm aquesta
        matriu:
        \begin{displaymath}
            \left(
                \begin{array}{cccc}
                    5 & 5 & 4 & 0\\
                    3 & 2 & 1 & 0\\
                    1 & 2 & 3 & 0\\
                    4 & 3 & 2 & 0
                \end{array}
            \right)
            \sim
            \left(
                \begin{array}{cccc}
                    1 & 2 & 3 & 0\\
                    5 & 5 & 4 & 0\\
                    3 & 2 & 1 & 0\\
                    4 & 3 & 2 & 0
                \end{array}
            \right)
            \sim
            \left(
                \begin{array}{cccc}
                    1 & 2 & 3 & 0\\
                    0 & -5 & -11 & 0\\
                    0 & -4 & -8 & 0\\
                    0 & -5 & -10 & 0
                \end{array}
            \right)
            \sim
        \end{displaymath}
        \begin{displaymath}
            \sim
            \left(
                \begin{array}{cccc}
                    1 & 2 & 3 & 0\\
                    0 & -4 & -8 & 0\\
                    0 & -5 & -11 & 0\\
                    0 & -5 & -10 & 0
                \end{array}
            \right)
            \sim
            \left(
                \begin{array}{cccc}
                    1 & 2 & 3 & 0\\
                    0 & 1 & 2 & 0\\
                    0 & -5 & -11 & 0\\
                    0 & -5 & -10 & 0
                \end{array}
            \right)
            \sim
            \left(
                \begin{array}{cccc}
                    1 & 2 & 3 & 0\\
                    0 & 1 & 2 & 0\\
                    0 & 0 & -1 & 0\\
                    0 & 0 & 0 & 0
                \end{array}
            \right)
        \end{displaymath}
        Llavors, $((1, 2, 3, 0), (0, 1, 2, 0), (0, 0, 1, 0))$ és una base de $F$, sent $\dim{F} = 3$.
        Comprovem que $v$ és de $F$:
        \begin{displaymath}
            \left(
                \begin{array}{cccc|cccc}
                    1 & 2 & 3 & 0 & 1 & 0 & 0 & 0\\
                    0 & 1 & 2 & 0 & 0 & 1 & 0 & 0\\
                    0 & 0 & -1 & 0 & 0 & 0 & 1 & 0\\
                    2022 & 2022 & 0 & 0 & 0 & 0 & 0 & 1
                \end{array}
            \right)
            \sim
            \left(
                \begin{array}{cccc|cccc}
                    1 & 2 & 3 & 0 & 1 & 0 & 0 & 0\\
                    0 & 1 & 2 & 0 & 0 & 1 & 0 & 0\\
                    0 & 0 & -1 & 0 & 0 & 0 & 1 & 0\\
                    0 & -2022 & -6066 & 0 & -2022 & 0 & 0 & 1
                \end{array}
            \right)
        \end{displaymath}
        \begin{displaymath}
            \sim
            \left(
                \begin{array}{cccc|cccc}
                    1 & 2 & 3 & 0 & 1 & 0 & 0 & 0\\
                    0 & 1 & 2 & 0 & 0 & 1 & 0 & 0\\
                    0 & 0 & -1 & 0 & 0 & 0 & 1 & 0\\
                    0 & 0 & -2022 & 0 & -2022 & 2022 & 0 & 1
                \end{array}
            \right)
            \sim
            \left(
                \begin{array}{cccc|cccc}
                    1 & 2 & 3 & 0 & 1 & 0 & 0 & 0\\
                    0 & 1 & 2 & 0 & 0 & 1 & 0 & 0\\
                    0 & 0 & -1 & 0 & 0 & 0 & 1 & 0\\
                    0 & 0 & 0 & 0 & -2022 & 2022 & -2022 & 1
                \end{array}
            \right)
        \end{displaymath}
        Com podem observar, en fer la matriu reduïda el vector $v$ és linealment dependent, per això
        pertany a $F$. A la part ampliada, tenim les coordenades (negatives) de $v$ respecte a la
        base abans calculada. És a dir, $v = 2022(1, 2, 3, 0) -2022(0, 1, 2, 0) + 2022(0, 0, -1, 0)$.
        $v$ en coordenades de la base és $(2022, -2022, 2022, 0)$.
    \end{solucio}
    \begin{exercici}
        És el vector de $F: (B, B, B-1, 0)$, combinació lineal dels vectors $(3, 2, 1, 0), (1, 2, 3, 0), (4, 3, 2, 0)$?
        En cas afirmatiu trobeu-ne una combinació lineal (és única aquesta combinació lineal?).
    \end{exercici}
    \begin{solucio}
        Com ja hem calculat a l'anterior exercici, $(5, 5, 4, 0)$ no és combinació lineal dels
        anteriors. Això ja que la tercera filera (on termina el vector $(5, 5, 4, 0)$) no termina reduïda.
    \end{solucio}
    \begin{exercici}
        Amplieu la base de $F$ a una base de $\mathbb{R}^4$.
    \end{exercici}
    \begin{solucio}
        Per ampliar la base de $F$ a una base de $\mathbb{R}^4$ és trivial, ja que només cal posar el
        vector $(0, 0, 0, 1)$ per fer-la base de $\mathbb{R}^4$.
    \end{solucio}
    \begin{exercici}
        Comproveu que el vector $w = (3, 3, 3, 0)$ és de $G$. Trobeu una base de $G$. Doneu les
        coordenades del vector $w$ respecte a la base de $G$ triada.
    \end{exercici}
    \begin{solucio}
        Primer, calculem una base de $G$:
        \begin{displaymath}
            \underbrace{\left(
                \begin{array}{cccc}
                    1 & 2 & -3 & 4
                \end{array}
            \right)}_{\text{"Matriu" }A}
            \left(
                \begin{array}{c}
                    x\\y\\z\\t
                \end{array}
            \right) =
            \left(
                \begin{array}{c}
                    0\\0\\0\\0
                \end{array}
            \right)
        \end{displaymath}
        Això és equivalent a trobar el nucli de $A$.
        \begin{displaymath}
            \left(
                \begin{array}{cccc}
                    1 & 2 & -3 & 4\\
                    \hline
                    1 & 0 & 0 & 0\\
                    0 & 1 & 0 & 0\\
                    0 & 0 & 1 & 0\\
                    0 & 0 & 0 & 1\\
                \end{array}
            \right)
            \sim
            \left(
                \begin{array}{cccc}
                    1 & 0 & 0 & 0\\
                    \hline
                    1 & -2 & 3 & -4\\
                    0 & 1 & 0 & 0\\
                    0 & 0 & 1 & 0\\
                    0 & 0 & 0 & 1\\
                    & \multicolumn{3}{c}{\raisebox{4ex}{$ \underbrace{\hspace{5em}}_{\text{Nucli de } A} $}}
                \end{array}
            \right)
        \end{displaymath}
        Llavors, una base de $G$ és $((-2, 1, 0, 0), (3, 0, 1, 0), (-4, 0, 0, 1))$. Ara, comprovem que $w$ és de $G$:
        \begin{displaymath}
            \left(
                \begin{array}{cccc|cccc}
                    -2 & 1 & 0 & 0 & 1 & 0 & 0 & 0\\
                    3 & 0 & 1 & 0 & 0 & 1 & 0 & 0\\
                    -4 & 0 & 0 & 1 & 0 & 0 & 1 & 0\\
                    3 & 3 & 3 & 0 & 0 & 0 & 0 & 1
                \end{array}
            \right)
            \sim
            \left(
                \begin{array}{cccc|cccc}
                    1 & -\frac{1}{2} & 0 & 0 & -\frac{1}{2} & 0 & 0 & 0\\
                    0 & \frac{3}{2} & 1 & 0 & \frac{3}{2} & 1 & 0 & 0\\
                    0 & -2 & 0 & 1 & -2 & 0 & 1 & 0\\
                    0 & \frac{9}{2} & 3 & 0 & \frac{3}{2} & 0 & 0 & 1
                \end{array}
            \right)
        \end{displaymath}
        \begin{displaymath}
            \sim
            \left(
                \begin{array}{cccc|cccc}
                    1 & -\frac{1}{2} & 0 & 0 & -\frac{1}{2} & 0 & 0 & 0\\
                    0 & 1 & \frac{2}{3} & 0 & 1 & \frac{2}{3} & 0 & 0\\
                    0 & 0 & \frac{4}{3} & 1 & 0 & \frac{4}{3} & 1 & 0\\
                    0 & 0 & 0 & 0 & -3 & -3 & 0 & 1
                \end{array}
            \right)
        \end{displaymath}
        Llavors, com $w$ és combinació lineal de les bases de $G$, $w \in G$. Per la base de $G$
        triada abans $w = 3(-2, 1, 0, 0) + 3(3, 0, 1, 0)$. Les coordenades de $w$ a la base triada de $G$
        és $(3, 3, 0)$.
    \end{solucio}
    \begin{exercici}
        Trobeu una base de $F+G$ i la dimensió de $F+G$. És $F \subseteq G$? Per què? És $F+G = \mathbb{R}^4$?
        Per què?
    \end{exercici}
    \begin{solucio}
        Posem les dues bases en una mateixa matriu, sent les 3 primeres de $G$ i la resta de $F$.
        \begin{displaymath}
            \left(
                \begin{array}{cccc|cccccc}
                    -2 & 1 & 0 & 0 & 1 & 0 & 0 & 0 & 0 & 0\\
                    3 & 0 & 1 & 0 & 0 & 1 & 0 & 0 & 0 & 0\\
                    -4 & 0 & 0 & 1 & 0 & 0 & 1 & 0 & 0 & 0\\
                    \hline
                    1 & 2 & 3 & 0 & 0 & 0 & 0 & 1 & 0 & 0\\
                    0 & 1 & 2 & 0 & 0 & 0 & 0 & 0 & 1 & 0\\
                    0 & 0 & 1 & 0 & 0 & 0 & 0 & 0 & 0 & 1
                \end{array}
            \right)
            \sim
            \left(
                \begin{array}{cccc|cccccc}
                    1 & -\frac{1}{2} & 0 & 0 & -\frac{1}{2} & 0 & 0 & 0 & 0 & 0\\
                    0 & \frac{3}{2} & 1 & 0 & \frac{3}{2} & 1 & 0 & 0 & 0 & 0\\
                    0 & -2 & 0 & 1 & -2 & 0 & 1 & 0 & 0 & 0\\
                    \hline
                    0 & \frac{5}{2} & 3 & 0 & \frac{1}{2} & 0 & 0 & 1 & 0 & 0\\
                    0 & 1 & 2 & 0 & 0 & 0 & 0 & 0 & 1 & 0\\
                    0 & 0 & 1 & 0 & 0 & 0 & 0 & 0 & 0 & 1
                \end{array}
            \right)
        \end{displaymath}
        \begin{displaymath}
            \sim
            \left(
                \begin{array}{cccc|cccccc}
                    1 & -\frac{1}{2} & 0 & 0 & -\frac{1}{2} & 0 & 0 & 0 & 0 & 0\\
                    0 & 1 & \frac{2}{3} & 0 & 1 & \frac{2}{3} & 0 & 0 & 0 & 0\\
                    0 & 0 & \frac{4}{3} & 1 & 0 & \frac{4}{3} & 1 & 0 & 0 & 0\\
                    \hline
                    0 & 0 & -\frac{4}{3} & 0 & -2 & -\frac{5}{3} & 0 & 1 & 0 & 0\\
                    0 & 0 & \frac{4}{3} & 0 & -1 & 0 & 0 & 0 & 1 & 0\\
                    0 & 0 & 1 & 0 & 0 & 0 & 0 & 0 & 0 & 1
                \end{array}
            \right)
            \sim
            \left(
                \begin{array}{cccc|cccccc}
                    1 & -\frac{1}{2} & 0 & 0 & -\frac{1}{2} & 0 & 0 & 0 & 0 & 0\\
                    0 & 1 & \frac{2}{3} & 0 & 1 & \frac{2}{3} & 0 & 0 & 0 & 0\\
                    0 & 0 & \frac{4}{3} & 1 & 0 & \frac{4}{3} & 1 & 0 & 0 & 0\\
                    \hline
                    0 & 0 & -\frac{4}{3} & 0 & -2 & -\frac{5}{3} & 0 & 1 & 0 & 0\\
                    0 & 0 & 0 & 0 & -3 & -\frac{5}{3} & 0 & 1 & 1 & 0\\
                    0 & 0 & 0 & 0 & -\frac{3}{2} & -\frac{5}{4} & 0 & \frac{3}{4} & 0 & 1
                \end{array}
            \right)
        \end{displaymath}
        Arribat aquest punt, ens podem adonar que l'única manera de fer la forma esglaonada reduïda de
        la matriu de les bases implica fer un intercanvi de files, de la 3 i la 4, això ens demostra que
        $F\not\subseteq G$. Una base de $F+G$ és $((1, -\frac{1}{2}, 0, 0), (0, 1, \frac{2}{3}, 0), (0, 0, \frac{4}{3}, 1), (0, 0, -\frac{4}{3}))$,
        i per allò, $\dim(F+G) = 4$. Com la dimensió és $4$, si o si ha de generar un subespai de
        dimensió $4$ de $\mathbb{R}^4$. L'únic subespai possible és el mateix $\mathbb{R}^4$.
    \end{solucio}
    \begin{exercici}
        Trobeu una base de $F \cap G$ i la dimensió de $F \cap G$.
    \end{exercici}
    \begin{solucio}
        Aprofitant els càlculs que he fet a l'anterior exercici amb la matriu ampliada, una base de
        $F\cap G$ és $((-3, -\frac{5}{3}, 0, 1), (-\frac{3}{2}, -\frac{5}{4}, 0, \frac{3}{4}))$.
        Això implica que $\dim(F\cap G) = 2$. Va en concordança amb el resultat trobat amb $\dim(F\cap G) = \dim(F+G) - (\dim(F) + \dim(G))$
    \end{solucio}
    \subsection{Considera un subespai vectorial $H := <1, \sin(x), \cos(x), e^{2x}>$ de les funcions contínues de domini $\mathbb{R}$ de funcions reals en una variable.}
    \begin{exercici}
        Proveu que $(1, \sin(x), \cos(x), e^{2x})$ és una base de $H$.
    \end{exercici}
    \begin{solucio}
        El subespai $H$ està generat per $(1, \sin(x), \cos(x), e^{2x})$. Com no són linealment
        dependents, per definició, $(1, \sin(x), \cos(x), e^{2x})$ és una base de $H$.
    \end{solucio}
    \begin{exercici}
        Considerem els subespais de les funcions reals en una variable
        \begin{displaymath}
            F' := <B+B\sin(x)+(B-1)\cos(x), 3+2\sin(x)+\cos(x), 1+2\sin(x)+3\cos(x), 4+3\sin(x)+2\cos(x)>
        \end{displaymath}
        \begin{displaymath}
            G' := \{a1 + b\sin(x) + c\cos(x)+de^{2x}\text{ }|\text{ }a,b,c,d \in \mathbb{R}, a+b-c+d = 0\}
        \end{displaymath}
        \begin{itemize}
            \item Proveu que $F'$, $G'$ són subespais vectorials de $H$.
            \item Trobeu una base i dimensió de $F'$, $G'$, $F'+G'$ i $F' \cap G'$
        \end{itemize}
    \end{exercici}
    \begin{solucio}
        Primer de tot, provem que $F'$ i $G'$ són subespais vectorials de $H$:
        \begin{itemize}
            \item Posem la base de $H$ i els vectors generadors de $F'$ en una matriu:
            \begin{displaymath}
                \left(
                    \begin{array}{c|cccccccc}
                        1 & 1 & 0 & 0 & 0 & 0 & 0 & 0 & 0\\
                        \sin(x) & 0 & 1 & 0 & 0 & 0 & 0 & 0 & 0\\
                        \cos(x) & 0 & 0 & 1 & 0 & 0 & 0 & 0 & 0\\
                        e^{2x} & 0 & 0 & 0 & 1 & 0 & 0 & 0 & 0\\
                        \hline
                        5+5\sin(x)+4\cos(x) & 0 & 0 & 0 & 0 & 1 & 0 & 0 & 0\\
                        3+2\sin(x)+\cos(x) & 0 & 0 & 0 & 0 & 0 & 1 & 0 & 0\\
                        1+2\sin(x)+3\cos(x) & 0 & 0 & 0 & 0 & 0 & 0 & 1 & 0\\
                        4+3\sin(x)+2\cos(x) & 0 & 0 & 0 & 0 & 0 & 0 & 0 & 1
                    \end{array}
                \right)
            \end{displaymath}
            \begin{displaymath}
                \sim
                \left(
                    \begin{array}{c|cccccccc}
                        1 & 1 & 0 & 0 & 0 & 0 & 0 & 0 & 0\\
                        \sin(x) & 0 & 1 & 0 & 0 & 0 & 0 & 0 & 0\\
                        \cos(x) & 0 & 0 & 1 & 0 & 0 & 0 & 0 & 0\\
                        e^{2x} & 0 & 0 & 0 & 1 & 0 & 0 & 0 & 0\\
                        \hline
                        0 & -5 & -5 & -4 & 0 & 1 & 0 & 0 & 0\\
                        0 & -3 & -2 & -1 & 0 & 0 & 1 & 0 & 0\\
                        0 & -1 & -2 & -3 & 0 & 0 & 0 & 1 & 0\\
                        0 & -4 & -3 & -2 & 0 & 0 & 0 & 0 & 1
                    \end{array}
                \right)
            \end{displaymath}
            Com podem veure, els vectors generadors de $F'$ desapareixen en reduir la matriu,
            per això $F'$ és un subespai de $H$.
            \item Per $G'$, si tenim en compte que $a1 + b\sin(x) + c\cos(x) + de^{2x} = (a, b, c, d)(1,\sin,\cos,e^{2x})^t$.
            Sabem que, independentment del valor de $(a, b, c, d)$, els vectors de $G'$ seran
            combinació lineal de la base de $H$. Per allò, $G'$ és un subespai de $H$.
        \end{itemize}
        Fixem la base $(1, \sin(x), \cos(x), e^{2x})$. $F'$ està llavors generat per $<(5, 5, 4, 0), (3, 2, 1, 0), (1, 2, 3, 0), (4, 3, 2, 0)>$
        en coordenades de la base abans esmentada (això està calculat indirectament a l'ampliada en
        fer la reducció. Com a petita observació, aquests vectors són els mateixos que generen $F$
        a l'apartat a, per això no tornaré a fer certs càlculs). $G'$ en les coordenades ja dites serà el càlcul del nucli següent:
        \begin{displaymath}
            \left(
                \begin{array}{cccc}
                    1 & 1 & -1 & 1\\
                    \hline
                    1 & 0 & 0 & 0\\
                    0 & 1 & 0 & 0\\
                    0 & 0 & 1 & 0\\
                    0 & 0 & 0 & 1\\
                \end{array}
            \right)
            \sim
            \left(
                \begin{array}{cccc}
                    1 & 0 & 0 & 0\\
                    \hline
                    1 & -1 & 1 & -1\\
                    0 & 1 & 0 & 0\\
                    0 & 0 & 1 & 0\\
                    0 & 0 & 0 & 1\\
                \end{array}
            \right)
        \end{displaymath}
        Sent el nucli $((-1, 1, 0, 0), (1, 0, 1, 0), (-1, 0, 0, 1))$, aquest nucli és la base de $G'$
        en coordenades abans dites.
        Llavors, $\dim(G') = 3$. Per a $F'$, la base serà (calculada a l'apartat 'a' exercici 1) $((1, 2, 3, 0), (0, 1, 2, 0), (0, 0, -1, 0))$
        i $\dim(F') = 3$. Per trobar $F'+G'$ i $F' \cap G'$ es posa en una matriu amb la base de $F'$
        primer i després la base de $G'$:
        \begin{displaymath}
            \left(
                \begin{array}{cccc|cccccc}
                    1 & 2 & 3 & 0 & 1 & 0 & 0 & 0 & 0 & 0\\
                    0 & 1 & 2 & 0 & 0 & 1 & 0 & 0 & 0 & 0\\
                    0 & 0 & 1 & 0 & 0 & 0 & 1 & 0 & 0 & 0\\
                    \hline
                    -1 & 1 & 0 & 0 & 0 & 0 & 0 & 1 & 0 & 0\\
                    1 & 0 & 1 & 0 & 0 & 0 & 0 & 0 & 1 & 0\\
                    -1 & 0 & 0 & 1 & 0 & 0 & 0 & 0 & 0 & 1\\
                \end{array}
            \right)
            \sim
            \left(
                \begin{array}{cccc|cccccc}
                    1 & 2 & 3 & 0 & 1 & 0 & 0 & 0 & 0 & 0\\
                    0 & 1 & 2 & 0 & 0 & 1 & 0 & 0 & 0 & 0\\
                    0 & 0 & 1 & 0 & 0 & 0 & 1 & 0 & 0 & 0\\
                    \hline
                    0 & 3 & 3 & 0 & 1 & 0 & 0 & 1 & 0 & 0\\
                    0 & -2 & -2 & 0 & -1 & 0 & 0 & 0 & 1 & 0\\
                    0 & 2 & 3 & 1 & 1 & 0 & 0 & 0 & 0 & 1\\
                \end{array}
            \right)
        \end{displaymath}
        \begin{displaymath}
            \sim
            \left(
                \begin{array}{cccc|cccccc}
                    1 & 2 & 3 & 0 & 1 & 0 & 0 & 0 & 0 & 0\\
                    0 & 1 & 2 & 0 & 0 & 1 & 0 & 0 & 0 & 0\\
                    0 & 0 & 1 & 0 & 0 & 0 & 1 & 0 & 0 & 0\\
                    \hline
                    0 & 0 & -3 & 0 & 1 & -3 & 0 & 1 & 0 & 0\\
                    0 & 0 & 2 & 0 & -1 & 2 & 0 & 0 & 1 & 0\\
                    0 & 0 & -1 & 1 & 1 & -2 & 0 & 0 & 0 & 1\\
                \end{array}
            \right)
            \sim
            \left(
                \begin{array}{cccc|cccccc}
                    1 & 2 & 3 & 0 & 1 & 0 & 0 & 0 & 0 & 0\\
                    0 & 1 & 2 & 0 & 0 & 1 & 0 & 0 & 0 & 0\\
                    0 & 0 & 1 & 0 & 0 & 0 & 1 & 0 & 0 & 0\\
                    \hline
                    0 & 0 & 0 & 0 & 1 & -3 & 3 & 1 & 0 & 0\\
                    0 & 0 & 0 & 0 & -1 & 2 & -2 & 0 & 1 & 0\\
                    0 & 0 & -1 & 1 & 1 & -2 & 0 & 0 & 0 & 1\\
                \end{array}
            \right)
        \end{displaymath}
        com podem observar, una base de $F'+G'$ és $((1,2,3,0), (0,1,2,0), (0, 0, 0, 1), (0, 0, -1, 1))$,
        el que fa que $\dim(F'+G') = 4$. En canvi, una base de $F' \cap G'$ és $((1, -3, 3, 1), (-1, 2, -2, 0))$,
        fent que $\dim(F' \cap G') = 2$.\\
        Recapitulant:
        \begin{itemize}
            \item Base de $F'$ és $(1+2\sin(x)+3\cos(x), 1\sin(x)+2\cos(x), \cos(x))$, $\dim(F') = 3$.
            \item Base de $G'$ és $(-2+\sin(x), 3+\cos(x), -4+e^{2x})$, $\dim(G') = 3$.
            \item Base de $F'+G'$ és $(1+2\sin(x)+3\cos(x), 1\sin(x)+2\cos(x), -\cos(x), e^{2x})$, $\dim(F'+G') = 4$.
            \item Base de $F'\cap G'$ és $(1-3\sin(x)+3\cos(x)+e^{2x}, -1+2\sin(x)-2\cos(x))$, $\dim(F'\cap G') = 2$.
        \end{itemize}
    \end{solucio}

    \section{Matriu d'aplicacions lineals en espais vectorials reals finits generats.}
    \subsection{Considera l'aplicació lineal $f: \mathbb{R}^3 \rightarrow \mathbb{R}^3$ definida per}
    \begin{displaymath}
        f(x,y,z) = (Bx+y+z, x-y+z, x+y+z)
    \end{displaymath}
    \begin{exercici}
        Calcula la matriu $A$ associada a $f = T_A$ en les bases canòniques. I calculeu $f(1,2,3)$ i l'antiimatge de $(1,0,0)$, és a dir $f^{-1}(1,0,0)$.
    \end{exercici}
    \begin{solucio}
        Per trobar $A$ associada en les bases canòniques, calculem $f(1, 0, 0), f(0, 1, 0)\text{ i }f(0, 0, 1)$.
        \begin{displaymath}
            f(1, 0, 0) = (5, 1, 1)\qquad f(0, 1, 0) = (1, -1, 1)\qquad f(0, 0, 1) = (1, 1, 1)
        \end{displaymath}
        Ara fet això, ho posem en coordenades de la canònica:
        \begin{displaymath}
            A = \left(
                \begin{array}{ccc}
                    5 & 1 & 1\\
                    1 & -1 & 1\\
                    1 & 1 & 1
                \end{array}
            \right)
        \end{displaymath}
        Ara, calculem $f(1, 2, 3)$ mitjançant la matriu:
        \begin{displaymath}
            f(1, 2, 3) = \left(
                \begin{array}{ccc}
                    5 & 1 & 1\\
                    1 & -1 & 1\\
                    1 & 1 & 1
                \end{array}
            \right)
            \left(
                \begin{array}{ccc}
                    1\\
                    2\\
                    3
                \end{array}
            \right)
            =
            \left(
                \begin{array}{ccc}
                    10\\
                    2\\
                    6
                \end{array}
            \right)
        \end{displaymath}
        I ara, calculem $f^{-1}(1, 0, 0)$. Això és trivial si tenim en compte el següent:
        \begin{displaymath}
            Ax = (1, 0, 0) \Rightarrow x = A^{-1}(1, 0, 0)
        \end{displaymath}
        Llavors, calculem $A^{-1}$:
        \begin{displaymath}
            \left(
                \begin{array}{ccc|ccc}
                    5 & 1 & 1 & 1 & 0 & 0\\
                    1 & -1 & 1 & 0 & 1 & 0\\
                    1 & 1 & 1 & 0 & 0 & 1
                \end{array}
            \right)
            \sim
            \left(
                \begin{array}{ccc|ccc}
                    1 & 0 & 0 & \frac{1}{4} & 0 & -\frac{1}{4}\\
                    0 & 1 & 0 & 0 & -\frac{1}{2} & \frac{1}{2}\\
                    0 & 0 & 1 & -\frac{1}{4} & \frac{1}{2} & \frac{3}{4}
                \end{array}
            \right)
        \end{displaymath}
        Com $A^{-1}$ existeix, podem fer el que hem dit abans:
        \begin{displaymath}
            x = A^{-1}(1, 0, 0) =
            \left(
                \begin{array}{ccc}
                    \frac{1}{4} & 0 & -\frac{1}{4}\\
                    0 & -\frac{1}{2} & \frac{1}{2}\\
                    -\frac{1}{4} & \frac{1}{2} & \frac{3}{4}
                \end{array}
            \right)
            \left(
                \begin{array}{c}
                   1\\
                   0\\
                   0
                \end{array}
            \right)
            =
            \left(
                \begin{array}{c}
                   \frac{1}{4}\\
                   0\\
                   -\frac{1}{4}
                \end{array}
            \right)
        \end{displaymath}
    \end{solucio}
    \begin{exercici}
        Calculeu una base i dimensió del nucli de $f$ i la imatge de $f$.
    \end{exercici}
    \begin{solucio}
        Com, sense voler, hem provat a l'anterior exercici, la matriu $A$ és invertible. Al ser
        invertible, el rang de $A$ és màxim, i per allò, $\dim(\ker(f)) = 0$ i $\dim(\text{im}(f)) = \text{rang}(f) = 3$.
        Trivialment $\ker(f) = 0$ i $\text{im}(f)$ seran les files de la matriu $A$, és a dir $\text{im}(f) = ((5, 1, 1), (1, -1, 1), (1, 1, 1))$.
    \end{solucio}
    \begin{exercici}
        Decidiu si $f$ és injectiva, exhaustiva o bijectiva.
    \end{exercici}
    \begin{solucio}
        Com la matriu $A$ és invertible, $f$ és bijectiva, és a dir, tant injectiva com exhaustiva.
    \end{solucio}
    \begin{exercici}
        Calculeu l'aplicació lineal $(f \circ f)$.
    \end{exercici}
    \begin{solucio}
        Calculem $f \circ f$:
        \begin{displaymath}
            (f \circ f)(x, y, z) = A^2(x, y, z)
        \end{displaymath}
        Llavors, és tan simple com calcular $A^2$.
        \begin{displaymath}
            A^2 =
            \left(
                \begin{array}{ccc}
                    5 & 1 & 1\\
                    1 & -1 & 1\\
                    1 & 1 & 1
                \end{array}
            \right)
            \left(
                \begin{array}{ccc}
                    5 & 1 & 1\\
                    1 & -1 & 1\\
                    1 & 1 & 1
                \end{array}
            \right)
            =
            \left(
                \begin{array}{ccc}
                    27 & 5 & 7\\
                    5 & 3 & 1\\
                    7 & 1 & 3
                \end{array}
            \right)
        \end{displaymath}
    \end{solucio}
    \begin{exercici}
        Considera la base de $\mathbb{R}^3: \mathfrak{B} = ((1,1,1), (0,1,1), (0,0,1))$. Calcula la
        matriu associada a $f$ amb bases d'inici i sortida $\mathfrak{B}$, és a dir calculeu $M(\mathfrak{B} \xleftarrow{f} \mathfrak{B})$.
    \end{exercici}
    \begin{solucio}
        Per calcular $M(\mathfrak{B} \xleftarrow{f} \mathfrak{B})$ faré servir la següent idea:
        \begin{displaymath}
            M(\mathfrak{B} \xleftarrow{f} \mathfrak{B}) = M(\mathfrak{B} \xleftarrow{id} Can_{\mathbb{R}^3}) M(Can_{\mathbb{R}^3} \xleftarrow{f} Can_{\mathbb{R}^3})  M(Can_{\mathbb{R}^3} \xleftarrow{id} \mathfrak{B})
        \end{displaymath}
        Tenint en compte que $M(Can_{\mathbb{R}^3} \xleftarrow{f} Can_{\mathbb{R}^3}) =
        \left(
            \begin{smallmatrix}
                5 & 1 & 1\\
                1 & -1 & 1\\
                1 & 1 & 1
            \end{smallmatrix}
        \right)$ i que $M(\mathfrak{B} \xleftarrow{id} Can_{\mathbb{R}^3})$ és la matriu de canvi de
        base, fent que $M(\mathfrak{B} \xleftarrow{id} Can_{\mathbb{R}^3}) = M(Can_{\mathbb{R}^3} \xleftarrow{id} \mathfrak{B})^{-1}$,
        trobem $M(\mathfrak{B} \xleftarrow{id} Can_{\mathbb{R}^3})$:
        \begin{displaymath}
            M(\mathfrak{B} \xleftarrow{id} Can_{\mathbb{R}^3}) =
            \left(
                \begin{array}{c|c|c}
                    1 & 0 & 0\\
                    1 & 1 & 0\\
                    1 & 1 & 1
                \end{array}
            \right)
        \end{displaymath}
        Ara, per trobar $M(Can_{\mathbb{R}^3} \xleftarrow{id} \mathfrak{B})$ fem servir la inversa
        de l'anterior matriu:
        \begin{displaymath}
            \left(
                \begin{array}{ccc|ccc}
                    1 & 0 & 0 & 1 & 0 & 0\\
                    1 & 1 & 0 & 0 & 1 & 0\\
                    1 & 1 & 1 & 0 & 0 & 1
                \end{array}
            \right)
            \sim
            \left(
                \begin{array}{ccc|ccc}
                    1 & 0 & 0 & 1 & 0 & 0\\
                    0 & 1 & 0 & -1 & 1 & 0\\
                    0 & 1 & 1 & -1 & 0 & 1
                \end{array}
            \right)
            \sim
            \left(
                \begin{array}{ccc|ccc}
                    1 & 0 & 0 & 1 & 0 & 0\\
                    0 & 1 & 0 & -1 & 1 & 0\\
                    0 & 0 & 1 & 0 & -1 & 1
                \end{array}
            \right)
        \end{displaymath}
        Llavors, ara que tenim les tres matrius, multipliquem:
        \begin{align*}
            M(\mathfrak{B} \xleftarrow{f} \mathfrak{B}) &=
            \left(
                \begin{array}{ccc}
                  1 & 0 & 0\\
                  -1 & 1 & 0\\
                  0 & -1 & 1
                \end{array}
            \right)
            \left(
                \begin{array}{ccc}
                    5 & 1 & 1\\
                    1 & -1 & 1\\
                    1 & 1 & 1
                \end{array}
            \right)
            \left(
                \begin{array}{ccc}
                    1 & 0 & 0\\
                    1 & 1 & 0\\
                    1 & 1 & 1
                \end{array}
            \right)\\
            &=
            \left(
                \begin{array}{ccc}
                    7 & 2 & 1\\
                    -6 & -2 & 0\\
                    2 & 2 & 0
                \end{array}
            \right)
        \end{align*}
    \end{solucio}
    \begin{exercici}
        Trobeu una matriu invertible $P$, on $A = PM(\mathfrak{B} \xleftarrow{f} \mathfrak{B})P^{-1}$. La
        matriu $P$ és una matriu de canvi de base, de coordenades de quina base a quina alta base és?
    \end{exercici}
    \begin{solucio}
        Com ja he calculat a l'anterior exercici, $P = M(\mathfrak{B} \xleftarrow{id} Can_{\mathbb{R}^3})$,
        és a dir:
        \begin{displaymath}
            P =
            \left(
                \begin{array}{ccc}
                    1 & 0 & 0\\
                    1 & 1 & 0\\
                    1 & 1 & 1
                \end{array}
            \right)
        \end{displaymath}
    \end{solucio}

    \subsection{Sigui $\mathbb{R}[x]_2$ l'espai vectorial dels polinomis de grau $\geq 2$.
    Considera l'aplicació $g: \mathbb{R}[x]_2 \rightarrow \mathbb{R}[x]_2$ definida per}
    \begin{displaymath}
        g(a+bx+cx^2) = (Ba+b+c)1 + (a-b+c)x + (a+b+c)x^2
    \end{displaymath}
    \begin{exercici}
        Justifiqueu perquè $g$ és una aplicació lineal.
    \end{exercici}
    \begin{solucio}
        $G$ és una aplicació lineal perquè és una aplicació que transforma un espai vectorial a un
        altre, mantenint la suma de vectors i la multiplicació per un escalar.
    \end{solucio}
    \begin{exercici}
        Fixeu la base $Can := (1, x, x^2)$ de $\mathbb{R}[x]_2$. Calculeu la matriu $A = M(Can \xleftarrow{g} Can)$.
        Calculeu $g(3+2x+x^2)$ i $g^{-1}(x)$, directament i usant la matriu $A$.
    \end{exercici}
    \begin{solucio}
        Tenint la base ja esmentada, per calcular la matriu $A$ associada calculem $g(1), g(x)\text{ i }g(x^2)$.
        \begin{displaymath}
            g(1) = (5+x+x^2)\qquad g(x) = (1-x+x^2)\qquad g(x^2) = (1+x+x^2)
        \end{displaymath}
        Llavors, ara calculem $M(Can \xleftarrow{g} Can_{\mathbb{R}^3})$:
        \begin{displaymath}
            M(Can \xleftarrow{g} Can_{\mathbb{R}^3}) = \left(
                \begin{array}{ccc}
                    5+x+x^2&1-x+x^2&1+x+x^2
                \end{array}
            \right)
        \end{displaymath}
        No obstant això, ens demanen $M(Can \xleftarrow{g} Can)$, i per allò hem de trobar un canvi
        de base $M(Can_{\mathbb{R}^3} \xleftarrow{id} Can)$. Això és tan simple com fer $M(Can \xleftarrow{id} Can_{\mathbb{R}^3})^{-1}$,
        una matriu trivial.
        \begin{displaymath}
            M(Can \xleftarrow{id} Can_{\mathbb{R}^3}) = \left(
                \begin{array}{c|c|c}
                    1 & 0 & 0\\
                    0 & x & 0\\
                    0 & 0 & x^2
                \end{array}
            \right)
        \end{displaymath}
        \begin{displaymath}
            \left(
                \begin{array}{ccc|ccc}
                    1 & 0 & 0 & 1 & 0 & 0\\
                    0 & x & 0 & 0 & 1 & 0\\
                    0 & 0 & x^2 & 0 & 0 & 1
                \end{array}
            \right)
            \sim
            \left(
                \begin{array}{ccc|ccc}
                    1 & 0 & 0 & 1 & 0 & 0\\
                    0 & 1 & 0 & 0 & \frac{1}{x} & 0\\
                    0 & 0 & 1 & 0 & 0 & \frac{1}{x^2}
                \end{array}
            \right)
        \end{displaymath}
        \begin{displaymath}
            M(Can_{\mathbb{R}^3} \xleftarrow{id} Can) = \left(
                \begin{array}{ccc}
                    1 & 0 & 0\\
                    0 & \frac{1}{x} & 0\\
                    0 & 0 & \frac{1}{x^2}
                \end{array}
            \right)
        \end{displaymath}
        Una vegada tenim això, calculem $A$:
        \begin{align*}
            A &= M(Can \xleftarrow{g} Can)\\
            &= M(Can \xleftarrow{g} Can_{\mathbb{R}^3}) M(Can_{\mathbb{R}^3} \xleftarrow{id} Can)\\
            &=
            \left(
                \begin{array}{ccc}
                    5+x+x^2&1-x+x^2&1+x+x^2
                \end{array}
            \right)
            \left(
                \begin{array}{ccc}
                    1 & 0 & 0\\
                    0 & \frac{1}{x} & 0\\
                    0 & 0 & \frac{1}{x^2}
                \end{array}
            \right)\\
            &=
            \left(
                \begin{array}{ccc}
                    5+x+x^2 & \frac{1-x+x^2}{x} & \frac{1+x+x^2}{x^2}
                \end{array}
            \right)
        \end{align*}
        Ara, calculem $g(3+2x+x^2)$:
        \begin{itemize}
            \item Directament:
            \begin{displaymath}
                g(3+2x+x^2) = (5\times 3+2+1)1+(3-2+1)x+(3+2+1)x^2
            \end{displaymath}
            \item Amb la matriu $A$:
            \begin{displaymath}
                \left(
                    \begin{array}{ccc}
                        5+x+x^2 & \frac{1-x+x^2}{x} & \frac{1+x+x^2}{x^2}
                    \end{array}
                \right)
                \left(
                    \begin{array}{c}
                        3\\
                        2x\\
                        x^2
                    \end{array}
                \right)
                =
                \left(
                    \begin{array}{ccc}
                        5+x+x^2 & \frac{1-x+x^2}{x} & \frac{1+x+x^2}{x^2}
                    \end{array}
                \right)
            \end{displaymath}
        \end{itemize}
        Ara, calculem $g^{-1}(x)$:
        \begin{itemize}
            \item Directament:
                \begin{displaymath}
                    g^{-1}(a1+bx+cx^2) = (\frac{1}{4}a-\frac{1}{4}c)1+(-\frac{1}{2}b + \frac{1}{2}c)x+(-\frac{1}{4}a+\frac{1}{2}b+\frac{3}{4}c)x^2
                \end{displaymath}
            \item Amb la matriu $A$:
                \begin{displaymath}
                    A^{-1} = \left(
                        \begin{array}{ccc}
                            \frac{1}{4}-\frac{1}{4}x^2 & -\frac{1}{2}+\frac{1}{2}x & \frac{-\frac{1}{4} + \frac{1}{2}x + \frac{3}{4}x^2}{x^2}
                        \end{array}
                    \right)
                \end{displaymath}
        \end{itemize}
    \end{solucio}
    \begin{exercici}
        Decidiu si $g$ és injectiva, exhaustiva i bijectiva.
    \end{exercici}
    \begin{solucio}
        Com $g^{-1}$ existeix, $g$ ha de ser bijectiva.
    \end{solucio}
    \begin{exercici}
        Proveu que $\mathfrak{B}b = (1 + x + x^2, x + x^2, x)$ és una base de $\mathbb{R}[x]_2$. I
        calculeu la matriu $M(\mathfrak{B}b \xleftarrow{g} \mathfrak{B}b)$.
    \end{exercici}
    \begin{solucio}
        Posem a una matriu els vectors de la base $Can$ i la base $\mathfrak{B}b$:
        \begin{displaymath}
            \left(
                \begin{array}{c|cccccc}
                    1 & 1 & 0 & 0 & 0 & 0 & 0\\
                    x & 0 & 1 & 0 & 0 & 0 & 0\\
                    x^2 & 0 & 0 & 1 & 0 & 0 & 0\\
                    \hline
                    1 + x + x^2 & 0 & 0 & 0 & 1 & 0 & 0\\
                    x + x^2 & 0 & 0 & 0 & 0 & 1 & 0\\
                    x & 0 & 0 & 0 & 0 & 0 & 1
                \end{array}
            \right)
            \sim
            \left(
                \begin{array}{c|cccccc}
                    1 & 1 & 0 & 0 & 0 & 0 & 0\\
                    x & 0 & 1 & 0 & 0 & 0 & 0\\
                    x^2 & 0 & 0 & 1 & 0 & 0 & 0\\
                    \hline
                    0 & -1 & -1 & -1 & 1 & 0 & 0\\
                    0 & 0 & -1 & -1 & 0 & 1 & 0\\
                    0 & 0 & -1 & 0 & 0 & 0 & 1
                \end{array}
            \right)
        \end{displaymath}
        Com en reduir la matriu els vectors de $\mathfrak{B}b$ desapareixen, $\mathfrak{B}b$ és una
        base d'un subespai de l'espai generat per $Can$. Com la dimensió de l'espai generat per $\mathfrak{B}b$
        és igual a la dimensió de l'espai generat per $Can$, $\mathfrak{B}b$ és una base de $\mathbb{R}[x]_2$.\\
        Ara, calculem $M(\mathfrak{B}b \xleftarrow{g} \mathfrak{B}b)$:\\
        Fem $g(1+x+x^2), g(x+x^2)\text{ i }g(x)$
        \begin{displaymath}
            g(1+x+x^2) = (7+x+3x^2)\qquad g(x+x^2) = (2+2x^2)\qquad g(x) = (1-x+x^2)
        \end{displaymath}
        Posant-ho en una matriu tenim el següent:
        \begin{displaymath}
            M(\mathfrak{B}b \xleftarrow{g} Can_{\mathbb{R}^3}) =
            \left(
                \begin{array}{ccc}
                    7+x+3x^2 & 2+2x^2 & 1-x+x^2
                \end{array}
            \right)
        \end{displaymath}
        Ara, fem un pas similar al de l'exercici 2:
        \begin{displaymath}
            M(\mathfrak{B}b \xleftarrow{id} Can_{\mathbb{R}^3}) = \left(
                \begin{array}{c|c|c}
                    1 & 0 & 0\\
                    0 & x & 0\\
                    0 & 0 & x^2
                \end{array}
            \right)
        \end{displaymath}
        \begin{displaymath}
            \left(
                \begin{array}{ccc|ccc}
                    1 & 0 & 0 & 1 & 0 & 0\\
                    x & x & x & 0 & 1 & 0\\
                    x^2 x^2 0 & 0 & 0 & 1
                \end{array}
            \right)
            \sim
            \left(
                \begin{array}{ccc|ccc}
                    1 & 0 & 0 & 1 & 0 & 0\\
                    0 & 1 & 0 & -1 & 0 & -\frac{1}{x^2}\\
                    0 & 0 & 1 & 0 & \frac{1}{x} & \frac{1}{x^2}
                \end{array}
            \right)
        \end{displaymath}
        \begin{displaymath}
            M(Can_{\mathbb{R}^3} \xleftarrow{id} \mathfrak{B}b) = 
            \left(
                \begin{array}{ccc}
                    1 & 0 & 0\\
                    -1 & 0 & -\frac{1}{x^2}\\
                    0 & \frac{1}{x} & \frac{1}{x^2}
                \end{array}
            \right)
        \end{displaymath}
        \begin{align*}
            M(\mathfrak{B}b \xleftarrow{g} \mathfrak{B}b) &= M(\mathfrak{B}b \xleftarrow{g} Can_{\mathbb{R}^3}) M(Can_{\mathbb{R}^3} \xleftarrow{id} \mathfrak{B}b)\\
            &= 
            \left(
                \begin{array}{ccc}
                    7+x+3x^2 & 2+2x^2 & 1-x+x^2
                \end{array}
            \right)
            \left(
                \begin{array}{ccc}
                    1 & 0 & 0\\
                    -1 & 0 & -\frac{1}{x^2}\\
                    0 & \frac{1}{x} & \frac{1}{x^2}
                \end{array}
            \right)\\
            &= 
            (5+x+x^2, \frac{1 - x + x^2}{x} \frac{1 + x + x^2}{x^2})
        \end{align*}
    \end{solucio}
\end{document}