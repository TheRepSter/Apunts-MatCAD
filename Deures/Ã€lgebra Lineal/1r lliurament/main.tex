\documentclass[a4paper, 12pt]{article}
\usepackage{deuresutils}
\usepackage{silence}
\WarningsOff

\title{Lliurament 1}
\asignatura{Àlgebra lineal}
\author{Eduardo Pérez Motato}
\niu{NIU: 1709992}
\date{12/01/2024}

\begin{document}
    \makeheader

    \begin{exercici}
        Sigui $M$ una matriu $N \times N$ amb coeficients reals tal que la suma dels coeficients de
        cada columna dona sempre el mateix número $c$, o sigui $\sum_{i=1}^{N} a_{ij}=c$ per a tota
        $1 \leq j \leq N$. Sigui $\vec{v} \in \mathbb{R}^N$ un vector en format columna tal que la
        suma dels seus coeficients és $k$. Demostreu que la suma dels coeficients de $A\vec{v}$ (on
        $\vec{v}$ és el vector escrit en columna) és $c * k$.
    \end{exercici}
    \begin{solucio}
        Definim $\vec{u} = A\vec{v}$. Volem demostrar que la suma de coeficients de $\sum_{i=1}^{N} u_i = c*k$.
        Sabem que cada coeficient de $\vec{u}$ a posició $i$ es calcula $u_i = \sum_{j=1}^{N} a_{ij}*v_j$.
        Com volem calcular la suma de tots els coeficients de $\vec{u}$, hem de calcular el sumatori
        següent:
        \begin{displaymath} 
            \sum_{i=1}^{N} u_i = \sum_{i=1}^{N} \sum_{j=1}^{N} a_{ij}*v_j  
        \end{displaymath}
        Per les propietats dels sumatoris es pot canviar l'ordre dels sumatoris, quedant la següent operació:
        \begin{displaymath}
            \sum_{j=1}^{N} \sum_{i=1}^{N} a_{ij}*v_j
        \end{displaymath}
        Com la $v_j$ no canvia al segon sumatori, es pot treure multiplicant, quedant així:
        \begin{displaymath}
            \sum_{j=1}^{N} v_j \sum_{i=1}^{N} a_{ij}
        \end{displaymath}
        Com ja sabem per l'enunciat, $\sum_{i=1}^{N} a_{ij}=c$ per tota $1 \leq j \leq N$, llavors ens queda això:
        \begin{displaymath}
            \sum_{j=1}^{N} v_j * c
        \end{displaymath}
        Ara, la $c$ no depèn de $j$, podem treure-la del sumatori com a constant:
        \begin{displaymath}
            c\sum_{j=1}^{N} v_j
        \end{displaymath}
        Finalment, com ja sabíem a l'enunciat, la suma dels coeficients de $\vec{v} = k$, i el sumatori
        que ens queda fa exactament això, una suma dels seus coeficients, per tant:
         \begin{displaymath}
            \sum_{i=1}^{N} u_i = c \sum_{j=1}^{N} v_j = c*k
        \end{displaymath}
    \end{solucio}

    \noindent A partir d'aquí fixem:
    \begin{itemize}
        \item $N$ és un enter positiu i $M$ és una matriu $N \times N$ amb coeficients reals positius
        (o zero) tal que la suma dels coeficients de cada columna dona sempre $1$.
        \item També considerarem els vectors escrits en columna per a fer les multiplicacions amb
        matrius.
        \item $p$ és un nombre real tal que $0 < p < 1$ (algunes de les afirmacions que es fan a sota
        no són certes pels casos $p = 0$ i $p = 1$).
    \end{itemize}
    
    \begin{exercici}
        Sigui $\vec{v}_1 \in \mathbb{R}^N$ un vector amb tots els coeficients positius (o zero) tal que
        la suma dels seus coeficients és $N$. A partir de $p \in (0, 1)$, definim el vector $\vec{p} = (p, p, \dots, p) \in \mathbb{R}^N$.
        Sigui $\vec{v}_2 = (1 - p)M\vec{v}_1 + \vec{p}$. Demostreu que tots els coeficients de $\vec{v}_2$
        són positius i la suma dels seus coeficients és $N$. 
    \end{exercici}
    \begin{solucio}
        A l'anterior exercici, sabíem que la suma dels coeficients de $M\vec{v} = c*k$. En aquest cas,
        sabem que $c = 1$ (ja que la suma de coeficients de cada columna de $M$ dona sempre 1) i per
        l'enunciat de l'exercici, $k = N$ (ja que la suma dels coeficients de $\vec{v}_1$ dona $N$).
        Això fa que ens quedi el següent:
        \begin{displaymath}
            \sum_{i=1}^{N} v_{2_i} = (1-p)N+\sum_{i=1}^{N}p_i
        \end{displaymath}
        De forma trivial $\sum_{i=1}^{N}p_i = N*p$, en reemplaçar-ho tenim:
        \begin{displaymath}
            \sum_{i=1}^{N} v_{2_i} = (1-p)N+p*N = (1-p+p)*N = N
        \end{displaymath}
        \hfill$\blacksquare$\\
        Per demostrar que tots els coeficients de $\vec{v}_2$ són positius hem de pensar que $(1 - p) v_{1_j} \sum_{i=1}^{N}M_{ij} \geq 0$
        per les fixacions anteriors, això ja que $(1 - p) \in (0, 1)$, $v_{1_j} \geq 0$ i $\sum_{i=1}^{N}M_{ij} \geq 0$
        (ja que a qualsevol fila dona 1).\\
        Sabent que $(1 - p) v_{1_j} \sum_{i=1}^{N}M_{ij} \geq 0$ per tota $1 \leq j \leq N$, llavors $(1 - p) v_{1_j} \sum_{i=1}^{N}M_{ij} + p \geq p > 0$
        per tota $1 \leq j \leq N$. Per allò, sent $(1 - p) v_{1_j} \sum_{i=1}^{N}M_{ij} + p$ el
        coeficient a la posició $j$ de $\vec{v}_2$, això implica que tots els coeficients de $\vec{v}_2$
        són positius.
    \end{solucio}
    
    \noindent Això permet definir, a partir de $M$ (matriu com fins ara), $p \in (0,1)$ i un vector inicial $\vec{v}_1$
    complint les condicions de l'exercici anterior, una successió de vectors amb la fórmula:
    \begin{equation}\label{eq:primera}
        \vec{v}_{k+1} = (1 - p)M\vec{v}_k + \vec{p}
    \end{equation}
    on $\vec{v}_k$ té tots els coeficients positius i amb la suma de coeficients constant igual a $N$
    per a tot $k \geq 1$.
    
    \begin{exercici}
        Demostreu que l'Equació \eqref{eq:primera} és equivalent a la igualtat:
        \begin{displaymath}
            \left(
                \begin{array}{c}
                    \vec{v}_{k+1} \\
                    \hline 1
                \end{array}
            \right)
            =
            \left(
                \begin{array}{c|c}
                    (1 - p)M & \vec{p} \\
                    \hline 0 & 1
                \end{array}
            \right)
            \left(
                \begin{array}{c}
                    \vec{v}_k \\
                    \hline 1
                \end{array}
            \right)
        \end{displaymath}
    \end{exercici}
    \begin{solucio}
        Sabem el següent:
        \begin{displaymath}
            \left(
                \begin{array}{c}
                    \vec{v}_{k+1} \\
                    \hline 1
                \end{array}
            \right)
            =
            \left(
                \begin{array}{c|c}
                    (1 - p)M & \vec{p} \\
                    \hline 0 & 1
                \end{array}
            \right)
            \left(
                \begin{array}{c}
                    \vec{v}_k \\
                    \hline 1
                \end{array}
            \right)
            = \left(
                \begin{array}{c|c}
                    (1 - p)M\vec{v}_k & \vec{p} \\
                    \hline 0 & 1
                \end{array}
            \right)
        \end{displaymath}
        I això és equivalent a ampliar ambdós vectors resultants amb un 1. 
    \end{solucio}
    
    \begin{exercici}
        Demostreu que la matriu
        $A
        =
        \left(
            \begin{array}{c|c}
                (1 - p)M & \vec{p} \\
                \hline 0 & 1
            \end{array}
        \right)$
        té $1$ com a valor propi. A més, si hi ha algun enllaç entre pàgines web (si i només si la
        matriu $M$ no és tot zeros), demostreu que aquest vector propi té alguns coeficients no nul a
        les primeres $N$ coordenades.
    \end{exercici}
    \begin{solucio}
        Primer, demostrar que té 1 com valor propi és pràcticament trivial, ja que sempre a l'última
        filera són tot zero menys l'últim valor, que és 1. Això dit equivaldria al següent:
        \begin{displaymath}
            A-\mathbb{I}_{N+1} = 
            \left(
                \begin{array}{c|c}
                    (1 - p)M-\mathbb{I}_{N} & \vec{p} \\
                    \hline 0 & 0
                \end{array}
            \right)
        \end{displaymath}
        Amb això en compte, per les propietats dels determinants (per ser específics: si els
        components d'una filera o una columna són zeros, el valor del determinant també serà zero)
        sabem que $\det(A-\mathbb{I}_{N+1}) = 0$. Com els valors propis es poden trobar mitjançant
        el polinomi característic en resoldre $\det(A-\lambda\mathbb{I}_{N+1}) = 0$, sabem que $1$
        ha de ser un valor propi.\hfill$\blacksquare$\\
        Si hi ha un enllaç entre pàgines web, sabem que per fer el $\ker(A-\mathbb{I}_{N+1})$ haurem
        de reduir per columnes (amb una ampliada per així trobar el nucli directament). L'última
        columna (la de $\vec{p}$) és la idònia per fer la reducció. Ara, la suma de cada component a
        cada columna serà $-p$ (ja que $(1-p)-1$) i per allò, en reduir la matriu, totes les
        transformacions elementals seran positives o $0$. Llavors, sempre hi tindrà com mínim un
        valor no nul.
    \end{solucio}
    
    \noindent Altres propietats de la matriu $A$ són:
    \begin{itemize}
        \item La dimensió de $\ker(A-\mathbb{I}_{N+1})$ és $1$.
        \item Si $\lambda \neq 1$ és un altre valor propi (real o complex) d'$A$, llavors $|\lambda| \leq (1 - p) < 1$,
        per tant, $1$ és el valor propi de mòdul més gran (valor propi dominant).
    \end{itemize}
    També, utilitzant el Teorema del Punt Fix de Browder\footnote{\textbf{Teorema (Browder)}: si $f : [0, 1]^N \rightarrow [0, 1]^N$
    és una aplicació contínua, llavors existeix $x \in [0, 1]^N$ tal que $f(x) = x$.} (se surt una
    mica de l'esperit del curs) es pot veure que, es pot considerar un vector propi de valor propi $1$
    amb totes les coordenades positives (els altres seran múltiples d'aquest).
\end{document}