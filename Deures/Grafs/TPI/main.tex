\documentclass[a4paper, 12pt]{article}
\usepackage{deuresutils}
\usepackage{listings}

\title{Treball Pràctic Individual}
\asignatura{Algorítmia i combinatòria en grafs. Mètodes heurístics}
\author{Eduardo Pérez Motato}
\niu{NIU: 1709992}
\date{24/05/2024}

\begin{document}
    \makeheader

    \section{Introducció}
    La finalitat d'aquest programa és calcular el cami més curt d'un punt d'un mapa entre dos punts.
    Aquest mapa es guarda en dos fitxers: \verb|Nodes.csv| que té la id del node i la seva
    geolocalització (latitud, longitud), i \verb|Carrers.csv| que tè la id del carrer i les ides
    dels nodes que el componen.\\
    Aquest problema hauria de poder trobar ràpidament la solució amb un mapa gran, i per això s'han
    fet diverses optimitzacions, com A$^*$. 
    \section{Sintaxi d'execució}
    Abans d'executar, s'ha de compilar mitjançants la seguent comanda:
    \begin{lstlisting}{bash}
cc AEstrella.c -o AEstrella -Wall -lm
    \end{lstlisting}
    i ara que està compilat, es pot executar.
    Per executar aquest programa s'ha de posar la seguent comanda a la consola estant a la carpeta
    on es té \verb|Carrers.csv| i \verb|Nodes.csv|:
    \begin{lstlisting}{bash}
./AEstrella id_node_inici id_node_final
    \end{lstlisting}
    \section{Explicació de l'estructura del codi}
    Primer s'importen les libreries i es defineixen constants (linies 3-9)\\
    Després es defineixen les estructures adients (linies 11-33), podem veure com es busca que
    estructures com \verb|INFO_ESTRELLA| tinguin un apuntador a l'estructura \verb|NODE| i que també
    \verb|NODE| pugui apuntar a \verb|INFO_ESTRELLA|. Per aixó està el typedef de la linea 11.\\
    Seguidament es defineixen les funcions necesaries (linea 35-114):
    \begin{enumerate}
        \item \verb|esferiquesACartesianes| (linea 35-39): Pasa la latitud i longitud d'un node a
        coordenades cartesianes.
        \item \verb|encuaAmbPrioritat| (linea 41-64): Mitjançants un node i l'inici de la cua, la
        funció encua amb prioritat tenint en compte la distància i la funció heuristica. 
        \item \verb|extreureDeLaCua| (linea 66-81): També fent servir un node i l'inici de la cua
        treu aquest node de la cua.
        \item \verb|distancia| (linea 83-95): Calcula la distància entre dos nodes, retorna aquesta.
        Fa servir la funció \verb|esferiquesACartesianes| i \verb|sqrt| (aquesta última definida a
        \verb|<math.h>|)
        \item \verb|buscaPosicioNode| (linea 97-114): Fa servir cerca binària per trobar la posició
        d'un node mitjançants la seva id.
    \end{enumerate}
    Finalment, tenim el main (linies 116-270):
    \begin{enumerate}
        \item Declaració de totes les variables necesaries (linies 117-131).
        \item Lectura de \verb|Nodes.csv| (linies 134-155).
        \item Lecutra de \verb|Carrers.csv| (linies 158-184).
        \item L'algorisme A$^*$ (linies 187-235).
        \item Mostrar el cami (linies 238-269).
    \end{enumerate}

    \section{Exemple de funcionament}
    Per exemple
    \begin{lstlisting}{bash}
./AEstrella 0259184345 1793441250
    \end{lstlisting}
    en el meu cas, retorna:
    \begin{lstlisting}{bash}
# La distancia de 259184345 a 1793441250 es de 507.88680 metres.
# Cami optim:
Id=0259184345 | 41.545380 | 2.106830 | Dist=0.00000
Id=0259437888 | 41.545752 | 2.106744 | Dist=42.04203
Id=0259437905 | 41.546388 | 2.106495 | Dist=115.72240
Id=0259438253 | 41.546734 | 2.107858 | Dist=235.47656
Id=0965459173 | 41.546963 | 2.107746 | Dist=262.61711
Id=0960085142 | 41.547169 | 2.107648 | Dist=286.90038
Id=1944921315 | 41.547281 | 2.107553 | Dist=301.62308
Id=2412854895 | 41.547777 | 2.107092 | Dist=368.87952
Id=1944921533 | 41.548161 | 2.107668 | Dist=433.05849
Id=1944921536 | 41.548240 | 2.107798 | Dist=446.89987
Id=1944921547 | 41.548280 | 2.107864 | Dist=454.04708
Id=1793441253 | 41.548396 | 2.108055 | Dist=474.49589
Id=1944921549 | 41.548399 | 2.108073 | Dist=476.04126
Id=1955175329 | 41.548407 | 2.108110 | Dist=479.20011
Id=1955175330 | 41.548452 | 2.108329 | Dist=498.15467
Id=1793441250 | 41.548481 | 2.108440 | Dist=507.88680
# ---------------------------------
    \end{lstlisting}
\end{document}