\pagenumbering{gobble}
\section{Càlcul amb una variable}
\newpage
\pagenumbering{arabic}
\subsection{Funcions hiperbòliques}
$\cosh(x) = \frac{e^x+e^{-x}}{2}$ "la catenària"\\
$\sinh(x) = \frac{e^x-e^{-x}}{2}$\\
$(\cosh x)^2 - (\sinh x)^2 = 1$\\
Part entera de $x$ = $[x]$ = màxim enter $\leq x$
\subsection{Limit}
\underline{Definició}: Diem que $\lim\limits_{x\rightarrow a}f(x) = L$ si i només si per a cada $\epsilon > 0$ hi ha $\delta > 0$ tal que si $0 < |x-a| < \delta$ llavors $|f(x)-L| < \epsilon$.\\
\underline{Definició}: Diem que $\lim\limits_{x\rightarrow a} f(x) = \infty$ si per a cada $M > 0$ hi ha $\delta > 0$ tal que $f(x)>M$ si $0<|x-a|<\delta$.\\
\underline{Definició}: Diem que $\lim\limits_{x\rightarrow a} f(x) = - \infty$ si per a cada $M > 0$ hi ha $\delta > 0$ tal que $f(x)<-M$ si $0<|x-a|<\delta$.\\
\underline{Definició}: $\lim\limits_{x\rightarrow \infty} f(x)=L$ si i només si per a cada $\epsilon>0$ hi ha $m>0$ tal que si $x>m$ llavors $|f(x)-L| < \epsilon$.\\
\underline{Obs}: Si hi ha un límit, és únic. 
\subsubsection{Límits laterals}
$\lim\limits_{x\rightarrow a^\pm} f(x)$. Si $a^+$ ens referim que $x > a$ i es tractaria de un \underline{límit per la dreta}. En cas de que sigui $a^-$ seria el contrari. \\
$\lim\limits_{x\rightarrow a} f(x) = L \Longleftrightarrow \lim\limits_{x\rightarrow a^+} f(x) = \lim\limits_{x\rightarrow a^-} f(x) = L$\\
Recordem que $\lim\limits_{x\rightarrow 0} (1+x)^\frac{1}{x} = e$\\
\subsection{Continuitat}
Diem que $f$ és continua en $a$ si $\lim\limits_{x\rightarrow a} f(x) = f(a)$ $f$ és continua a $D$ quan és continua a tots els $x \in D$. $f$ contínua en $a \Longleftrightarrow \forall \epsilon > 0$ hi ha un $\delta > 0$ tal que a $|x-a| < \delta$ llavors $|f(x) - f(a)| < \epsilon \Longleftrightarrow$ per a tota successió ($x_n)$ amb $\lim\limits_{n\rightarrow \infty} x_n = a$ se compleix que $\lim\limits_{n\rightarrow \infty} f(x_n) = f(a)$\\
Tipus de discontinuïtat
\begin{itemize}
    \item Evitable (existeix $\lim\limits_{x\rightarrow a} f(x) \in \mathbb{R}$ peró aquest límit $\neq f(x)$)
    \item Salt finit (No existeix el límit) 
    \item Salt infinit (Un limit lateral anirà a més o menys infinit)
\end{itemize}
\subsection{Teorema de Bolzano}
Pues me la agarras con la mano. Te falta calle.\\
$f: [a, b] \rightarrow \mathbb{R}$ contínua amb $f(a)f(b)<0$ aleshores hi ha $c\in (a,b)$ tal que $f(c) = 0$. Garanteix la existència de un punt, però podria haver-hi més.
\subsection{Teorema de Weiestrass}
Pues me la agarras por detrás. Te falta calle.\\
$f: [a, b] \text{ tancat i fitat. } \rightarrow \mathbb{R}$ llavors hi ha extrems absoluts, es a dir $m, M \in [a, b]$ tal que $f(m) \leq f(x) \leq f(M)$ per a cada $x \in [a, b]$.
\subsection{Derivada en un punt}
Diem que $f$ és derivable a $x_0$ si existeix $\lim\limits_{x\rightarrow x_0} \frac{f(x)-f(x_0)}{x-x_0} = \lim\limits_{h\rightarrow 0} \frac{f(x_0+h)-f(x_0)}{h} = f'(x)$\\
\underline{Obs:} Si $f$ es derivable a $x_0$, llavors $f$ ha de ser continua a $x_0$.
\subsection{Recta tangent}
A la gràfica d'una funció al punt $(a, f(a))$. El pendent es $f'(a)$ si hi ha derivada, sino no hi ha pendent.
\subsection{Propietats de la derivada}
$f, g$ derivables en $a$, $\alpha \in \mathbb{R}$
\begin{itemize}
    \item $(f \pm g)'(a) = f'(a) \pm g'(a)$
    \item $(\alpha f)'(a) = \alpha f'(a)$
    \item $(f\times g)'(a) = f'(a)g(a) + f(a)g'(a)$
    \item $(\frac{f}{g})'(a) = \frac{f'(a)g(a) - f(a)g'(a)}{(g(a))^2}$
    \item Si $f$ derivable en $a$ i $g$ derivable en $f(a) \Rightarrow$ $(g \circ f)'(a) = g'(f(a))f'(a)$
\end{itemize}
Volem derivar $f(x)^{g(x)}$. Recordem $A = e^{\log{A}}$, llavors $(f(x)^{g(x)})' = (e^{g(x)\log{f(x)}})' = f(x)^{g(x)}(g'(x)\log{f(x)} + g(x) \frac{f'(x)}{f(x)}$. La $f$ ha de ser positiva!\\
\underline{Definició}: màxim relatiu i mínim relatiu:\\
Diem que la funció $f$ té un màxim relatiu a $x_0$ si hi ha un interval concentrat a $x_0$ $(x_0 - \varepsilon, x_0 + \varepsilon),$ $\varepsilon > 0$, tal que $f(x_0) \geq f(x)$ $\forall |x - x_0| < \varepsilon$.\\
Diem que la funció $f$ té un mínim relatiu a $x_0$ si hi ha un interval concentrat a $x_0$ $(x_0 - \varepsilon, x_0 + \varepsilon),$ $\varepsilon > 0$, tal que $f(x_0) \leq f(x)$ $\forall |x - x_0| < \varepsilon$.\\
Els punts candidats a extrems relatius son aquells on la derivada de la funció $f$ val $0$ o no existeix.\\
Pel Teorema de Weistrass, si $f: [a, b] \rightarrow \mathbb{R}$ contínua hi ha extrems absoluts. Els candidats a extrems absoluts son $\{a, b, \text{Punts on } f'=0 \text{ o bé } f' \text{ no existeix}\}$
\subsection{Teorema de Rolle}
$f: [a, b] \rightarrow \mathbb{R}$ contínua i derivable en $(a, b)$ tal que $f(a) = f(b)$ aleshores hi ha $c \in (a, b)$ tal que $f'(c) = 0$.\\
Demostració:\\
Pel Teorema de Weistrass sabem que hi ha màxim absolut i mínim absolut. Si el màxim i el mínim s'assoleixen a $x_0 \in (a, b) \Rightarrow f'(x_0) = 0$. Resolt :).\\
Si el màxim i el mínim es prenen als extrems de l'interval, com que $f(a)=f(b)$ tenim que $f$ és constant $\Rightarrow f'(x) = 0$ $\forall x \in (a, b)$.
\subsection{Teorema del valor mitjà (de Lagrange)}
$f: [a, b] \rightarrow \mathbb{R}$ contínua i derivable en $(a, b)$. Aleshores existeix $c \in (a, b)$ tal que $f'(c) = \frac{f(b) - f(a)}{b-a} \leftrightarrow f(b) - f(a) = f'(c)(b-a)$.\\
Demostració:\\
Definim $g(x) = f(x) - (\frac{f(b)-f(a)}{b-a})(x-a)$ $g$ continua i derivable (igual que $f$). $g(a) = f(a) - (\frac{f(b)-f(a)}{b-a})(a-a) = f(a)$ i $g(b) = f(b) - (\frac{f(b)-f(a)}{b-a})(b-a) = f(a)$. Llavors $g(a) = g(a)$. Pel Teorema de Rolle, $\exists c \in (a,b)$ amb $g'(c) = 0$, $g'(x) = f'(x) - (\frac{f(b)-f(a)}{b-a}$. Llavors $f'(c) = \frac{f(b)-f(a)}{b-a}$.\\
\underline{Corol·lari}: Creixement i decreixement d'una funció a un interval.
\begin{itemize}
    \item Si $f'(x) \geq 0$ $\forall x \in I \Longleftrightarrow f$ creixent a $I$.
    \item Si $f'(x) \leq 0$ $\forall x \in I \Longleftrightarrow f$ decreixent a $I$.
    \item Si $f'(x) = 0$ $\forall x \in I \Longleftrightarrow f$ constant a $I$.
\end{itemize}
\underline{Observació} Si $f'(x) > 0$ a $I \Rightarrow f$ estrictament creixent a $I$. Anàleg amb les decreixents.\\
Extrems relatius criteris de la segona derivada.\\
$f'(x_0) = 0$ i $f''(x_0) > 0 \rightarrow$ mínim relatiu.\\
$f'(x_0) = 0$ i $f''(x_0) < 0 \rightarrow$ màxim relatiu.\\
És un criteri útil? Sí. És imprescindible? No.\\
\subsection{Convexitat i concavitat.}
$\bigcup$ CONVEXA ($f''(x) \geq 0$ a $I$)i $\bigcap$ CÓNCAVA ($f''(x) \leq 0$ a $I$). El criterio bueno. Como siempre, la cerveza fria y la concavidad $\bigcap$.
\subsection{Regla de L'Hôpital}
\underline{Teorema} (L'Hôpital):\\
$f, g$ derivables amb $g'(x) \neq 0$ en un interval que conté el punt $a$ (excepte, pot ser, el punt $a$) tals que $\lim\limits_{x \rightarrow a} \frac{f'(x)}{g'(x)} = L \in \mathbb{R} \cup {-\infty, \infty}$.
\begin{itemize}
    \item Si $\lim\limits_{x \rightarrow a} f(x) = \lim\limits_{x \rightarrow a} g(x) = 0 \Rightarrow \lim\limits_{x \rightarrow a} \frac{f(x)}{g(x)} = \lim\limits_{x \rightarrow a} \frac{f'(x)}{g'(x)}$
    \item Si $\lim\limits_{x \rightarrow a} f(x) = \pm \infty$, $\lim\limits_{x \rightarrow a} g(x) = \pm \infty \Rightarrow \lim\limits_{x \rightarrow a} \frac{f(x)}{g(x)} = \lim\limits_{x \rightarrow a} \frac{f'(x)}{g'(x)}$
\end{itemize}
També és vàlid per a límits laterals i límits a l'infinit. 
\subsection{Polinomis de Taylor}
Les funcions més sencilles són els polinomis. Donada una funció $f$ en un interval $I$ volem trobar un polinomi $P$ de manera que $P$ sigui proper a $f$ a l'interval $I$. Sigui $a \in I$ volem que $f(a) = P(a); f'(a) = P'(a); \dotsb; f^{(N)}(a) = P^{(N)}(a)$. Hi ha un únic polinomi de quin $\leq N$ que compleix aquesta propietat.\\
\underline{Definició}: Sigui $f$ una funció $N$ vegades derivable a $I$. Sigui $a \in I$. El polinomi de Taylor d'ordre $N$ (o grau $N$) de $f$ en el punt $a$ és:
\begin{displaymath}
    \underbrace{P_N[f, a](x) = P_N(x)}_\text{Notació} = \sum\limits_{j = 0}^N \frac{f^{(j)}(a)}{j!}(x-a)^j
\end{displaymath}
La propietat fonamental que caracteritza el polinomi de Taylor de grau $N$ de $f$ en el punt $a$
\begin{displaymath}
    \lim_{x \rightarrow a}\frac{f(x)-P_N(x)}{(x-a)^N} = 0
\end{displaymath}
\underline{Teorema de Taylor}:\\
Sigui $f$ $N+1$ vegades derivable en $I$. Sigui $a \in I$. Aleshores $\overbrace{f(x) = P_N(x) + R_N(x)}^\text{Això no te cap merit}$. On $R_N = \frac{f^{(N+1)}(c_x)}{(N+1)!}(x-a)^{N+1}$ on $c_x$ és un punt entre $x$ i $a$.\\
D'aquest expressió del residu se'n diu residu de Lagrange. Hi ha altres expressions, de Cauchy i de integral, aquestes dos no les utilitzarem. El error serà $|R_N(x)|$. Si fem el limit abans anotat sabem que
\begin{displaymath}
    \lim_{x \rightarrow a}\frac{f(x)-P_N(x)}{(x-a)^N} = \lim_{x \rightarrow a}\frac{R_N(x)}{(x-a)^N} = \lim_{x \rightarrow a}\frac{\frac{f^{(N+1)}(c_x)}{(N+1)!}(x-a)^{N+1}}{(x-a)^N} = \lim_{x \rightarrow a} \frac{f^{(N+1)}(c_x)}{(N+1)!}(x-a) = 0
\end{displaymath}
\underline{Observació}:\\
$\lim\limits_{x \rightarrow a} \frac{f(x)-P_N(X)}{(x-a)^N} = 0 \Rightarrow$ \underline{NOTACIÓ}: $f(x) - P_N(x) = o((x-a)^N)$ o $f(x) = P_N(x) + o((x-a)^N)$.
\underline{Infinitessims}:\\
Diem que $f$ és un infinitèssim quan $x \rightarrow a$ si $\lim\limits_{x \rightarrow a} f(x) = 0$.\\
Diem que $f$ i $g$ són inifitèssim equivalents quan $x \rightarrow a$ si $\lim\limits_{x \rightarrow a} f(x) = \lim\limits_{x \rightarrow a} g(x) = 0$ i també $\lim\limits_{x \rightarrow a} \frac{f(x}{g(x)} \in \mathbb{R} \setminus 0$.
\subsection{Integral de Riemann}
\underline{Definició}: Una partició de l'interval $[a, b]$ és un nombre finit de punts $x_0 = a < x_1 < x_2 < x_3 < \dots < x_n = b$. Aixó es possa com intervals de la seguent forma: $[x_0, x_1], [x_1, x_2], [x_2, x_3],\dots, [x_{n-1}, x_n]$. Llavors la mida es $\delta(P) := max(x_0 - x_1, x_1 - x_2, x_2 - x_3,\dots, x_{n-1} - x_n)$.\\
Sigui $f: [a, b] \rightarrow \mathbb{R}$ fitada i $P$ una partició de $[a, b]$:\\
$M_i = \sup\limits_{x \in [x_i-\delta, x_i]} f(x); i = 1, 2,\dots, n$\\
$M_i = \inf\limits_{x \in [x_i-\delta, x_i]} f(x); i = 1, 2,\dots, n$
Suma superior de de $f$ en $[a, b]$ respecte la partició $P$:\\
$S(f, P) = \sum\limits_{i=1}^n M_i(x_i-x_{i-1}$\\
$I(f, P) = \sum\limits_{i=1}^n m_i(x_i-x_{i-1}$\\
$I(f, P) \leq I(f, P') \leq S(f, P') \leq S(f, P)$\\
$P'$ més fina que $P$\\
\underline{Definició}:\\
$f: [a, b] \rightarrow \mathbb{R}$ fitada diem que és integrable en el sentit de Riemann (ho denotem per $f \in \textgoth{R}[a, b]$) quan $\sup\limits_P I(f, P) = \inf\limits_P S(f, P) = \lim\limits_{\delta{P} \rightarrow 0} I(f,P) = \lim\limits_{\delta{P} \rightarrow 0} S(f,P)$. Quan $f \in \textgoth{R}[a, b]$ ho denotem com $\int_a^b f$\\
\underline{Teorema}:\\
$f:[a, b] \rightarrow \mathbb{R}$ fitada. Si $f$ continua, $f$ continua llevat d'un nombre finit de punts o $f$ és monòtona llavors $f \in \textgoth{R}[a, b]$.\\
Sabem que l'àrea $= \int_a^b f = \int_a^b f dx$\\
\underline{Propietats de la integral de Riemann}:\\
Tenim $f, g \in \textgoth{R}[a, b]$, i $\alpha \in \mathbb{R}$\\
\begin{enumerate}
    \item $f\pm g \in \textgoth{R}[a, b]$ i $\int_a^b (f\pm g) = \int_a^b f \pm \int_a^b g$
    \item $\alpha f \in \textgoth{R}[a, b]$ i $\int_a^b \alpha f = \alpha \int_a^b f$.
    \item $a < c < b \Rightarrow \int_a^b f = \int_a^c f + \int_c^b f$
    \item Si $f \geq 0 \Rightarrow \int_a^b f \geq 0$ (Si $f \geq g \Rightarrow \int_a^b f \geq \int_a^b g$)
    \item Si $m \leq f \leq M \Rightarrow m(b-a) \leq \int_a^b f \leq M(b-a)$
    \item $|\int_a^b f| \leq \int_a^b |f|$
    \item $f \in \textgoth{R}[a, b] \Rightarrow |f| \in \textgoth{R}[a, b]$
    \item $fg \in \textgoth{R}[a, b]$ ATENCIÓ! $\int_a^b fg \neq \int_a^b f \int_a^b g$
\end{enumerate}
Si una funció es integral de Riemann i la canvio en un punt, aquesta funció seguira sent integral de Riemann i el valor de la integral seguira sent.\\
\subsection{Sumes de Riemann}:\\
$f: [a, b] \rightarrow \mathbb{R}$ fitada\\
$P = \{a = x_0, x_1, x_2, \dots, x_n = b\}$\\
$Z_i \in [x_{i-1}, x_i]$\\
Una suma de Riemann respecte la partició $P$ es $\sum\limits_{i=1}^n f(Z_i)(x_i-x_{i-1})$\\
\underline{Teorema}: $f \in \textgoth{R}[a, b]$\\
$\lim\limits_{\delta(P) \rightarrow 0} \underbrace{\sum\limits_{i=1}^n f(z_i)(x_i-x_{i-1})}_\text{Suma de Riemann} = \int_a^b f$\\
\underline{Corol·lari}:\\
Considerem $[0, 1]$ i la partició $P_n = \{0, \frac{1}{n}, \frac{2}{n}, \frac{3}{n}, \dots, \frac{n-1}{n}, 1\}$.\\
$\int_0^1 f = \lim\limits_{n \rightarrow \infty} (\frac{f(\frac{1}{n})}{n}+\frac{f(\frac{2}{n})}{n}+\frac{f(\frac{3}{n})}{n}+\dots+\frac{f(\frac{n-1}{n})}{n}+\frac{f(1)}{n})$\\
\subsection{Integrals impropies en sentit de Riemann}
En la integració de Riemann hi ha dos aspectes a destacar:
\begin{itemize}
    \item Funcions fitades
    \item Intervals de longitud finita
\end{itemize}
Si volem donar sentit als intervals de longitud infinita o a les funcions no fitades tractarem $[a, b)$, $(a, b]$, $(a, b)$ on $b \in \mathbb{R}\cup \{\infty\}$ i $a \in \mathbb{R}\cup \{-\infty\}$.\\
\underline{Definició}: Diem que $f$ és funció localment integrable a l'interval $[a, b)$ si $f\in \textgoth{R}[a, c]$ per a cada $c < b$. Idem. per a la resta de tipus de intervals.\\
\underline{Definició}: Sigui $f$ localment integrable a $[a, b)$, considerem $\lim\limits_{c\rightarrow b} \int_a^cf(x)dx$.
\begin{itemize}
    \item Si aquest límit està a $\mathbb{R}$ direm que la integral impròpia $\int_a^bf(x)dx$ és convergent i el seu valor és aquest límit.
    \item Si el límit és $\pm \infty$ direm que és divergent cap a $\pm \infty$.
    \item Si el límit no existeix direm que la integral impròpia no existeix.
\end{itemize}
\underline{Obs}: Tenim $f(x) \geq 0 \forall x \in [a, b)$  tal que $f$ es localment integrable a $[a, b)$. Llavors $\int_a^b f(x) dx$ és convergent $\Leftrightarrow \int_a^b f(x) dx < \infty$\\
Sigui $f$ localment integrable a $(a, b)$ tal que $a < \alpha < \beta < b$, vol dir $f \in \textgoth{R}[\alpha, \beta]$ diem que $\int_a^b f(x)dx$ és convergent en el sentit impropi de Riemann si fixat $c \in (a, b)$ es compleix $\int_a^c f(x)dx$ és convergent i $\int_c^b f(x)dx$ és convergent. En aquest cas $\int_a^b f(x)dx = \int_a^c f(x)dx + \int_c^b f(x)dx$\\
\underline{Criteri de comparació}: $0 \leq f(x) \leq g(x)$ quan $x \in [a, b)$ $f$ i $g$ dues funcions localment integrables a $[a, b)$ llavors:
\begin{displaymath}
    \int_a^b g(x)dx < \infty \Rightarrow \int_a^b f(x) dx < \infty
\end{displaymath}
\begin{displaymath}
    \int_a^b f(x)dx = \infty \Rightarrow \int_a^b g(x) dx = \infty
\end{displaymath}
\underline{Criteri de comparació al límit}: $0 \leq f(x)$ $0 < g(x) \forall$ $x \in [a, b)$ $f$ i $g$ dues funcions localment integrables a $[a, b)$. Suposem que $\lim\limits_{x \rightarrow b} \frac{f(x)}{g(x)} = L \in [0, \infty]$.\\
\begin{enumerate}
    \item Si $0 < L < \infty$ llavors tenen el mateix caracter.
    \item Si $L = 0$ llavors $\int_a^b g(x) dx < \infty \Rightarrow \int_a^b f(x) dx < \infty$
    \item Si $L = \infty$ llavors $\int_a^b f(x) dx < \infty \Rightarrow \int_a^b g(x) dx < \infty$
\end{enumerate}
\underline{Criteri de Dirichlet}: $f, g$ dues funcions contínues a $[a, b)$ i $g$ amb derivada contínua. Suposem que \begin{itemize}
    \item $|\int_a^x f(t) dt| \leq K$ per a cada $x \in (a, b)$
    \item $g$ monòtonament decreixent tal que $g(t) \longrightarrow 0$ quan $t \rightarrow b$
\end{itemize}
Aleshores $\int_a^b f(t) g(t) dt$ és convergent.\\
\subsection{Sèries de potències}
Una sèrie de potencies centrada en $a$ és una expressió del tipus $\sum\limits_{n=0}^\infty a_n(x-a)^n$.\\
El radi de convergencia d'una sèrie de potències és el valor $R$ tal que
\begin{displaymath}\begin{cases}
    \text{Si } |x_0 - a| < R \text{ llavors } \sum\limits_{n=0}^\infty a_n(x-a)^n \text{ és convergent.}\\
    \text{Si } |x_0 - a| > R \text{ llavors } \sum\limits_{n=0}^\infty a_n(x-a)^n \text{ és divergent.}\\
    \text{Quan } |x_0 - a| = R \text{ cal estudiar en cada cas si les sèries}\\
    \text{correspondents són convergents o divergents.}
\end{cases}\end{displaymath}

Si aquest límits existeixen sabem que $R = \frac{1}{\lim\limits_{n\rightarrow \infty} |\frac{a_{n+1}}{a_n}|} = \frac{1}{\lim\limits_{n\rightarrow \infty} \sqrt{|a_n|}}$
\newpage