\documentclass[a4paper,12pt]{article}
\usepackage{tabularx}
\usepackage{titling}
\usepackage{listings}
\usepackage{hyperref}
\usepackage{helvet}
\renewcommand{\familydefault}{\sfdefault}

\usepackage{lstlangarm}

\lstset{
	belowcaptionskip=1\baselineskip,
    frame=single,
	language=[x86]Assembler,
    frameround=tttt,
    xleftmargin=\parindent,
    tabsize=4,
    numbers=left,
    numbersep=8pt,
    stepnumber=1,
    numberstyle=\tiny\color{Gray}, 
    columns = fullflexible,
	literate=%
         {á}{{\'a}}1
         {à}{{\`a}}1
         {í}{{\'i}}1
         {ì}{{\`i}}1
         {é}{{\'e}}1
         {è}{{\`e}}1
         {ú}{{\'u}}1
         {ù}{{\`u}}1
         {ó}{{\'o}}1
         {ò}{{\`o}}1
		 {ç}{{\c{c}}}1
		 {·}{{$\cdot$}}1
		 {ü}{{\"u}}1
		 {0}{{{\color{Number}0}}}1
		 {1}{{{\color{Number}1}}}1
		 {2}{{{\color{Number}2}}}1
		 {3}{{{\color{Number}3}}}1
		 {4}{{{\color{Number}4}}}1
		 {5}{{{\color{Number}5}}}1
		 {6}{{{\color{Number}6}}}1
		 {7}{{{\color{Number}7}}}1
		 {8}{{{\color{Number}8}}}1
		 {9}{{{\color{Number}9}}}1
		 {w4}{{{\color{Identifier}w4}}}1
		 {2P}{{{\color{Identifier}2P}}}1
		 {r2}{{{\color{Identifier}r2}}}1
}
\AddToHook{cmd/section/before}{\clearpage}
\renewcommand{\contentsname}{Índex}

\begin{document}
\pagenumbering{gobble}
\begin{titlepage}
\scshape

\centering
\vspace{1cm}
\raisebox{-\baselineskip}{\rule{\textwidth}{1px}}
\rule{\textwidth}{1px}
\vspace{0.01cm}

{\huge{{Apunts}}}\par \vspace{0.1cm}
La primera meitat del 1r curs

\rule{\textwidth}{2px}

\vspace{1cm}

\begin{tabularx}{\textwidth}{X r}
 & Autor:\\
 & \large{EDUARDO PÉREZ MOTATO} \\
\end{tabularx}

\vspace{1.3cm}


\vfill
Setembre - Desembre 2023

\end{titlepage}
\pagenumbering{arabic}
\tableofcontents
\section{Descripció}
Aquest projecte tracta de crear un 4 en ratlla en ensamblador. Per aixó hem fet servir un archiu \verb|.c| ja creat per l'apartat gràfic, del qual no parlarem, i unes subrutines dintre del \verb|.asm|. Aquest projecte ha sigut subdividit en nivells de dificultat i per allò a l'apartat subrutines està separat per nivells.
\section{Objectius}
El joc del 4 en ratlla té com a objectiu principal col·locar quatre fitxes iguals en línia, ja sigui horitzontal, vertical o diagonal, abans que l'adversari ho faci. Això implica prevenir l'altra persona de formar aquesta línia i, al mateix temps, crear estratègies per aconseguir la teva.
\section{Variables utilitzades}
\begin{itemize}
    \item \lstinline|DWORD[colScreen]| (4 \textit{bytes}): Columna on volem posicionar el cursor a la pantalla.
    \item \lstinline|DWORD[rowScreen]| (4 \textit{bytes}): Fila on volem posicionar el cursor a la pantalla.
    \item \lstinline|DWORD[rowScreenIni]| (4 \textit{bytes}): Fila de la primera posició de la matriu a la pantalla.
    \item \lstinline|DWORD[colScreenIni]| (4 \textit{bytes}): Columna de la primera posició de la matriu a la pantalla.
    \item \lstinline|DWORD[player]| (4 \textit{bytes}): Variable que indica el jugador al que correspon tirar.
    \item \lstinline|BYTE[mBoard]| (1 \textit{byte}): Matriu 6x7 on tenim les dades del taulell.
    \item \lstinline|BYTE[tecla]| (1 \textit{byte}): Variable on s’emmagatzema la tecla pitjada en codi ascii.
    \item \lstinline|BYTE[colCursor]| (1 \textit{byte}): Variable que indica la columna en la que es troba el cursor.
    \item \lstinline|DWORD[row]| (4 \textit{bytes}): Fila per a accedir a la matriu mBoard.
    \item \lstinline|BYTE[col]| (1 \textit{byte}): Columna per a accedir a la matriu mBoard.
    \item \lstinline|DWORD[pos]| (4 \textit{bytes}): Índex per a accedir a la matriu mBoard.
    \item \lstinline|DWORD[inaRow]| (4 \textit{bytes}): Comptador per a saber el nombre de fitxes en ratlla.
    \item \lstinline|DWORD[row4Complete]| (4 \textit{bytes}): Indicador de si hem arribat a 4 fitxes en ratlla o no.
\end{itemize}
\section{Subrutines} \label{subrut}
\subsection{Nivell bàsic}
Aquest nivell tracta de posar a punt totes les subrutines base per aixi tindre un mínim de funcionalitat.
\subsubsection{showCursor}
\begin{lstlisting}[firstnumber=246]
showCursor:
	push rbp
	mov  rbp, rsp
	
	push rax
	push rbx
	mov eax, 0
	mov ebx, 0

	mov eax, DWORD [RowScreenIni]
	mov DWORD [rowScreen], eax
	mov al, BYTE [colCursor]
	sub eax, 'A'  
	shl eax, 2
	add eax, DWORD [ColScreenIni]
	mov DWORD [colScreen], eax

	call gotoxy

	pop rbx
	pop rax

	mov rsp, rbp
	pop rbp
	ret
\end{lstlisting}
Aquesta subrutina mostra la localització del cursor en pantalla. Per allò, fa servir \lstinline|rowScreen|, \lstinline|colScreen| i els seus corresponents \lstinline|rowScreenIni|, \lstinline|colScreenIni|. En aquesta subrutina no va haver cap dificultat.
\clearpage
\subsubsection{showPlayer}
\begin{lstlisting}[firstnumber=294]
showPlayer:
	push rbp
	mov  rbp, rsp
	
	push rax

	mov eax, DWORD [player]
	add eax, 48  
	mov DWORD [rowScreen], 23
	mov DWORD [colScreen], 20

	call gotoxy
	mov dil, al  	
	call printch

	pop rax

	mov rsp, rbp
	pop rbp
	ret
\end{lstlisting}
Aquesta subrutina mostra el jugador que té que jugar a la posició de la pantalla on està. Aquesta subrutina no va comportar cap problema.
\clearpage
\subsubsection{showBoard}
\begin{lstlisting}[firstnumber=337]
showBoard:
	push rbp
	mov  rbp, rsp

	push rsi
	push rdi
	push rax
	 
	mov eax, 0
	mov rdi, 0
	mov rsi, 0
		
	mov DWORD [rowScreen], 8
	buclefilas:
		cmp DWORD [rowScreen], 20
		jge fi_showBoard
		add DWORD [rowScreen], 2
		mov DWORD [colScreen], 4
	buclecolumnas:
		cmp DWORD [colScreen], 32
		jge buclefilas
		add DWORD [colScreen], 4
		mov dil, BYTE [mBoard + eax]
		inc eax
		call gotoxy
		call printch
		jmp buclecolumnas

	fi_showBoard:

	call showPlayer

	pop rax
	pop rdi
	pop rsi

	mov rsp, rbp
	pop rbp
	ret
\end{lstlisting}
Aquesta subrutina ens mostra el tauler amb les fitxes que hagim col·locat. Aquesta va ser la tasca que més dificultats ens va comportar el nivell bàsic i hi vam trobar dos problemes principals. El primer va ser un error de numeració de files i columnes que al calcular malament el marge feia que col·loquèssim punts fora i el segon va ser que no accediem per iteracció a la matriu \lstinline|mBoard| i s'omplia de \lstinline|'.'| malgrat tindre fitxes col·locades. El primer el vam arreglar dibuixant la matriu en paper per imaginar-ho millor i el segon comprovant-ho tot fins que ens vam adonar de que no havíem fet us de \lstinline|mBoard|.
\subsubsection{moveCursor}
\begin{lstlisting}[firstnumber=406]
moveCursor:  
	push rbp
	mov  rbp, rsp
	call showCursor  
	bucle_moveCursor:
		call getch
		cmp BYTE [tecla], 'j'
		je moveLeft
		cmp BYTE [tecla], 'k'
		je moveRight
		cmp BYTE [tecla], ' '
		je fi_moveCursor
		cmp BYTE[tecla], 27  
		je fi_moveCursor
		jmp bucle_moveCursor

	moveLeft:
		cmp BYTE [colCursor], 'A'
		je fi_moveCursor
		dec BYTE [colCursor]
		jmp fi_moveCursor

	moveRight:
		cmp BYTE [colCursor], 'G'
		je fi_moveCursor
		inc BYTE [colCursor]

	fi_moveCursor:
		call showCursor

	mov rsp, rbp
	pop rbp
	ret
\end{lstlisting}
Aquesta subrutina permet moure el cursor una sola posició a esquerra o a dreta utilitzant les tecles \lstinline|'j'| i \lstinline|'k'| respectivament. No ens va representar molta dificultat donat que aquesta tasca només és com una prèvia a \lstinline|moveCursorContinous|.
\clearpage
\subsubsection{moveCursorContinuous}
\begin{lstlisting}[firstnumber=454]
moveCursorContinuous:
	push rbp
	mov  rbp, rsp

	bucle_moveCursorContinuous:
		call moveCursor
		cmp BYTE [tecla], ' '
		je fi_moveCursorContinuous
		cmp BYTE [tecla], 27  
		je fi_moveCursorContinuous
		jmp bucle_moveCursorContinuous

	fi_moveCursorContinuous:

	mov rsp, rbp
	pop rbp
	ret
\end{lstlisting}
Aquesta subrutina utilitza \lstinline|moveCursor| però elimina la limitació de que el moviment sigui només d'una posició. Vam trobar problemes amb evitar que ens ensortíssim de la matriu i el cursor arroseguès el marge de la matriu. Vam solucionar-ho afegint restriccions de comprobació que no permetessin al cursor arribar al marge dret.
\clearpage
\subsubsection{calcIndex}
\begin{lstlisting}[firstnumber=485]
calcIndex:
	push rbp
	mov  rbp, rsp

	push rax
	push rbx
	
	mov eax, DWORD [row]
	imul eax, 7
	mov bl, BYTE [col]
	sub bl, 'A'
	add eax, ebx
	mov DWORD [pos], eax

	pop rbx
	pop rax
	 
	mov rsp, rbp
	pop rbp
	ret
\end{lstlisting}
Aquesta subrutina ens permet accedir a cada component de la matriu mBoard via la fórmula canònica del tipus $i \times m + j$, on $i$ és la fila a què es vol accedir, \lstinline|DWORD[row]|; $m$ és la mida de les files ($7$, ja que \lstinline|mBoard| és una matriu $6\times 7$); i $j$ és la columna a què es vol accedir, \lstinline|BYTE[col]| (es passa de forma alfabètica a forma numèrica restant-li el valor ascii de \lstinline|'A'|). No van sorgir problemes majors en la realització d'aquesta subrutina. 
\clearpage
\subsubsection{putPiece}
\begin{lstlisting}[firstnumber=533]
putPiece:  
	push rbp
	mov  rbp, rsp

	push rax
	push rbx
	
	mov eax, 0
	mov ebx, 0
	
	call showBoard
	call moveCursorContinuous
	
	cmp BYTE [tecla], ' '
	jne fi_putPiece
	
	mov al, BYTE [colCursor]
	mov BYTE [col], al
	mov DWORD [row], 5
	
	bucle_putPiece:
		call calcIndex
		mov eax, DWORD [pos]
		mov bl, BYTE [mBoard + eax]  
		cmp bl, '.'  
		je colocacion
		cmp DWORD [row], 0  
		jl fi_putPiece
		dec DWORD [row]
		jmp bucle_putPiece
	
	colocacion:
		mov BYTE [mBoard + eax], 'X'  
		call showBoard  
		
	fi_putPiece:
		
	pop rbx
	pop rax

	mov rsp, rbp
	pop rbp
	ret
\end{lstlisting}
Aquesta subrutina permet col·locar una fitxa a \lstinline|mBoard|. Quan es prem l'espai, es col·loca la fitxa tot controlant que no hi hagi una fitxa existent. Un problema que vam tenir va ser que aquesta subrutina, en un principi, no comprovava que la columna es trobés plena i això resultava en "desbordaments". Ho vam solucionar programant millor el desbordaments.
\clearpage
\subsection{Nivell mig}
Aquest nivell implementa mes funcionalitats, la posibilitat de dos jugadors i una versió del joc força simplificat pero funcional.
\subsubsection{checkRow}
\begin{lstlisting}[firstnumber=524]
checkRow:
	push rbp
	mov  rbp, rsp

	push rax
	push rbx
	
	mov rax, 0
	mov rbx, 0
	mov DWORD [inaRow], 0
	
	mov eax, DWORD [pos]  
	mov bl, BYTE [mBoard + eax]	 
	bucle_vertical:
		cmp eax, 42  
		jge fi_vertical
		cmp bl, BYTE [mBoard + eax]
		jne fi_vertical
		add eax, 7  
		inc DWORD [inaRow]  
		cmp DWORD [inaRow], 4
		jne bucle_vertical
		mov DWORD [row4Complete], 1  
		
	fi_vertical:
	pop rbx
	pop rax

	mov rsp, rbp
	pop rbp
	ret
\end{lstlisting}
Aquesta subrutina comprova si algun dels dos jugadors ha guanyat o no. Només ho comprova en vertical així que no té una dificultat excesiva donat que aquesta tasca és també una prèvia de la que porta el mateix nom al nivell avançat. Per allò, no van sortir problemes.
\clearpage
\subsubsection{putPiece}
\begin{lstlisting}[firstnumber=770]
putPiece:  
	push rbp
	mov  rbp, rsp

	push rax
	push rbx
	mov eax, 0
	mov ebx, 0
	call showBoard
	call moveCursorContinuous
	cmp BYTE [tecla], ' '
	jne fi_putPiece
	
	mov al, BYTE [colCursor]
	mov BYTE [col], al
	mov DWORD [row], 5
	
	bucle_putPiece:
		call calcIndex
		mov eax, DWORD [pos]
		mov bl, BYTE [mBoard + eax]  
		cmp bl, '.'  
		je colocacion
		cmp DWORD [row], 0  				
		jl fi_putPiece
		dec DWORD [row]
		jmp bucle_putPiece
	
	colocacion:
		cmp DWORD [player], 1
		jne jugador2
		mov BYTE [mBoard + eax], '0'  
		jmp fi_putPiece
		jugador2:
			mov BYTE [mBoard + eax], 'X'  
		
	fi_putPiece:
		call showBoard  
		call checkRow  

	pop rbx
	pop rax

	mov rsp, rbp
	pop rbp
	ret
\end{lstlisting}
Aquesta subrutina és una millora de la del nivell bàsic. Aquesta versió controla el jugador actual per tal de mostrar per pantalla la fitxa \lstinline|'O'| o \lstinline|'X'|, segons si és el jugador $1$ o el $2$, respectivament. No van sorgir problemes majors en la realització d'aquesta subrutina.
\subsubsection{put2Players}
\begin{lstlisting}[firstnumber=841]
put2Players:
	push rbp
	mov  rbp, rsp

	mov DWORD [player], 1  
	
	bucle_put2Players:
		call showPlayer  
		call putPiece  
		cmp BYTE [tecla], ' '
		jne fi_put2Players  
		cmp DWORD [row4Complete], 1
		je fi_put2Players
		cmp DWORD [player], 2
		je fi_put2Players
		mov DWORD [player], 2
		jmp bucle_put2Players
		
	fi_put2Players:

	mov rsp, rbp
	pop rbp
	ret
\end{lstlisting}
Aquesta subrutina permet una jugada, formada per un torn per cada jugador. Es força que el primer torn el jugui el jugador $1$ i es controla que la subrutina finalitzi en el cas que el jugador guanyi en el seu torn (no podem permetre que el jugador $2$ jugui si ja s'ha acabat el joc), així com que el jugador $2$ jugui només un cop (sortim del bucle si aquest ja ha jugat un cop i/o si ha aconseguit guanyar, òbviament). No van sorgir problemes majors en la realització d'aquesta subrutina.
\clearpage
\subsubsection{Play}
\begin{lstlisting}[firstnumber=880]
Play:
	push rbp
	mov  rbp, rsp

	bucle_Play:
		call put2Players
		cmp BYTE [tecla], ' '
		jne fi_Play
		cmp DWORD [row4Complete], 1
		je fi_Play
		jmp bucle_Play
		
	fi_Play:

	mov rsp, rbp
	pop rbp
	ret
\end{lstlisting}
Aquesta subrutina implementa la jugabilitat. Primer crida a put2Players, després comprova l'ultima tecla pitjada, si aquesta no es un espai surt (ja que per força haura sigut un ESC) i comprova el valor de \lstinline|DWORD[row4Complete]|, en cas de ser 1 ha guanyat i per allò surt. Si no es torna el bucle. No hi van haver problemes a aquesta subrutina.
\clearpage
\subsection{Nivell avançat}
Aquest nivell implementa les més opcions del \lstinline|checkRow|.
\subsubsection{checkRow}
\begin{lstlisting}[firstnumber=524]
checkRow:
	push rbp
	mov rbp, rsp

	push rax
    push rbx
    push rcx
    push rdx

    mov rax, 0
    mov rcx, 0
    mov rbx, 0
    mov rdx, 0
    mov DWORD [inaRow], 0
    mov ecx, DWORD [pos]  
    mov bl, BYTE [mBoard + ecx]    
    cmp bl, '.'
    je fi_vertical
    
	bucle_horizontal_adelante:
		cmp bl, BYTE [mBoard + ecx]
        jne fi_bucle_horizontal_adelante
        inc ecx  
        inc DWORD [inaRow]  
        cmp DWORD [inaRow], 4
        je fi_row4  
        mov eax, ecx
        push rcx
        mov rdx, 0
        mov ecx, 7
        div ecx  
        pop rcx
        cmp edx, 0
        je fi_bucle_horizontal_adelante
        jmp bucle_horizontal_adelante

    fi_bucle_horizontal_adelante:
        mov ecx, DWORD [pos]  
        dec ecx  

    bucle_horizontal_atras:
        cmp ecx, 0 
        jl fi_bucle_horizontal_atras
        cmp bl, BYTE [mBoard + ecx]
        jne fi_bucle_horizontal_atras
        dec ecx  
        inc DWORD [inaRow]
        cmp DWORD [inaRow], 4
        je fi_row4  
        mov eax, ecx
        push rcx
        mov rdx, 0
        mov ecx, 7
        div ecx  
        pop rcx
        cmp edx, 6
        je fi_bucle_horizontal_atras
        jmp bucle_horizontal_atras

    fi_bucle_horizontal_atras:
        mov ecx, DWORD [pos]  
        mov DWORD [inaRow], 0
         
    bucle_diagonal_adelante:
		cmp ecx, 42  
        jge fi_bucle_diagonal_adelante
		cmp bl, BYTE [mBoard + ecx]
        jne fi_bucle_diagonal_adelante
        add ecx, 8  
        inc DWORD [inaRow]
        cmp DWORD [inaRow], 4
        je fi_row4  
        mov eax, ecx
        push rcx
        mov rdx, 0
        mov ecx, 7
        div ecx  
        pop rcx
        cmp edx, 0
        je fi_bucle_diagonal_adelante
        jmp bucle_diagonal_adelante

    fi_bucle_diagonal_adelante:
        mov ecx, DWORD [pos]
        sub ecx, 8 

    bucle_diagonal_atras:
        cmp ecx, 0 
        jl fi_bucle_diagonal_atras
        cmp bl, BYTE [mBoard + ecx]
        jne fi_bucle_diagonal_atras
        sub ecx, 8  
        inc DWORD [inaRow]  
        cmp DWORD [inaRow], 4
        je fi_row4 
        mov eax, ecx
        push rcx
        mov rdx, 0
        mov ecx, 7
        div ecx 
        pop rcx
        cmp edx, 6
        je fi_bucle_diagonal_atras
        jmp bucle_diagonal_atras

    fi_bucle_diagonal_atras:
        mov ecx, DWORD [pos]  
        mov DWORD [inaRow], 0
        
    bucle_antidiagonal_adelante:
		cmp ecx, 42  
        jge fi_bucle_antidiagonal_adelante
		cmp bl, BYTE [mBoard + ecx]
        jne fi_bucle_antidiagonal_adelante
        add ecx, 6  
        inc DWORD [inaRow]  
        cmp DWORD [inaRow], 4
        je fi_row4 
        mov eax, ecx
        push rcx
        mov rdx, 0
        mov ecx, 7
        div ecx 
        pop rcx
        cmp edx, 6
        je fi_bucle_antidiagonal_adelante
        jmp bucle_antidiagonal_adelante

    fi_bucle_antidiagonal_adelante:
        mov ecx, DWORD [pos]  
        sub ecx, 6 

    bucle_antidiagonal_atras:
        cmp ecx, 0 
        jl fi_bucle_antidiagonal_atras
        cmp bl, BYTE [mBoard + ecx]
        jne fi_bucle_antidiagonal_atras
        sub ecx, 6  
        inc DWORD [inaRow]  
        cmp DWORD [inaRow], 4
        je fi_row4 
        mov eax, ecx
        push rcx
        mov rdx, 0
        mov ecx, 7
        div ecx 
        pop rcx
        cmp edx, 0
        je fi_bucle_antidiagonal_atras
        jmp bucle_antidiagonal_atras

    fi_bucle_antidiagonal_atras:
        mov ecx, DWORD [pos]  
        mov DWORD [inaRow], 0

    bucle_vertical:
        cmp ecx, 42  
        jge fi_vertical
        cmp bl, BYTE [mBoard + ecx]
        jne fi_vertical
        add ecx, 7  
        inc DWORD [inaRow]  
        cmp DWORD [inaRow], 4
        jne bucle_vertical

    fi_row4:
    mov DWORD [row4Complete], 1  
    
    fi_vertical:

    pop rdx
    pop rcx
    pop rbx
    pop rax

	mov rsp, rbp
	pop rbp
	ret
\end{lstlisting}
Aquesta subrutina és una millora de la del nivell mig. Aquesta versió no només controla la vertical, horitzontal, diagonal i antidiagonal, sinó que ho fa independentment d'on s'hagi col·locat la última fitxa (al principi de la fila, al final o al mig). Van sorgir problemes diversos i tots eren relacionats amb la manera en què es comprovava si s'havia fet 4 en ratlla. Es va solucionar realitzant una documentació sobre la instrucció \lstinline|div| (per tal d'aturar el bucle en cas de passar-se de la fila de la matriu \lstinline|mBoard|) i reordenant el codi (així com afegint comprovacions a \lstinline|checkRow|).
\section{Altres}
Aquest grup ha tingut la sort de comptar amb un integrant del grup que ja tenia coneixements previs sobre programació, però en altre cas ens hauria sigut pràcticament impossible arribar al nivell avançat en els termes de temps establerts. Tot i que el seguiment proporcionat pels professors durant les pràctiques ens ha sigut de gran ajuda, la introducció a les pràctiques es va fer de forma massa abrupta per a aquells que no haguèssim fet res més d'ensamblador que els codis de classe. Finalment ens agradaria mencionar que tot i que ha sigut un repte, hem gaudit prou de les pràctiques i que amb una introducció més detallada o uns enunciats més esclaridors seria ideal.
\end{document}