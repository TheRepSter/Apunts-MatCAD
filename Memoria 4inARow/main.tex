\documentclass[a4paper,12pt]{article}
\usepackage{tabularx}
\usepackage{titling}
\usepackage{listings}
\usepackage{lstlangarm}

\lstset{
	belowcaptionskip=1\baselineskip,
    frame=single,
	language=[x86]Assembler,
    frameround=tttt,
    xleftmargin=\parindent,
    tabsize=4,
    numbers=left,
    numbersep=8pt,
    stepnumber=1,
    numberstyle=\tiny\color{Gray}, 
    columns = fullflexible,
	literate=%
         {á}{{\'a}}1
         {à}{{\`a}}1
         {í}{{\'i}}1
         {ì}{{\`i}}1
         {é}{{\'e}}1
         {è}{{\`e}}1
         {ú}{{\'u}}1
         {ù}{{\`u}}1
         {ó}{{\'o}}1
         {ò}{{\`o}}1
		 {ç}{{\c{c}}}1
		 {·}{{$\cdot$}}1
		 {ü}{{\"u}}1
		 {0}{{{\color{Number}0}}}1
		 {1}{{{\color{Number}1}}}1
		 {2}{{{\color{Number}2}}}1
		 {3}{{{\color{Number}3}}}1
		 {4}{{{\color{Number}4}}}1
		 {5}{{{\color{Number}5}}}1
		 {6}{{{\color{Number}6}}}1
		 {7}{{{\color{Number}7}}}1
		 {8}{{{\color{Number}8}}}1
		 {9}{{{\color{Number}9}}}1
}

\begin{document}
\pagenumbering{gobble}
\begin{titlepage}
\scshape

\centering
\vspace{1cm}
\raisebox{-\baselineskip}{\rule{\textwidth}{1px}}
\rule{\textwidth}{1px}
\vspace{0.01cm}

{\huge{{Apunts}}}\par \vspace{0.1cm}
La primera meitat del 1r curs

\rule{\textwidth}{2px}

\vspace{1cm}

\begin{tabularx}{\textwidth}{X r}
 & Autor:\\
 & \large{EDUARDO PÉREZ MOTATO} \\
\end{tabularx}

\vspace{1.3cm}


\vfill
Setembre - Desembre 2023

\end{titlepage}
\pagenumbering{arabic}
\section{Descripció}
Hola amigos
\section{Objectius}
Que tal, aquí yo
\section{Variables utilitzades}
\begin{itemize}
    \item Hola
    \item Que tal
\end{itemize}
\section{Subrutines}
\subsection{Nivell basic}
Uououo nivel basico
\subsubsection{showCursor}
\begin{lstlisting}[firstnumber=246]
showCursor:
	push rbp
	mov  rbp, rsp

	;Salvar registres a la pila
	push rax
	push rbx
	;Inicialització dels registres
	mov eax, 0
	mov ebx, 0

	;Codi de la Pràctica
	;rowScreen=rowScreenIni
	mov eax, DWORD [RowScreenIni]
	mov DWORD [rowScreen], eax
	;colScreen=colScreenIni+(colCursor*4)
	mov al, BYTE [colCursor]
	sub eax, 'A'  ;Passem de codi ASCII a nombre restant-li 'A' (=65)
	shl eax, 2
	add eax, DWORD [ColScreenIni]
	mov DWORD [colScreen], eax

	call gotoxy  ;Sent x=[rowScreen] i y=[colScreen]

	;Recuperar registres de la pila
	pop rbx
	pop rax

	mov rsp, rbp
	pop rbp
	ret
\end{lstlisting}
\subsubsection{showPlayer}
\begin{lstlisting}[firstnumber=294]
showPlayer:
	push rbp
	mov  rbp, rsp

	;Salvar registres a la pila
	push rax

	;Passar DWORD [player] a ASCII i mostrar-lo segons
	;[rowScreen] i [colScreen]
	mov eax, DWORD [player]
	add eax, 48  ;Passem el nombre a codi ASCII (1=49 i 2=50)
	mov DWORD [rowScreen], 23
	mov DWORD [colScreen], 20

	;Cridar gotoxy per posicionar cursor
	call gotoxy

	;Emmagatzemar valor de caràcter a dil
	mov dil, al  ;al conté el valor de player+65 (player és un enter)

	;Cridar printch per mostrar caràcter
	call printch

	;Recuperar registres de la pila
	pop rax

	mov rsp, rbp
	pop rbp
	ret
\end{lstlisting}
\subsubsection{showBoard}
\begin{lstlisting}[firstnumber=337]
showBoard:
	push rbp
	mov  rbp, rsp

	;Salvar registres a la pila
	push rsi
	push rdi
	push rax
	 
	;Inicialitzar registres
	mov eax, 0
	mov rdi, 0
	mov rsi, 0
	 
	;Fila i columna per començar des del principi
	
	mov DWORD [rowScreen], 8
	buclefilas:
		cmp DWORD [rowScreen], 20
		jge fi_showBoard
		add DWORD [rowScreen], 2

		mov DWORD [colScreen], 4
	buclecolumnas:
		cmp DWORD [colScreen], 32
		jge buclefilas
		add DWORD [colScreen], 4
		mov dil, BYTE [mBoard + eax]
		inc eax

		;Cridar gotoxy per posicionar cursor
		call gotoxy

		;Cridar printch per mostrar caràcter (segons el contingut de dil)
		call printch

		;Reiniciem el bucle
		jmp buclecolumnas
	
	fi_showBoard:

	;Cridar showPlayer per mostrar el jugador
	call showPlayer

	;Recuperar registres de la pila
	pop rax
	pop rdi
	pop rsi

	mov rsp, rbp
	pop rbp
	ret
\end{lstlisting}
\subsubsection{moveCursor}
\begin{lstlisting}[firstnumber=406]
moveCursor:  
	push rbp
	mov  rbp, rsp
	call showCursor  ;Mostrem el cursor on toca
	bucle_moveCursor:
		call getch
		cmp BYTE [tecla], 'j'
		je moveLeft
		cmp BYTE [tecla], 'k'
		je moveRight
		cmp BYTE [tecla], ' '
		je fi_moveCursor
		cmp BYTE[tecla], 27  ;En ASCII 27 és [ESC]
		je fi_moveCursor
		jmp bucle_moveCursor

	moveLeft:
		cmp BYTE [colCursor], 'A'
		je fi_moveCursor
		dec BYTE [colCursor]
		jmp fi_moveCursor

	moveRight:
		cmp BYTE [colCursor], 'G'
		je fi_moveCursor
		inc BYTE [colCursor]

	fi_moveCursor:
		call showCursor

	mov rsp, rbp
	pop rbp
	ret
\end{lstlisting}
\subsubsection{moveCursorContinuous}
\begin{lstlisting}[firstnumber=454]
moveCursorContinuous:
	push rbp
	mov  rbp, rsp

	bucle_moveCursorContinuous:
		call moveCursor
		cmp BYTE [tecla], ' '
		je fi_moveCursorContinuous
		cmp BYTE [tecla], 27  ;En ASCII 27 és [ESC]
		je fi_moveCursorContinuous
		jmp bucle_moveCursorContinuous

	fi_moveCursorContinuous:

	mov rsp, rbp
	pop rbp
	ret
\end{lstlisting}
\subsubsection{calcIndex}
\begin{lstlisting}[firstnumber=485]
calcIndex:
	push rbp
	mov  rbp, rsp

	;Salvar registres a la pila
	push rax
	push rbx
	
	mov eax, DWORD [row]
	imul eax, 7
	mov bl, BYTE [col]
	sub bl, 'A'
	add eax, ebx
	mov DWORD [pos], eax

	;Recuperar registres de la pila
	pop rbx
	pop rax
	 
	mov rsp, rbp
	pop rbp
	ret
\end{lstlisting}
\subsubsection{putPiece}
\begin{lstlisting}[firstnumber=533]
	putPiece:  
	push rbp
	mov  rbp, rsp

	;Salvar registres a la pila
	push rax
	push rbx
	
	;Inicialitzar registres
	mov eax, 0
	mov ebx, 0
	
	call showBoard
	call moveCursorContinuous
	
	cmp BYTE [tecla], ' '
	jne fi_putPiece
	
	mov al, BYTE [colCursor]
	mov BYTE [col], al
	mov DWORD [row], 5
	
	bucle_putPiece:
		call calcIndex
		mov eax, DWORD [pos]
		mov bl, BYTE [mBoard + eax]  ;Accedim a l'índex de la matriu
		cmp bl, '.'  ;Si no hi ha res, podem col·locar la peça
		je colocacion
		cmp DWORD [row], 0  ;Si ja hi ha una fitxa, hem de mirar que no ens haguem passat dels límits de mBoard
		jl fi_putPiece
		dec DWORD [row]
		jmp bucle_putPiece
	
	colocacion:
		mov BYTE [mBoard + eax], 'X'  ;Col·loquem la peça
		call showBoard  ;Mostrem l'estat actual de la taula
		
	fi_putPiece:
		
	;Recuperar registres de la pila
	pop rbx
	pop rax

	mov rsp, rbp
	pop rbp
	ret
\end{lstlisting}
\subsection{Nivell mig}
Nivel medio que guay
\subsubsection{checkRow}
\begin{lstlisting}[firstnumber=524]
checkRow:

	push rbp
	mov  rbp, rsp

	push rax
	push rbx
	
	mov rax, 0
	mov rbx, 0
	mov DWORD [inaRow], 0
	
	mov eax, DWORD [pos]  ;eax conté l'índex de l'última fitxa col·locada

	mov bl, BYTE [mBoard + eax]  ;eax conté l'offset de mBooard
								 ;bl conté el char ('X' ó 'O'). bl mai serà '.'
	
	bucle_vertical:
		cmp eax, 42  ;42 = últim índex de mBoard
		jge fi_vertical
		;Si no ens hem sortit de la matriu...
		cmp bl, BYTE [mBoard + eax]
		jne fi_vertical
		;Si no ens hem sortit de la matriu
		;i els chars correlatius coincideixen...
		add eax, 7  ;Anem a la següent fila
		inc DWORD [inaRow]  ;Incrementem el comptador
		cmp DWORD [inaRow], 4
		jne bucle_vertical
		;tas feli, has ganao
		mov DWORD [row4Complete], 1  ;S'activa l'indicador
		
	fi_vertical:

	pop rbx
	pop rax

	mov rsp, rbp
	pop rbp
	ret
\end{lstlisting}
\subsubsection{putPiece}
\begin{lstlisting}[firstnumber=770]
putPiece:  
	push rbp
	mov  rbp, rsp

	;Salvar registres a la pila
	push rax
	push rbx
	
	;Inicialitzar registres
	mov eax, 0
	mov ebx, 0
	
	call showBoard
	call moveCursorContinuous
	
	cmp BYTE [tecla], ' '
	jne fi_putPiece
	
	mov al, BYTE [colCursor]
	mov BYTE [col], al
	mov DWORD [row], 5
	
	bucle_putPiece:
		call calcIndex
		mov eax, DWORD [pos]
		mov bl, BYTE [mBoard + eax]  ;Accedim a l'índex de la matriu
		cmp bl, '.'  ;Si no hi ha res, podem col·locar la peça
		je colocacion
		cmp DWORD [row], 0  ;Si ja hi ha una fitxa, hem de mirar
							;que no ens haguem passat dels límits de mBoard
		jl fi_putPiece
		dec DWORD [row]
		jmp bucle_putPiece
	
	colocacion:
		cmp DWORD [player], 1
		jne jugador2
		mov BYTE [mBoard + eax], '0'  ;Col·loquem la peça del jugador 1
		jmp fi_putPiece
		jugador2:
			mov BYTE [mBoard + eax], 'X'  ;Col·loquem la peça del jugador 2
		
	fi_putPiece:
		call showBoard  ;Mostrem l'estat actual de la taula
		call checkRow  ;Comprovem si s'han encadenat 4 fitxes iguals
		
	;Recuperar registres de la pila
	pop rbx
	pop rax

	mov rsp, rbp
	pop rbp
	ret
\end{lstlisting}
\subsubsection{put2Players}
\begin{lstlisting}[firstnumber=841]
put2Players:
	push rbp
	mov  rbp, rsp

	mov DWORD [player], 1  ;Forcem que primer jugui el jugador 1
	
	bucle_put2Players:
		call showPlayer  ;Mostrem el jugador
		call putPiece  ;Permetem la jugada
		cmp BYTE [tecla], ' '
		jne fi_put2Players  ;Si no és l'espai, per força serà [ESC]
		
		;Només hem de deixar jugar a player2 si player1 no ha guanyat
		cmp DWORD [row4Complete], 1
		je fi_put2Players
		;Evitem que player2 jugui infinitament
		cmp DWORD [player], 2
		je fi_put2Players
		;Si player1 no ha guanyat, passem el torn a player2
		mov DWORD [player], 2
		jmp bucle_put2Players
		
	fi_put2Players:

	mov rsp, rbp
	pop rbp
	ret
\end{lstlisting}
\subsubsection{Play}
\begin{lstlisting}[firstnumber=880]
Play:
	push rbp
	mov  rbp, rsp

	bucle_Play:
		call put2Players
		cmp BYTE [tecla], ' '
		jne fi_Play
		cmp DWORD [row4Complete], 1
		je fi_Play
		jmp bucle_Play
		
	fi_Play:

	mov rsp, rbp
	pop rbp
	ret

\end{lstlisting}
\subsection{Nivell avançat}
Nivel avanzado que avanzado
\subsubsection{checkRow}
\begin{lstlisting}[firstnumber=524]
checkRow:

	push rbp
	mov rbp, rsp
	
	;Salvar registres a la pila
	push rax
    push rbx
    push rcx
    push rdx

    ;Inicialitzar registres
    mov rax, 0
    mov rcx, 0
    mov rbx, 0
    mov rdx, 0
    mov DWORD [inaRow], 0
    mov ecx, DWORD [pos]  ;eax conté l'índex de l'última fitxa col·locada
    mov bl, BYTE [mBoard + ecx]  ;eax conté l'offset de mBooard
                                  ;bl conté el char ('X' ó 'O'). bl mai
                                  ;serà '.' menys en un cas especific
    cmp bl, '.'
    je fi_vertical
    
	bucle_horizontal_adelante:
		;Si no ens hem sortit de la matriu...
		cmp bl, BYTE [mBoard + ecx]
        jne fi_bucle_horizontal_adelante
        ;i els chars correlatius coincideixen...
        inc ecx  ;Anem a la següent columna
        inc DWORD [inaRow]  ;Incrementem el comptador
        ;Això és per saber si ens passem de columna
        cmp DWORD [inaRow], 4
        je fi_row4  ;En cas que sigui 4, saltem a fi_row4
        mov eax, ecx
        push rcx
        mov rdx, 0
        mov ecx, 7
        div ecx  ;Aquesta operació divideix el que hi ha a eax
				 ;(en aquest cas, la posició) per el nombre i
				 ;emmagatzema a edx el reminder
        pop rcx
        cmp edx, 0
        je fi_bucle_horizontal_adelante
        ;En cas contrari, podem continuar
        jmp bucle_horizontal_adelante

    fi_bucle_horizontal_adelante:
        mov ecx, DWORD [pos]  ;eax conté l'índex de l'última fitxa
							  ;col·locada
        dec ecx  ;Si comprovem de primeres el mateix, seríen 3 i no 4

    bucle_horizontal_atras:
        cmp ecx, 0 ;0 = primer índex de mBoard
        jl fi_bucle_horizontal_atras
        ;Si no ens hem sortit de la matriu...
        cmp bl, BYTE [mBoard + ecx]
        jne fi_bucle_horizontal_atras
        ;i els chars correlatius coincideixen...
        dec ecx  ;Anem a la següent columna
        inc DWORD [inaRow]  ;Incrementem el comptador
        ;Això és per saber si ens passem de columna
        cmp DWORD [inaRow], 4
        je fi_row4  ;En cas que sigui 4, saltem a fi_row4
        mov eax, ecx
        push rcx
        mov rdx, 0
        mov ecx, 7
        div ecx  ;Aquesta operació divideix el que hi ha a eax
				 ;(en aquest cas, la posició) per el nombre i
				 ;emmagatzema a edx el reminder
        pop rcx
        cmp edx, 6
        je fi_bucle_horizontal_atras
        ;En cas contrari, podem continuar
        jmp bucle_horizontal_atras

    fi_bucle_horizontal_atras:
        mov ecx, DWORD [pos]  ;eax conté l'índex de l'última fitxa
							  ;col·locada
        mov DWORD [inaRow], 0
         
    bucle_diagonal_adelante:
		;Si no ens hem sortit de la matriu...
		cmp ecx, 42  ;42 = últim índex de mBoard
        jge fi_bucle_diagonal_adelante
		cmp bl, BYTE [mBoard + ecx]
        jne fi_bucle_diagonal_adelante
        ;i els chars correlatius coincideixen...
        add ecx, 8  ;Anem a la següent columna
        inc DWORD [inaRow]  ;Incrementem el comptador
        ;Això és per saber si ens passem de columna
        cmp DWORD [inaRow], 4
        je fi_row4  ;En cas que sigui 4, saltem a fi_row4
        mov eax, ecx
        push rcx
        mov rdx, 0
        mov ecx, 7
        div ecx  ;Aquesta operació divideix el que hi ha a eax
				 ;(en aquest cas, la posició) per el nombre i
				 ;emmagatzema a edx el reminder
        pop rcx
        cmp edx, 0
        je fi_bucle_diagonal_adelante
        ;En cas contrari, podem continuar
        jmp bucle_diagonal_adelante

    fi_bucle_diagonal_adelante:
        mov ecx, DWORD [pos]  ;eax conté l'índex de l'última fitxa
							  ;col·locada
        sub ecx, 8 ;Si comprovem de primeres el mateix, seríen 3 i no 4

    bucle_diagonal_atras:
        cmp ecx, 0 ;0 = primer índex de mBoard
        jl fi_bucle_diagonal_atras
        ;Si no ens hem sortit de la matriu...
        cmp bl, BYTE [mBoard + ecx]
        jne fi_bucle_diagonal_atras
        ;i els chars correlatius coincideixen...
        sub ecx, 8  ;Anem a la següent columna
        inc DWORD [inaRow]  ;Incrementem el comptador
        ; aixo es per saber si ens passem de columna
        cmp DWORD [inaRow], 4
        je fi_row4 ; en cas de que sigui 4, saltem a fi_row4
        mov eax, ecx
        push rcx
        mov rdx, 0
        mov ecx, 7
        div ecx ;aquesta operació divideix el que hi ha a eax (en aquest cas, la posició) per el nombre i emmagatzema a edx el reminder
        pop rcx
        cmp edx, 6
        je fi_bucle_diagonal_atras
        ; en cas contrari, podem continuar
        jmp bucle_diagonal_atras

    fi_bucle_diagonal_atras:
        mov ecx, DWORD [pos]  ;eax conté l'índex de l'última fitxa col·locada
        mov DWORD [inaRow], 0
        
    bucle_antidiagonal_adelante:
		cmp ecx, 42  ;42 = últim índex de mBoard
        jge fi_bucle_antidiagonal_adelante
		;Si no ens hem sortit de la matriu...
		cmp bl, BYTE [mBoard + ecx]
        jne fi_bucle_antidiagonal_adelante
        ;i els chars correlatius coincideixen...
        add ecx, 6  ;Anem a la següent columna
        inc DWORD [inaRow]  ;Incrementem el comptador
        ; aixo es per saber si ens passem de columna
        cmp DWORD [inaRow], 4
        je fi_row4 ; en cas de que sigui 4, saltem a fi_row4
        mov eax, ecx
        push rcx
        mov rdx, 0
        mov ecx, 7
        div ecx ;aquesta operació divideix el que hi ha a eax (en aquest cas, la posició) per el nombre i emmagatzema a edx el reminder
        pop rcx
        cmp edx, 6
        je fi_bucle_antidiagonal_adelante
        ; en cas contrari, podem continuar
        jmp bucle_antidiagonal_adelante

    fi_bucle_antidiagonal_adelante:
        mov ecx, DWORD [pos]  ;eax conté l'índex de l'última fitxa col·locada
        sub ecx, 6 ; Si comprovem de primeres el mateix, serien 3 i no 4

    bucle_antidiagonal_atras:
        cmp ecx, 0 ;0 = primer índex de mBoard
        jl fi_bucle_antidiagonal_atras
        ;Si no ens hem sortit de la matriu...
        cmp bl, BYTE [mBoard + ecx]
        jne fi_bucle_antidiagonal_atras
        ;i els chars correlatius coincideixen...
        sub ecx, 6  ;Anem a la següent columna
        inc DWORD [inaRow]  ;Incrementem el comptador
        ; aixo es per saber si ens passem de columna
        cmp DWORD [inaRow], 4
        je fi_row4 ; en cas de que sigui 4, saltem a fi_row4
        mov eax, ecx
        push rcx
        mov rdx, 0
        mov ecx, 7
        div ecx ;aquesta operació divideix el que hi ha a eax (en aquest cas, la posició) per el nombre i emmagatzema a edx el reminder
        pop rcx
        cmp edx, 0
        je fi_bucle_antidiagonal_atras
        ; en cas contrari, podem continuar
        jmp bucle_antidiagonal_atras

    fi_bucle_antidiagonal_atras:
        mov ecx, DWORD [pos]  ;eax conté l'índex de l'última fitxa col·locada
        mov DWORD [inaRow], 0

    bucle_vertical:
        cmp ecx, 42  ;42 = últim índex de mBoard
        jge fi_vertical
        ;Si no ens hem sortit de la matriu...
        cmp bl, BYTE [mBoard + ecx]
        jne fi_vertical
        ;i els chars correlatius coincideixen...
        add ecx, 7  ;Anem a la següent fila
        inc DWORD [inaRow]  ;Incrementem el comptador
        cmp DWORD [inaRow], 4
        jne bucle_vertical

    fi_row4:
    mov DWORD [row4Complete], 1  ;S'activa l'indicador
    
    fi_vertical:
    ;Recuperar registres de la pila
    pop rdx
    pop rcx
    pop rbx
    pop rax

	mov rsp, rbp
	pop rbp
	ret

\end{lstlisting}
\section{Altres}
\end{document}