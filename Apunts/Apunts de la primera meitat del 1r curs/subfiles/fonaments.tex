\documentclass[../main.tex]{subfiles}
\graphicspath{{\subfix{../images/}}}

\begin{document}
    Prob 1, tema 3:
    MOV ESI, 0\\
    BUCLE:\= CMP ESI, \underline{128}\\
    \underline{JGE} FI\\
    MOV EAX, \underline{A[ESI]}\\
    ADD EAX, \underline{B[ESI]}\\
    MOV C[ESI], \underline{EAX}\\
    ADD ESI, \underline{4}\\
    \underline{JMP} BUCLE\\
    FI:
    
    Prob 2, tema 3:\\
    INI: MOV ESI, 0\\
    MOV \underline{AL}, 1\\
    BUCLE: CMP ESI, \underline{80}\\
    \underline{JGE} FI2\\
    \underline{MOV} EBX, Vector[ESI]\\
    ADD ESI, \underline{4}\\
    CMP EBX, \underline{Vector[ESI]}\\
    \underline{JGE} FI\\
    \underline{JMP} BUCLE\\
    FI: MOV AL, 0\\
    FI2: HALT
    
    Prob 3, tema 3:\\
    INI: MOV ESI, 0\\
    MOV EDI, 19\\
    MOV [Palin], 1\\ 
    BUCLE: MOV AL, \underline{NOM[ESI]}\\
    CMP AL, \underline{NOM[EDI]}\\
    \underline{JE} SEG\\
    MOV \underline{[Palin]}, 0\\
    \underline{JMP} FI\\
    SEG: ADD ESI, \underline{1}\\
    SUB EDI, \underline{1}\\
    CMP ESI, EDI\\
    JL BUCLE\\
    FI:
    
    Prob 4, tema 3:\\
    INI: MOV [Sum], 0\\
    MOV ESI, 0\\
    BUCLE: MOV EAX, V[ESI]\\
    ADD Sum, EAX\\
    ADD ESI, 4\\
    CMP ESI, 16\\
    JL BUCLE\\
    FI: \\
    
    MOV EAX, [Row]\\
    MUL EAX, 5\\
    ADD EAX, [Column]\\
    SHL EAX, 2\\
    MOV [IndexMat], EAX\\
    
    Registres CPU. Son MOLT ràpids, pero hi ha pocs.\\
    Memoria cache: Memòria que té copiats trocets de la memoria principal, fent que sigui més ràpid.\\
    Memoria Principal: RAM. Es ràpida.\\
    Memoria secundaria: Fa servir altres mitjans de suport (la resta son amb semiconductors, aquesta no). Aquestos son no volàtils (la resta ho son, no es perden al apagar l'ordinador). Bastant lent.\\
    Imaginem que tenim temps de la memoria cache ($T_{mc} = 1$ns) i temps de la memoria principal ($T_{mp} = 15$ns), (sempre $T_{mc} < T_{mp}$). La memoria cache funciona en principis de localitat espacial i localitat temporal, fent que hi hagui major \textit{hit ratio} ($H$). Llavors, el temps mitjà ($T_m$) serà
    \begin{displaymath}
        T_m = T_{mc}H + T_{mp}(1-H)
    \end{displaymath}
    També podem parlar de \textit{miss ratio} (que es $1-H$).\\
    Correspondència directa (no tengo): Presentación to guapa.\\
    Correspondència assciocativa (no tengo tampoco): Presentación to guapa.
\end{document}