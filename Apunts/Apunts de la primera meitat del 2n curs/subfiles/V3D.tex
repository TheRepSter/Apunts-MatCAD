\documentclass[../main.tex]{subfiles}
\graphicspath{{\subfix{../images/}}}

\begin{document}
\subsection*{Horari}
\begin{itemize}
    \item Dimecres 9-11h.
    \item Divendres 9-11h.
\end{itemize}
\subsection{Geometria euclidiana 3D}
\subsubsection{L'espai euclidia estàndard 3-dimensional}
Treballem a l'espai 3-dimensional en el qual vivim i que identifiquem amb $\mathbb{R}^3$.
\begin{notacio}
    En l'espai euclidia tenim l'origen a $\begin{psmallmatrix}0\\0\\0\end{psmallmatrix}$ i un punt
    arbitrari $P = \begin{psmallmatrix}x\\y\\z\end{psmallmatrix} = \left(0,0,0\right) \in \mathbb{R}^3 = \mathbb{R} \times \mathbb{R} \times \mathbb{R}$. 
    Ambdues notacions (vertical i horitzontal) són vàlides malgrat representar diferents conceptes
    tecnicament.
\end{notacio}
\begin{definicio}
    La norma euclidiana d'un vector $V \in \mathbb{R}$ es defineix com $\left\lVert V\right\rVert = \sqrt{x^2+y^2+z^2} \in \mathbb{R}_+$
    que compleix $\forall V, W \in \mathbb{R}^3, \forall \lambda \in \mathbb{R}$
    \begin{itemize}
        \item $\left\lVert V+W\right\rVert \leq \left\lVert V\right\rVert + \left\lVert W\right\rVert$
        \item $\left\lVert \lambda V\right\rVert = \left\lvert \lambda\right\rvert\left\lVert V\right\rVert$
        \item $\left\lVert V\right\rVert\iff V = 0$
    \end{itemize}
    Aquesta norma medeix la distància euclidiana entre dos punts $P_1 = (x_1, y_1, z_1)\text{ i }P_2 = (x_2, y_2, z_2)$ 
    per la fórmula 
    $\mathcal{D}\left(P_1, P_2\right) = \left\lVert P_1 - P_2\right\rVert$ que compleix
    \begin{itemize}
        \item $\mathcal{D}\left(P_1, P_3\right) \leq \mathcal{D}\left(P_1, P_2\right) + \mathcal{D}\left(P_2, P_3\right)$
        \item $\mathcal{D}\left(P_1, P_2\right) = \mathcal{D}\left(P_2, P_1\right)$
        \item $\mathcal{D}\left(P_1, P_2\right) = 0 \iff P_2 = P_1$
    \end{itemize}
\end{definicio}
\begin{obs}
    La norma euclidiana d'un vector $V$ correspon exactament a la seva longitud.
\end{obs}
\begin{demostracio}
    Recordem el teorema de Pitàgores. Aleshores, veiem que $\left\lVert V\right\rVert$ és exactament
    aplicar el teorema de Pitàgores dos cops, tal que
    \begin{itemize}
        \item $\left\lVert V\right\rVert = d^2 + V_3^2$
        \item $d^2 = V_1^2+V_2^2$
    \end{itemize}
\end{demostracio}
Com la norma d'un vector $V \in \mathbb{R}^3$ correspon a la seva longitud, de forma equivalent la
distància entre dos punts a $\mathbb{R}^3$ correspon a la longitud del segment que els uneix.\\
Si tenim $P_1\text{ i }P_2$ punts que defineixen un segment, la longitud $\mathcal{D}\left(P_1, P_2\right) = \left\lVert P_2-P_1\right\rVert$\\
Per tant, la distància euclidiana entre dos punts és la que coneixem!
Aquesta norma (i distància) euclidiana prové d'una estructura que a més de les longítuds contè la
nociò d'ortogonalitat:
\begin{definicio}
    Anomenem producte escalar a $\mathbb{R}^3$ la funció\\
    $\left\langle\dots, \dots\right\rangle: \mathbb{R}^3 \times \mathbb{R}^3 \to \mathbb{R}$\\
    $\left(V, W\right) \mapsto \left\langle V, W\right\rangle = V_1W_1 + V_2W_2 + V_3W_3$
    que és
    \begin{itemize}
        \item Bilineal: $\left\langle V+\lambda \bar{V}, W\right\rangle = \left\langle V, W\right\rangle + \lambda\left\langle \bar{V}, W\right\rangle$
        \\ idem si $V \leftrightsquigarrow W$
        \item Simètrica: $\left\langle V, W\right\rangle = \left\langle W, V\right\rangle$
        \item Definit positiu: $\left\langle V, V\right\rangle > 0$ si $V \neq 0$
    \end{itemize}
\end{definicio}
Observem que $\forall V \in \mathbb{R}^3,\;\left\lVert V\right\rVert = \sqrt{\left\langle V, V\right\rangle}$:
la norma es pot definir en funció del producte escalar.\\
Reciprocament, veiem facilment la \underline{Identitat de Polarització}
\begin{displaymath}
    \left\langle V, W\right\rangle = \frac{1}{2}\left(\left\lVert V+W\right\rVert^2 - \left\lVert V\right\rVert^2 - \left\lVert W\right\rVert^2\right)\\
    \forall V, W \in \mathbb{R}^3
\end{displaymath}
\begin{exercici}
    Arribar a la Identitat de Polarització apartir d'allò.
\end{exercici}
El producte escalar permet definir la noció d'ortogonalitat. Per veure aixó, necessitem primer el
resultat següent:
\begin{teorema}[Desigualtat de Cauchy-Schwartz]
    \begin{displaymath}
        \forall V, W \in \mathbb{R}^3, \left\lvert \left\langle V, W\right\rangle \right\rvert \leq \left\lVert V\right\rVert \left\lVert W\right\rVert
    \end{displaymath}
    A més, la igualtat s'assoleix només si $\exists \lambda \in \mathbb{R}: V = \lambda W$
\end{teorema}
\begin{demostracio}
    Fixem $V, W \in \mathbb{R}^3$ qualsevol. Aleshores definim $\forall\lambda\in\mathbb{R}$\\$\mathcal{P}(\lambda):=\left\lVert V+\lambda W\right\rVert^2 \geq 0$\\
    Observem que $\mathcal{P}(\lambda)=\left\langle V+\lambda W, V+\lambda W\right\rangle = \left\lVert V\right\rVert^2 + 2\lambda\left\langle V,W\right\rangle + \lambda^2\left\lVert W\right\rVert^2 \Rightarrow \mathcal{P}$
    és un polinomi en $\lambda$ de grau $2$. Llavors $\Delta = 4\left(\left\langle V,W\right\rangle^2 - \left\lVert V\right\rVert^2\left\lVert W\right\rVert^2\right)$ ha de ser $\leq 0$, ja que $\mathcal{P} \geq 0$.\\
    Deduim que $\Delta \leq 0 \iff \left\langle V, W\right\rangle^2-\left\lVert V\right\rVert^2\left\lVert W\right\rVert^2 \leq 0 \iff \left\langle V, W\right\rangle^2\leq\left\lVert V\right\rVert^2\left\lVert W\right\rVert^2$\\
    Si $\Delta = 0$ aixó implica que $\exists \lambda_0 \in \mathbb{R} : \mathcal{P}\left(\lambda_0\right) = 0$, aleshores $\mathcal{P}\left(\lambda_0\right) = 0 \iff \left\lVert V+\lambda_0 W\right\rVert = 0 \iff V = - \lambda_0 W$
\end{demostracio}
Com a conseqüencia, obtenim que el número
\begin{displaymath}
    \frac{\left\langle V, W\right\rangle}{\left\lVert V\right\rVert\left\lVert W\right\rVert } \in \left[-1, 1\right]\;\;\forall V, W \in \mathbb{R}^3\setminus\left\{0\right\}
\end{displaymath}
\begin{displaymath}
    \left(\iff \left\lvert \left\langle V, W\right\rangle \right\rvert\leq\left\lVert V\right\rVert\left\lVert W\right\rVert\right)
\end{displaymath}
\begin{definicio}
    L'ùnic $\theta \in \left[0, \pi\right] : \cos{\theta} = \frac{\left\langle V, W\right\rangle}{\left\lVert V\right\rVert\left\lVert W\right\rVert }$
    s'anomena angle euclidia entre $V\text{ i }W$.
\end{definicio}
L'angle és efectivament l'angle que coneixem.\\
Si $V = (1,0,0)\text{ i }W = (\cos{\alpha}, \sin{\alpha}, 0)$ obtenim que $\cos{\theta} = \frac{\left\langle V, W\right\rangle}{\left\lVert V\right\rVert\left\lVert W\right\rVert } = \frac{\cos{\alpha}}{1\times1} = \cos{\alpha} \Rightarrow \theta = \alpha$
\begin{definicio}
    Diem que $V\text{ i }W$ són ortogonals si $\left\langle V, W\right\rangle = 0$ o, de forma
    equivalent, si l'angle entre $V\text{ i }W$ és $\frac{\pi}{2}$.\\
    \begin{notacio}
        Denotem dos vectors ortogonals entre si com $V\perp W$
    \end{notacio}
    Un conjunt de 3 vectors és base ortogonal si $\left\langle U, V\right\rangle = \left\langle V, W\right\rangle = \left\langle U, W\right\rangle = 0$,
    És base ortonormal, si és base ortogonal i a més $\left\lVert U\right\rVert = \left\lVert V\right\rVert = \left\lVert W\right\rVert = 1$.
\end{definicio}

\end{document}