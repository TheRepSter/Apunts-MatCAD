\documentclass[../main.tex]{subfiles}
\graphicspath{{\subfix{../images/}}}

\begin{document}
\subsection*{Horari}
\begin{itemize}
    \item Dimecres 9-11h.
    \item Divendres 9-11h.
\end{itemize}
\subsection{Geometria euclidiana 3D}
\subsubsection{L'espai euclidia estàndard 3-dimensional}
Treballem a l'espai 3-dimensional en el qual vivim i que identifiquem amb $\mathbb{R}^3$.
\begin{notacio}
    En l'espai euclidia tenim l'origen a $\begin{psmallmatrix}0\\0\\0\end{psmallmatrix}$ i un punt
    arbitrari $P = \begin{psmallmatrix}x\\y\\z\end{psmallmatrix} = \left(0,0,0\right) \in \mathbb{R}^3 = \mathbb{R} \times \mathbb{R} \times \mathbb{R}$. 
    Ambdues notacions (vertical i horitzontal) són vàlides malgrat representar diferents conceptes
    tecnicament.
\end{notacio}
\begin{definicio}
    La norma euclidiana d'un vector $V \in \mathbb{R}$ es defineix com $\left\lVert V\right\rVert = \sqrt{x^2+y^2+z^2} \in \mathbb{R}_+$
    que compleix $\forall V, W \in \mathbb{R}^3, \forall \lambda \in \mathbb{R}$
    \begin{itemize}
        \item $\left\lVert V+W\right\rVert \leq \left\lVert V\right\rVert + \left\lVert W\right\rVert$
        \item $\left\lVert \lambda V\right\rVert = \left\lvert \lambda\right\rvert\left\lVert V\right\rVert$
        \item $\left\lVert V\right\rVert\iff V = 0$
    \end{itemize}
    Aquesta norma medeix la distància euclidiana entre dos punts $P_1 = (x_1, y_1, z_1)\text{ i }P_2 = (x_2, y_2, z_2)$ 
    per la fórmula 
    $\mathcal{D}\left(P_1, P_2\right) = \left\lVert P_1 - P_2\right\rVert$ que compleix
    \begin{itemize}
        \item $\mathcal{D}\left(P_1, P_3\right) \leq \mathcal{D}\left(P_1, P_2\right) + \mathcal{D}\left(P_2, P_3\right)$
        \item $\mathcal{D}\left(P_1, P_2\right) = \mathcal{D}\left(P_2, P_1\right)$
        \item $\mathcal{D}\left(P_1, P_2\right) = 0 \iff P_2 = P_1$
    \end{itemize}
\end{definicio}
\begin{obs}
    La norma euclidiana d'un vector $V$ correspon exactament a la seva longitud.
\end{obs}
\begin{demostracio}
    Recordem el teorema de Pitàgores. Aleshores, veiem que $\left\lVert V\right\rVert$ és exactament
    aplicar el teorema de Pitàgores dos cops, tal que
    \begin{itemize}
        \item $\left\lVert V\right\rVert = d^2 + V_3^2$
        \item $d^2 = V_1^2+V_2^2$
    \end{itemize}
\end{demostracio}
Com la norma d'un vector $V \in \mathbb{R}^3$ correspon a la seva longitud, de forma equivalent la
distància entre dos punts a $\mathbb{R}^3$ correspon a la longitud del segment que els uneix.\\
Si tenim $P_1\text{ i }P_2$ punts que defineixen un segment, la longitud $\mathcal{D}\left(P_1, P_2\right) = \left\lVert P_2-P_1\right\rVert$\\
Per tant, la distància euclidiana entre dos punts és la que coneixem!
Aquesta norma (i distància) euclidiana prové d'una estructura que a més de les longítuds contè la
nociò d'ortogonalitat:
\begin{definicio}
    Anomenem producte escalar a $\mathbb{R}^3$ la funció\\
    $\left\langle\dots, \dots\right\rangle: \mathbb{R}^3 \times \mathbb{R}^3 \to \mathbb{R}$\\
    $\left(V, W\right) \mapsto \left\langle V, W\right\rangle = V_1W_1 + V_2W_2 + V_3W_3$
    que és
    \begin{itemize}
        \item Bilineal: $\left\langle V+\lambda \bar{V}, W\right\rangle = \left\langle V, W\right\rangle + \lambda\left\langle \bar{V}, W\right\rangle$
        \\ idem si $V \leftrightsquigarrow W$
        \item Simètrica: $\left\langle V, W\right\rangle = \left\langle W, V\right\rangle$
        \item Definit positiu: $\left\langle V, V\right\rangle > 0$ si $V \neq 0$
    \end{itemize}
\end{definicio}
Observem que $\forall V \in \mathbb{R}^3,\;\left\lVert V\right\rVert = \sqrt{\left\langle V, V\right\rangle}$:
la norma es pot definir en funció del producte escalar.\\
Reciprocament, veiem facilment la \underline{Identitat de Polarització}
\begin{displaymath}
    \left\langle V, W\right\rangle = \frac{1}{2}\left(\left\lVert V+W\right\rVert^2 - \left\lVert V\right\rVert^2 - \left\lVert W\right\rVert^2\right)\\
    \forall V, W \in \mathbb{R}^3
\end{displaymath}
\begin{exercici}
    Arribar a la Identitat de Polarització apartir d'allò.
\end{exercici}
El producte escalar permet definir la noció d'ortogonalitat. Per veure aixó, necessitem primer el
resultat següent:
\begin{teorema}[Desigualtat de Cauchy-Schwartz]
    \begin{displaymath}
        \forall V, W \in \mathbb{R}^3, \left\lvert \left\langle V, W\right\rangle \right\rvert \leq \left\lVert V\right\rVert \left\lVert W\right\rVert
    \end{displaymath}
    A més, la igualtat s'assoleix només si $\exists \lambda \in \mathbb{R}: V = \lambda W$
\end{teorema}
\begin{demostracio}
    Fixem $V, W \in \mathbb{R}^3$ qualsevol. Aleshores definim $\forall\lambda\in\mathbb{R}$\\$\mathcal{P}(\lambda):=\left\lVert V+\lambda W\right\rVert^2 \geq 0$\\
    Observem que $\mathcal{P}(\lambda)=\left\langle V+\lambda W, V+\lambda W\right\rangle = \left\lVert V\right\rVert^2 + 2\lambda\left\langle V,W\right\rangle + \lambda^2\left\lVert W\right\rVert^2 \Rightarrow \mathcal{P}$
    és un polinomi en $\lambda$ de grau $2$. Llavors $\Delta = 4\left(\left\langle V,W\right\rangle^2 - \left\lVert V\right\rVert^2\left\lVert W\right\rVert^2\right)$ ha de ser $\leq 0$, ja que $\mathcal{P} \geq 0$.\\
    Deduim que $\Delta \leq 0 \iff \left\langle V, W\right\rangle^2-\left\lVert V\right\rVert^2\left\lVert W\right\rVert^2 \leq 0 \iff \left\langle V, W\right\rangle^2\leq\left\lVert V\right\rVert^2\left\lVert W\right\rVert^2$\\
    Si $\Delta = 0$ aixó implica que $\exists \lambda_0 \in \mathbb{R} : \mathcal{P}\left(\lambda_0\right) = 0$, aleshores $\mathcal{P}\left(\lambda_0\right) = 0 \iff \left\lVert V+\lambda_0 W\right\rVert = 0 \iff V = - \lambda_0 W$
\end{demostracio}
Com a conseqüencia, obtenim que el número
\begin{displaymath}
    \frac{\left\langle V, W\right\rangle}{\left\lVert V\right\rVert\left\lVert W\right\rVert } \in \left[-1, 1\right]\;\;\forall V, W \in \mathbb{R}^3\setminus\left\{0\right\}
\end{displaymath}
\begin{displaymath}
    \left(\iff \left\lvert \left\langle V, W\right\rangle \right\rvert\leq\left\lVert V\right\rVert\left\lVert W\right\rVert\right)
\end{displaymath}
\begin{definicio}
    L'ùnic $\theta \in \left[0, \pi\right] : \cos{\theta} = \frac{\left\langle V, W\right\rangle}{\left\lVert V\right\rVert\left\lVert W\right\rVert }$
    s'anomena angle euclidia entre $V\text{ i }W$.
\end{definicio}
L'angle és efectivament l'angle que coneixem.\\
Si $V = (1,0,0)\text{ i }W = (\cos{\alpha}, \sin{\alpha}, 0)$ obtenim que $\cos{\theta} = \frac{\left\langle V, W\right\rangle}{\left\lVert V\right\rVert\left\lVert W\right\rVert } = \frac{\cos{\alpha}}{1\times1} = \cos{\alpha} \Rightarrow \theta = \alpha$
\begin{definicio}
    Diem que $V\text{ i }W$ són ortogonals si $\left\langle V, W\right\rangle = 0$ o, de forma
    equivalent, si l'angle entre $V\text{ i }W$ és $\frac{\pi}{2}$.\\
    \begin{notacio}
        Denotem dos vectors ortogonals entre si com $V\perp W$
    \end{notacio}
    Un conjunt de 3 vectors és base ortogonal si $\left\langle U, V\right\rangle = \left\langle V, W\right\rangle = \left\langle U, W\right\rangle = 0$,
    És base ortonormal, si és base ortogonal i a més $\left\lVert U\right\rVert = \left\lVert V\right\rVert = \left\lVert W\right\rVert = 1$.
\end{definicio}

$\mathbb{R}^3$ admet una estructura adicional que permet multiplicar dos vectors:
\begin{definicio}
    $\forall V, W \in \mathbb{R}^3$, definim el seu producte vectorial
    \begin{displaymath}
        V\wedge W = \begin{psmallmatrix}
        V_2W_3-V_3W_2\\
        V_3W_1-V_1W_3\\
        V_1W_2-V_2W_1
        \end{psmallmatrix} \in \mathbb{R}^3
    \end{displaymath}
    que compleix
    \begin{itemize}
        \item Bilinealitat: $(U + \lambda V) \wedge W = U\wedge W + \lambda V\wedge W$ idem a la
        dreta.
        \item Antisimetria: $V\wedge W = -W \wedge V$
    \end{itemize}
\end{definicio}
Veiem facilment que $\forall V, W \in \mathbb{R}^3$ $\left\langle V\wedge W, V\right\rangle = 0 = \left\langle V\wedge W, W\right\rangle$
Més enllà
\begin{proposition}
    $\forall U, V, W \in \mathbb{R}^3, \left\langle U, V\wedge W\right\rangle = \det{\left(U, V, W\right)}$
\end{proposition}
\subsubsection{Moviments rigids i grup ortogonal}
Observar un objecte que es desplaça és equivalent que desplaçar-se observant aquest objecte fix.
La visió 3D utilitza l'observació d'un mateix objecte des de 2 punts de vista $\neq$ (un per cada
ull). Però això equival estrictament a l'observació d0un mateix objecte desplaçant-se a l'espai.\\
Per això primer estudiarem aquestes transformacions de l'espai que preservan un objecte (s'anomenen
moviments rigids). Són transformacions que preserven les distàncies entre qualsevol parell de punts
de l'objecte.\\
Començem estudiant un conjunt particular de transformació a l'espai.
\begin{definicio}
    El grup ortogonal ès el conjunt d'aplicacions lineals que preserven el producte escalar:
    \begin{displaymath}
        O(3) := \left\{M \in \mathcal{M}_3\left(\mathbb{R}\right) | \left\langle MV, MW \right\rangle = \left\langle V, W\right\rangle \forall V, W \in \mathbb{R}^3 \right\} 
    \end{displaymath}
\end{definicio}
\begin{obs}
    Si $M \in O(3)$ i $V \in \mathbb{R}^3$, $\left\lVert MV\right\rVert = \left\lVert V\right\rVert$\\
    Si $M\in O(3)$ i $V\perp W \Rightarrow MV \perp MW$
\end{obs}
Com $\forall V, W \in \mathbb{R}^3$, $\left\langle MV, MW\right\rangle = V^tM^tMW$ per tant,
\begin{displaymath}
    M \in O(3) \Leftrightarrow M^tM = \mathbb{I}_3  
\end{displaymath}
Al final obtenim
\begin{displaymath}
    O(3) = \left\{M \in \mathcal{M}_3\left(\mathbb{R}\right) | M^tM = \mathbb{I}_3\right\}
\end{displaymath}
\begin{proposicio}
    Si $M \in O(3)$, llavors $\det\left(M\right) = \pm 1$
\end{proposicio}
\begin{definicio}
    Es defineix el grup especial ortogonal
    \begin{displaymath}
        SO(3) := \left\{M \in O(3) | \det{M} = 1\right\} 
    \end{displaymath}
\end{definicio}
Per contstrucció els elements de $SO(3)$ són aquestes transformacions lineals que preserven les
bases ortogonals positives. És a dir, són aquestes que preserven l'orientació i més concretament,
$\left(Me_1, Me_2, Me_3\right)$ compleix
\begin{enumerate}
    \item $\left(Me_1, Me_2, Me_3\right)$ és una base ortogonal
    \item $\det{\left(Me_1, Me_2, Me_3\right)} = \det{M} = 1$ i doncs $\left\langle Me_1\wedge Me_2, Me_3\right\rangle = 1 \Rightarrow Me_3 = Me_1\wedge Me_2$
\end{enumerate}
Un example de $M \in O(3)\setminus SO(3)$ és la matriu $M = \begin{psmallmatrix}
    1 & 0 & 0\\
    0 & 1 & 0\\
    0 & 0 & 1
\end{psmallmatrix}$\\
Aquestes transformacions de $O(3)\setminus SO(3)$ canvien l'orientació no corresponen al context de
la visió 3D com no podem canviar l'orientació d'un objecte desplaçant-ho a l'espai. Per aixó tindrem
especial ènfasi en el subgrup $SO(3)$!
\begin{obs}
    $O(3)$ i $SO(3)$ són grups (noció d'àlgebra) el que diu el següent:
    \begin{itemize}
        \item Si $M, N \in O(3)$ (o $SO(3)$), $MN \in O(3)$.
        \item $\mathbb{I}_3 \in O(3)$.
        \item Si $M \in O(3)$, aleshores $M$ és invertible i $M^{-1} \in O(3)$.
    \end{itemize}    
\end{obs}
Més generalment,
\begin{definicio}
    Un conjunt $G$ és un grup Si
    \begin{itemize}
        \item $\exists$ operació interna $\cdot : G\cdot G \to G$
        \item $\forall g_1, g_2, g_3 \in G, (g_1\cdot g_2) \cdot g_3 = g_1\cdot (g_2 \cdot g_3)$
        \item $\exists e \in G$ tal que $e\cdot g = g \cdot e = g$ (element neutre)
        \item $\forall g \in G, \exists g^{-1} \in G$ tq $ g\cdot g^{-1} = g^{-1}g = e$ (inversa)
    \end{itemize}
\end{definicio}
\begin{teorema}
    Sigui $f: \mathbb{R}^3 \to \mathbb{R}^3$ una aplicació que preserva les distáncies $\forall P, Q \in \mathbb{R}^3, \mathcal{D}\left(f(P), f(Q)\right) = \mathcal{D}(P, Q) \Leftrightarrow \left\lVert f(Q)-f(P)\right\rVert = \left\lVert Q-P\right\rVert$
    Aleshores $\exists P_o \in \mathbb{R}^3,\; M \in O(3)$ tal que $\forall P \in \mathbb{R}^3, f(P) = P_o + MP$
\end{teorema}
\begin{definicio}
    Si a més $f$ preserva l'orientació, obtenim que $M \in SO(3)$.
    Un tal $f$ s'anomena moviment rigid i correspon al fet de desplaçar un objecte a $\mathbb{R}^3$
    (o de forma equivalent, canviar de punt de vista).
\end{definicio}
\subsubsection{Grup de rotacions}
Ara l'objectiu és entendre millor l'estructura dels grups $O(3)$ i $SO(3)$, i observar que el
subgrup $SO(3)$ està compost de les rotacions.
Primer observem que si treballem a l'espai euclidia de dimensió $n$ (on $n \in \mathbb{N}$), és a
dir, treballem a $\mathbb{R}^n$ amb el producte escalar $\left\langle V, W\right\rangle = \sum\limits_{i = 1}^{n} V_i W_i\; \forall V, W \in \mathbb{R}^n$
podem definir $O(n) = \left\{M \in \mathcal{M}_n(\mathbb{R}) | M^tM = \mathbb{I}_n\right\}$ i $SO(n) = \left\{M \in O(n) | \det{M} = 1\right\}$\\
Per entendre la dimensió $3$, necessitem entendre primer les dimensions inferiors.
\begin{itemize}
    \item $n = 1$:\\
    $M = (a) \in O(1) \iff M^tM = \mathbb{I}_1 = 1 \iff (a^2) = 1$\\$\iff a = \pm 1 \iff M = \pm \mathbb{I}_1$\\
    Deduïm que $O(1) = \left\{\pm \mathbb{I}_1\right\}$ i $SO(1) = \left\{\mathbb{I}_1\right\}$
    \item $n = 2$
    \begin{proposicio}
        Sigui $M \in O(2)$. $\exists! \theta \in \left[0, 2\pi\right)$ tal que\\
        \begin{displaymath}
            M = \begin{pmatrix}
                \cos{\theta} & -\sin{\theta}\\
                \sin{\theta} & \cos{\theta}
            \end{pmatrix}\;\text{o}\; M = \begin{pmatrix}
                \cos{\theta} & \sin{\theta}\\
                \sin{\theta} & -\cos{\theta}
            \end{pmatrix}
        \end{displaymath}
        Això genera una rotació d'angle $\theta$ en el primer cas i en el segon una simètria d'un
        eix horitzontal compost amb una rotació d'angle $\theta$.
    \end{proposicio}
    \begin{obs}
        En particular, si $M \in SO(2),\; \exists! \theta \in \left[0, 2\pi\right)$ tal que
        \begin{displaymath}
            M = R_\theta := \begin{pmatrix}
                \cos{\theta} & -\sin{\theta}\\
                \sin{\theta} & \cos{\theta}
            \end{pmatrix}
        \end{displaymath}
    \end{obs}
    \item $n = 3$
    \begin{proposicio}
        Sigui $M \in O(3)$. Aleshores $\exists B \in SO(3), \exists \theta \in \left[0, 2\pi\right)$
        tal que \begin{displaymath}
            BMB^{-1} = \begin{pmatrix}
                \pm 1 & 0 & 0\\
                0 & \cos{\theta} & -\sin{\theta}\\
                0 & \sin{\theta} & \cos{\theta}
            \end{pmatrix}
        \end{displaymath}
    \end{proposicio}
    Geomètricament, això significa que si $M = \mathcal{M}_{\textit{Can}} (L),\; \exists \mathcal{B} = \left(U, V, W\right)\text{ i }\exists \theta \in \left[0, 2\pi\right)$
    tal que $B = P_{\textit{Can}\to\mathcal{B}}$ i $BMB^{-1} = Mat_\mathcal{B}(L) = \begin{psmallmatrix}
        \pm 1 & 0 & 0\\
        0 & \cos{\theta} & -\sin{\theta}\\
        0 & \sin{\theta} & \cos{\theta}
    \end{psmallmatrix}$
    Com $B \in SO(3)$, vol dir que la base $\mathcal{B}$ és base ortonormal positiva (és a dir, té
    la mateixa orientació que la base canonica $\Leftrightarrow \det{\left(u,v,w\right)} = 1$)
    \begin{itemize}
        \item Cas on $\pm 1 = 1$:\\
        $U = L(U)$\\
        $L(V) = \cos{\theta}V + \sin{\theta}W$\\
        $L(W) = -\sin{\theta}V + \cos{\theta}W$\\
        Llavors $L$ és una rotació d'eix $U$ i d'angle $\theta$.
        \item Cas on $\pm 1 = -1$:\\
        $-U = L(U)$\\
        $L$ és la composició d'una simètria $\perp$ al pla $U^\perp = Vect(V, W)$ i de la rotació
        d'angle $\theta$ i eix $U$
    \end{itemize}
    \begin{teorema}[de les rotacions d'Euler]
        Sigui $M\in SO(3)$. Llavors $\exists B \in SO(3)$ i $\theta \in \left[0, 2\pi\right)$ tal que
        \begin{displaymath}
            \begin{pmatrix}
                1 & 0 & 0\\
                0 & \cos{\theta} & -\sin{\theta}\\
                0 & \sin{\theta} & \cos{\theta}
            \end{pmatrix}
        \end{displaymath}
    \end{teorema}
    Per això, $SO(3)$ rep el nom de \underline{grup de les rotacions}.\\
    En particular, i \underline{no és gens evident}, si componem dues rotacions a l'espai, obtenim
    una tercera.\\
    \underline{Cas particular}: Si $U = U'$, $R_{\theta_1,U'}\circ R_{\theta_2,U} = R_{\theta,U}$.
    Aixó és evident, però $R_{\theta_1,U}\circ R_{\theta_2,V} = R_{\theta,W}$ i és molt dificil
    obtenir $\theta$ i $W$.
\end{itemize}
\end{document}