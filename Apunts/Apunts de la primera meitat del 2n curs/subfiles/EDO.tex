\documentclass[../main.tex]{subfiles}
\graphicspath{{\subfix{../images/}}}

\begin{document}
\subsection*{Horari}
\begin{itemize}
    \item Dimarts 9-11h.
    \item Divendres 11-13h.
\end{itemize}
\setcounter{subsection}{-1}
\subsection{Introducció}
Les equacions diferencials són una eina molt important de modelització.
\begin{definicio}
    Les equacions diferencials són equacions que relacionen una funció (incògnita) amb les seves
    derivades.
\end{definicio}
\begin{definicio}
    Si la funció és d'una variable $u: I \subset \mathbb{R} \rightarrow \mathbb{R}\;||\;t \mapsto u(t)$
    es diuen Equacions Diferencials Ordinàries.
\end{definicio}
\begin{definicio}
    Si la funció és de diverses variables $u : \Omega \subset \mathbb{R}^n \to \mathbb{R}$ Es diuen
    Equacions de Derivades Parcials.
\end{definicio}
\subsection{Equacions diferencials de 1r ordre}
\begin{definicio}
    Una equació diferencial ordinària de primer ordre per una funció $y(x)$ és una equació
    \begin{displaymath}
        F(x,y,y') = 0
    \end{displaymath}
\end{definicio}
\begin{definicio}
    La forma explícita d'una equació diferencial ordinària de 1r ordre és
    \begin{displaymath}
        \frac{dy}{dx} = y' = f\left(x, f\left(x\right)\right)
    \end{displaymath}
\end{definicio}
\begin{definicio}
    La forma explícita d'una equació diferencial ordinària de 1r ordre es diu autònoma si $f$ no
    depèn explícitament de $x$ o sigui, és de la forma $y' = f(y)$
\end{definicio}
\begin{definicio}
    Una solució de la forma explícita d'una equació diferencial ordinària de 1r ordre és una funció
    $y(x)$ diferenciable definida en un interval $I$ tal que per a tot $x \in I$ se satisfà
    l'equació diferencial ordinària de 1r ordre.
\end{definicio}
En general, les solucions d'una EDO de 1r ordre formen una família uniparamètrica de funcions d'un
paràmetre constant. Aquesta expressió s'anomena solució general de l'EDO de 1r ordre.
\begin{definicio}
    Una equació diferencial de primer ordre amb una condició inicial s'anomena problema de valor inicial
    \begin{displaymath}
        \begin{cases}
            y' = f(x, y)\\
            y(x_0) = y_0
        \end{cases}
    \end{displaymath}
    La solució d'un problema de valor inicial s'anomena solució particular de l'equació.
\end{definicio}
\begin{definicio}
    Un cas particular de solucions són els equilibris que són les solucions que no depenen de la
    variable independent.
\end{definicio}
Una solució d'equilibri de $y' = f(x, y)$ és una solució de la forma $y(x) = y^*$ (numero). Es
compleix que $y(x) = y^*$ és solució d'equilibri de $y' = f(x, y)$ si i només si $f(x, y^*) = 0$ per
a tot $x$ per al que $f(x, y)$ estigui ben definit.\\
Si l'equació és autònoma $\left(y' = f(y)\right)$ les solucions d'equilibri $y(x) = y$ estan donades pels zeros
de $f$ i estan definides $\forall x \in \mathbb{R}$.
\begin{teorema}[Picardo-Gindelof]
    Sigui $\mathcal{R}$ una regió rectangular del pla $xy$\\definida per $R = \left\{(x, y) | a \leq x \leq b, c \leq y \leq d\right\}$
    que conté el punt $(x_0, y_0)$.\\
    Suposem que $f$ i que $\frac{\partial f}{\partial y}$ sigui contínues a $\mathcal{R}$.\\
    Aleshores existeix una única solució $y(x)$ definida a un interval $I_0 = (x_0-h, x_0+h), h > 0$
    contingut a $\left[a, b\right]$ del PVI
    \begin{displaymath}
        \begin{cases}
            y' = f(x,y)\\
            y(x_0) = y_0
        \end{cases}
    \end{displaymath}
    A més a més si denotem la solució de l'anterior sistema per $y(x; x_0, y_0)$ es compleix que $y(x; x_0, y_0)$
    és una funció continua respecte $x_0, y_0$.
    \begin{obs}
        Per assegurar unicitat és suficient amb què $f$ sigui de Lipschitz respecte a la variable $y$
        \begin{obs}
            Que sigui de Lipschitz significa que $\exists L > 0: \left\lvert f(x,y) - f(x,z)\right\rvert < L \left\lvert y-z\right\rvert(c, d)\;\forall(y, z, c, d)$
        \end{obs}
    \end{obs}
\end{teorema}
\begin{teorema}[de Peano]
    Si $f$ és contínua, existeix solució del sistema.
\end{teorema}
\begin{teorema}
    Si $f$ i que $\frac{\partial f}{\partial y}$ són contínues a $\mathcal{R}$ aleshores dues corbes
    solució de $y' = f(x,y)$ diferents no es poden tallar a $\mathbb{R}$.
\end{teorema}
\subsubsection{Alguns mètodes analítics de resolució d'EDO de 1r ordre}
\begin{enumerate}
    \item \textbf{EDO separable o de variable separada.}\\
    Una edo de variables separades és de la forma
    \begin{displaymath}
        y' = g\left(x\right) h\left(y\right)
    \end{displaymath}
    Si $h\left(y\right) \not\equiv 0$ llavors podem fer $\frac{1}{h\left(y\right)} y'\left(x\right) = g\left(x\right)$\\
    Integrant respecte a $x$ tenim
    \begin{displaymath}
        \int \frac{1}{h\left(y(x)\right)} y'(x) dx = \int g(x) dx
    \end{displaymath}
    Denotem per $H$ una primitiva de $\frac{1}{h\left(y\right)}$ i per $G$ una primitiva de $g(x)$,
    llavors tenim
    \begin{displaymath}
        H(y(x)) = G(x) + C
    \end{displaymath}
    Llavors $y(x) = H^{-1}\left(G(x)+C\right)$.\\
    Si $h\left(y\right) \equiv 0$, llavors $y(x) = y^*$, és a dir, té una solució d'equilibri.
    \item \textbf{EDO lineal.}\\
    Una EDO de 1r ordre lineal és de la forma
    \begin{displaymath}
        y'(x) = a(x) y(x) + b(x)
    \end{displaymath}
    on $a(x)$ i $b(x)$ són funcions arbitràries.
    \begin{obs}
        Si $b(x) \equiv 0$ llavors és una equació de variable separada. S'anomena l'equació
        homogènia associada a l'equació lineal.
    \end{obs}
    \begin{proposicio}
        Sigui $y_1(x)$ i $y_2(x)$ dues solucions de l'equació lineal no homogènia. Aleshores $y_1(x)-y_2(x)$
        és solució de l'equació homogènia associada.
    \end{proposicio}
    \begin{corolari}
        La solució general de l'equació homogènia és igual que una solució particular de l'homogènia
        més la solució general de l'equació homogènia.
        \begin{displaymath}
            y(x) = y_h(x) + y_p(x)
        \end{displaymath}
    \end{corolari}
    Per trobar $y_p(x)$ farem servir el "mètode de variació de les constants".
    Buscarem una solució particular de la forma \begin{displaymath}
        y_p(x) = C(x)e^{-\int a(x) dx}
    \end{displaymath}
    Volem que es compleixi $y'_p + a(x)y_p = b_x$, això passa si
    \begin{displaymath}
        b(x) = C'(x)e^{-\int a(x) dx} \Rightarrow C(x) = \int b(x)e^{\int a(x)dx}dx
    \end{displaymath}
    \item \textbf{EDO homogènia.}\\
    Una equació homogènia (de primer grau) és de la forma
    \begin{displaymath}
        y' = f\left(\frac{y}{x}\right) \leftarrow \substack{\text{Canvi de variable per}\\\text{transformar-la en variables separades}}
    \end{displaymath}
    El canvi de variable serà $u(x) = \frac{y(x)}{x} \Leftrightarrow y(x) = xu(x)$\\
    $y'(x) = u(x) + xu'(x) = f(u(x)) \rightarrow u'(x) = \frac{f(u(x))-u(x)}{x}$
\end{enumerate}
\subsubsection{Mètodes qualitatius}
\begin{enumerate}
    \item \textbf{Camps de direccions}\\
    Tenim $y' = f(x,y)$. Sigui $y(x)$ la solució d'aquesta equació que passa per $(x_0, y_0)$. Sabem
    llavors que $y(x_0) = y_0$ i $y'(x_0) = f(x_0, y_0)$.\\
    A cada punt del pla $(x, y)$ li podem associar un valor $f(x,y)$ que representarem dibuixant el
    punt $(x,y)$ un petit segment que tingui pendent $f(x,y)$. Obtenim així el camp de direccions.\\
    També podem fer $f(x,y) = m$, que defineix un conjunt de corbes al pla $(x,y)$ al llarg de les
    quals tots els vectors pendents han de ser $m$, es diuen isoclines.
\end{enumerate}
\subsubsection{Equacions diferencials autònomes}
$y' = f(y)$ (no depèn de manera explícita de la variable independent). Aquestes equacions són de
variables separades. EQUILIBRIS $\rightarrow$ solucions constants, zeros de la funció $f$.
\begin{teorema}[Comportament asintotic d'equacions autònomes]
    Donada una equació diferencial autònoma $y' = f(y)$, on $f$ és contínua, aleshores
    \begin{itemize}
        \item Si $y(x)$ una solució de l'equació autònoma, aleshores per qualsevol constant $C \in \mathbb{R}$
        també és solució.
        \item Si $y(x)$ és una solució de l'equació autònoma que no és un equilibri, és dir no es
        constant, aleshores no canvia de monotonia.
        \item Una solució acotada de l'equació autònoma tendeix (quan $x$ tendeix a $\pm \infty$) a
        una solució d'equilibri.
        \item Si $f(a) = 0$, $f(b) = 0$ i $f(y) > 0$ per a $y \in (a, b)$ i $y(x_0) \in (a, b)$
        aleshores $\lim_{x \to -\infty} y(x) = a$ i $\lim_{x \to \infty} y(x) = b$. I viceversa.
    \end{itemize}
\end{teorema}
\end{document}
