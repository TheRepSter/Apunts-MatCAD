\documentclass[../main.tex]{subfiles}
\graphicspath{{\subfix{../images/}}}

\begin{document}
\setcounter{subsection}{-1}
\subsection{Introducció}
Les equacions diferencials són una eina molt important de modelització.
\begin{definicio}
    Les equacions diferencials són equacions que relacionen una funció (incògnita) amb les seves
    derivades.
\end{definicio}
\begin{definicio}
    Si la funció és d'una variable $u: I \subset \mathbb{R} \rightarrow \mathbb{R}\;||\;t \mapsto u(t)$
    es diuen Equacions Diferencials Ordinàries.
\end{definicio}
\begin{definicio}
    Si la funció és de diverses variables $u : \Omega \subset \mathbb{R}^n \to \mathbb{R}$ Es diuen
    Equacions de Derivades Parcials.
\end{definicio}
\subsection{Equacions diferencials de 1r ordre}
\begin{definicio}
    Una equació diferencial ordinària de primer ordre per una funció $y(x)$ és una equació
    \begin{displaymath}
        F(x,y,y') = 0
    \end{displaymath}
\end{definicio}
\begin{definicio}
    La forma explícita d'una equació diferencial ordinària de 1r ordre és
    \begin{equation}
        \frac{dy}{dx} = y'(x) = f\left(x, f\left(x\right)\right)
        \label{eq:edo1}
    \end{equation}
\end{definicio}
\begin{definicio}
    La equació \eqref{eq:edo1} es diu autònoma si $f$ no
    depèn explícitament de $x$ o sigui, és de la forma
    \begin{displaymath}
    	y'(y) = f(y)
    \end{displaymath}
\end{definicio}
\begin{definicio}
    Una solució de \eqref{eq:edo1} és una funció
    $y(x)$ diferenciable definida en un interval $I : \forall x \in I$ es satisfà \eqref{eq:edo1}.
\end{definicio}
En general, les solucions d'una EDO de 1r ordre formen una família uniparamètrica de funcions d'un
paràmetre constant. Aquesta expressió s'anomena solució general de l'EDO de 1r ordre.
\begin{definicio}
    Una equació diferencial de primer ordre amb una condició inicial s'anomena problema de valor inicial i es de la forma
    \begin{equation}
        \begin{cases}
            y' &= f(x, y)\\
            y(x_0) &= y_0
        \end{cases}
        \label{eq:edo1pvi}
    \end{equation}
    La solució d'un problema de valor inicial s'anomena solució particular de l'equació.
\end{definicio}
\begin{definicio}
	Una solució d'equilibri de $y' = f(x, y)$ és una solució de la forma $y(x) = y^*$ on $y^*$ és una constant.
	Ha de cumplir que $y' = f(x, y) \iff f(x, y^*) = 0 \;\forall x \in \text{dom}\;f(x, y)$ estigui ben definit.\\
\end{definicio}
Si l'equació és autònoma $\left(y' = f(y)\right)$ les solucions d'equilibri $y(x) = y$ estan donades pels zeros
de $f$ i estan definides $\forall x \in \mathbb{R}$.
\subsubsection{Existència i unicitat i continuïtat de les solucions}
\begin{teorema}[Picardo-Gindelof]
    Sigui $\mathcal{R}$ una regió rectangular del pla $xy$\\definida per $R = \left\{(x, y) | a \leq x \leq b, c \leq y \leq d\right\}$
    que conté el punt $(x_0, y_0)$.\\
    Suposem que $f$ i que $\frac{\partial f}{\partial y}$ són contínues a $\mathcal{R}$.\\
    Aleshores existeix una única solució $y(x)$ definida a un interval $I_0 = (x_0-h, x_0+h), h > 0$
    contingut a $\left[a, b\right]$ del problema de valor inicial \eqref{eq:edo1pvi}.

    A més a més si denotem la solució de l'anterior sistema per $y(x; x_0, y_0)$ es compleix que $y(x; x_0, y_0)$
    és una funció continua respecte $x_0, y_0$.
    \begin{obs}
        Per assegurar unicitat és suficient amb què $f$ sigui de Lipschitz respecte a la variable $y$
        \begin{obs}
            Que sigui de Lipschitz significa que $\exists L > 0: \left\lvert f(x,y) - f(x,z)\right\rvert < L \left\lvert y-z\right\rvert(c, d)\;\forall(y, z, c, d)$
        \end{obs}
    \end{obs}
\end{teorema}
Com a conseqüència del teorema anterior tenim
\begin{teorema}
	Si $f$ i $\frac{\partial f}{\partial y}$ són contínues a $\mathbb{R}$ aleshores dues corbes solució de $y' = f(x,y)$ diferents no es poden tallar a $\mathbb{R}$.
\end{teorema}
Un altre teorema útil és el seguent
\begin{teorema}[de Peano]
    Si $f$ és contínua, existeix solució del sistema.
\end{teorema}
\subsubsection{Alguns mètodes analítics de resolució d'EDO de 1r ordre}
\begin{enumerate}
    \item \textbf{EDO separable o de variable separada.}\\
    Una edo de variables separades és de la forma
    \begin{displaymath}
        y' = g\left(x\right) h\left(y\right)
    \end{displaymath}
    Si $h\left(y\right) \not\equiv 0$ llavors podem fer $\frac{1}{h\left(y\right)} y'\left(x\right) = g\left(x\right)$\\
    Integrant respecte a $x$ tenim
    \begin{displaymath}
        \int \frac{1}{h\left(y(x)\right)} y'(x) dx = \int g(x) dx
    \end{displaymath}
    Denotem per $H$ una primitiva de $\frac{1}{h\left(y\right)}$ i per $G$ una primitiva de $g(x)$,
    llavors tenim
    \begin{displaymath}
        H(y(x)) = G(x) + C
    \end{displaymath}
    Llavors $y(x) = H^{-1}\left(G(x)+C\right)$.\\
    Si $h\left(y\right) \equiv 0$, llavors $y(x) = y^*$, és a dir, té una solució d'equilibri.

    \item \textbf{EDO lineal.}\\

    Una EDO de 1r ordre lineal és de la forma
    \begin{equation}
        y'(x) = a(x) y(x) + b(x)
        \label{eq:edo1lineal}
    \end{equation}
    on $a(x)$ i $b(x)$ són funcions arbitràries.
    \begin{obs}
        Si $b(x) \equiv 0$ llavors és una equació de variable separada. S'anomena l'equació
        homogènia associada a l'equació lineal.
        \begin{equation}
        	y'(x) = a(x) y(x)
			\label{eq:edo1linhomo}
        \end{equation}
    \end{obs}
    \begin{proposicio}
        Sigui $y_1(x)$ i $y_2(x)$ dues solucions de l'equació lineal \eqref{eq:edo1lineal}. Aleshores $y_1(x)-y_2(x)$
        és solució de l'equació homogènia associada \eqref{eq:edo1linhomo}.
    \end{proposicio}
    \begin{corolari}
        La solució general de \eqref{eq:edo1lineal} és igual que una solució particular de \eqref{eq:edo1linhomo}
        \begin{displaymath}
            y(x) = y_\text{homogénia}(x) + y_\text{particular}(x)
        \end{displaymath}
    \end{corolari}
    Per trobar $y_p(x)$ farem servir el "mètode de variació de les constants".
    Buscarem una solució particular de la forma \begin{displaymath}
        y_p(x) = C(x)e^{-\int a(x) dx}
    \end{displaymath}
    Volem que es compleixi $y'_p + a(x)y_p = b_x$, això passa si
    \begin{displaymath}
        b(x) = C'(x)e^{-\int a(x) dx} \Rightarrow C(x) = \int b(x)e^{\int a(x)dx}dx
    \end{displaymath}
    \item \textbf{EDO homogènia.}\\
    Una equació homogènia (de primer grau) és de la forma
    \begin{equation}
        y' = f\left(\frac{y}{x}\right) \leftarrow \substack{\text{Canvi de variable per}\\\text{transformar-la en variables separades}}
        \label{eq:edo1homo}
    \end{equation}
    Es tracta d'un tipus d'equacions que fent un canvi de variable es transforma en una equació de variables separades.
    El canvi de variable serà $u(x) = \frac{y(x)}{x} \Leftrightarrow y(x) = xu(x)$\\
    $y'(x) = u(x) + xu'(x) = f(u(x)) \rightarrow u'(x) = \frac{f(u(x))-u(x)}{x}$
\end{enumerate}
\subsubsection{Mètodes qualitatius: Camps de direccions}
Tenim $y' = f(x,y)$. Sigui $y(x)$ la solució d'aquesta equació que passa per $(x_0, y_0)$. Sabem
llavors que $y(x_0) = y_0$ i $y'(x_0) = f(x_0, y_0)$.\\
A cada punt del pla $(x, y)$ li podem associar un valor $f(x,y)$ que representarem dibuixant el
punt $(x,y)$ un petit segment que tingui pendent $f(x,y)$. Obtenim així el camp de direccions.\\
També podem fer $f(x,y) = m$, que defineix un conjunt de corbes al pla $(x,y)$ al llarg de les
quals tots els vectors pendents han de ser $m$, es diuen isoclines.

\subsubsection{Equacions diferencials autònomes}
Son de la forma
\begin{displaymath}
	y' = f(y)
\end{displaymath}
Aquestes no depèn de manera explícita de la variable independent. Aquestes equacions són de
variables separades i els seus equilibris son zeros de la funció $f$.
\begin{teorema}[Comportament asintotic d'equacions diferencials autònomes]
    Donada una equació diferencial autònoma $y' = f(y)$, on $f$ és contínua, aleshores
    \begin{itemize}
        \item Si $y(x)$ una solució de l'equació autònoma, aleshores per qualsevol constant $C \in \mathbb{R}$
        també és solució $y_c(x):= y(x+c)$.
        \item Si $y(x)$ és una solució de l'equació autònoma que no és un equilibri, és dir no es
        constant, aleshores no canvia de monotonia.
        \item Una solució acotada de l'equació autònoma tendeix (quan $x \to \pm \infty$) a
        una solució d'equilibri.
        \item Si $f(a) = 0$, $f(b) = 0$ i $f(y) > 0$ per a $y \in (a, b)$ i $y(x_0) \in (a, b)$
        aleshores $\lim\limits_{x \to -\infty} y(x) = a$ i $\lim\limits_{x \to \infty} y(x) = b$.\\
        Si $f(a) = 0$, $f(b) = 0$ i $f(y) < 0$ per a $y \in (a, b)$ i $y(x_0) \in (a, b)$
        aleshores $\lim\limits_{x \to -\infty} y(x) = b$ i $\lim\limits_{x \to \infty} y(x) = a$.
    \end{itemize}
\end{teorema}\newpage
\subsection{Sistemes d'equacions diferencials ordinaries lineals i EDOs d'ordre superior}
Una EDO d'ordre $n$ és una equació de la forma $F(x, y, y', \ldots, y^{(n)}) = 0$.\\
La forma estandard és
\begin{equation}
	y^{(n)} = f(x, y, y', \ldots, y^{(n-1)})
	\label{eq:edolinstand}
\end{equation}
Una solució de l'equació es una funció real $y(x)$ definida a un interval $I$, $n$ vegades diferenciable i que $\forall x \in I$ és compleix \eqref{eq:edolinstand}.\\
L'equació \eqref{eq:edolinstand} és equivalent a un sistema d'equacions de primer ordre $\vec{z}' = f(x, \vec{z})$\\
Una equació diferencial lineal d'ordre $n$ és una EDO de la forma
\begin{displaymath}
	y^{(n)} + a_{n-1}(x)y^{(n-1)} + \ldots + a_1(x)y' + a_0(x)y = f(x)
\end{displaymath}
Quan $f(x)\equiv0$ l'equació s'anomena homogenia.\\
\subsubsection{Equacions lineals de segon ordre}
\begin{equation}
	y'' + a(x)y' + b(x)y = f(x)
	\label{eq:edolin2n}
\end{equation}
La homogènia associada és
\begin{equation}
	y'' + a(x)y' + b(x)y = 0\label{eq:edolin2nhomo}
\end{equation}
Com en el cas de les líneals de primer ordre, la solució general de \eqref{eq:edolin2n} és la suma de la solució de la homogènia \eqref{eq:edolin2nhomo} i una solució particular de la no homogènia \eqref{eq:edolin2n}.\\
Anem a veure com solucionar la homogènia
\begin{proposicio}
	Siguin $y_1(x)$ i $y_2(x)$ dues solucions de \eqref{eq:edolin2nhomo}, aleshores $y(x) = Ay_1(x) + By_2(x)$ també és solució \eqref{eq:edolin2n} per constants $A$ i $B$ qualssevol.
\end{proposicio}
\begin{teorema}
	Donat el problema de valor inicial
	\begin{displaymath}
		\begin{aligned}[t]
			y'' + a(x)y' + b(x)y = f(x)\\
			y(x_0) = y_0\\
			y'(x_0) = y_0'
		\end{aligned}
	\end{displaymath}
	Si les funcions $a, b, f$ són contínues en un interval $I$ que conté $x_0$, aleshores el problema té una única solució en $I$.
\end{teorema}
\begin{definicio}
	Si $y_1(x)$ i $y_2(x)$ són solucions de l'equació lineal homogènia de segon ordre \eqref{eq:edolin2nhomo} el determinant $W(y_1,y_2)(x) := \det\begin{psmallmatrix}y_1(x) & y_2(x)\\y_1'(x) & y_2'(x)\end{psmallmatrix}$ s'anomena Wronskià de $y_1$ i $y_2$.
\end{definicio}
\begin{teorema}
	Si $y_1$ i $y_2$ són dues solucions de l'equació lineal homogènia de segon ordre \eqref{eq:edolin2nhomo} i $W(y_1,y_2)(x_0) \neq 0$ per algun $x_0$ aleshores
	\begin{displaymath}
		y(x) = Ay_1(x) + By_2(x)
	\end{displaymath}
	és la solució general de l'equació.
\end{teorema}
\begin{teorema}[de Abel]
	Si $y_1$ i $y_2$ són dues solucions de l'equació lineal homogènia de segon ordre \eqref{eq:edolin2nhomo} aleshores el Wronskià $W(y_1,y_2)(x)$ és una funció exponencial i $\forall x$ serà sempre $0$ o no ho serà.
\end{teorema}
\subsubsection{Equacions lineals de segon ordre amb coeficients constants}
\begin{equation}
	y'' + by' + cy = 0
	\label{eq:edolin2homoconst}
\end{equation}
Tindrem un polinomi característic de l'equació homogènia \eqref{eq:edolin2homoconst} de la forma $P(\lambda) = \lambda^2 + b\lambda + c = 0$. Llavors tenim tres casos:
\begin{enumerate}
	\item Si $b^2 - 4c > 0$ llavors les arrels són reals i diferents $\lambda_1$ i $\lambda_2$ i la solució general serà
		\begin{displaymath}
			y(x) = Ae^{\lambda_1x} + Be^{\lambda_2x}
		\end{displaymath}
	\item Si $b^2 - 4c = 0$ llavors les arrels són reals i iguals $\lambda_1 = \lambda_2 = \lambda$ i la solució general serà
		\begin{displaymath}
			y(x) = Ae^{\lambda x} + Bxe^{\lambda x}
		\end{displaymath}
	\item Si $b^2 - 4c < 0$ llavors les arrels són complexes $\lambda_1 = \alpha + i\beta$ i $\lambda_2 = \alpha - i\beta$ i la solució general serà
		\begin{displaymath}
			y(x) = e^{\alpha x}(A\cos(\beta x) + B\sin(\beta x))
		\end{displaymath}
\end{enumerate}

També podem fer servir el metode de variació de parametres per trobar la solució general de l'equació no homogènia \eqref{eq:edolin2n}.\\
	Sigui $\{y_1(x), y_2(x)\}$ un conjunt fonamental de solucions de l'equació homogènia \eqref{eq:edolin2nhomo} $y_h(x) = C_1 y_1(x) + C_2 y_2(x)$. Buscarem $y_p(x)$ de la forma $y_p(x) = C_1(x)y_1(x) + C_2(x)y_2(x)$. $y_p'(x) = C_1'(x)y_1(x) + C_2'(x)y_2(x) + C_1(x)y_1'(x) + C_2(x)y_2'(x)$. Imposem $C_1'(x)y_1(x) + C_2'(x)y_2(x) = 0$. Aleshores $y_p''(x) = C_1'(x)y_1'(x) + C_2'(x)y_2'(x) + C_1(x)y_1''(x) + C_2(x)y_2''(x)$. Substituïm a l'equació no homogènia \eqref{eq:edolin2n} i trobem $C_1'(x)y_1'(x) + C_2'(x)y_2'(x) = q(x)$. Resolem aquest sistema d'equacions per trobar $C_1(x)$ i $C_2(x)$.
	\begin{displaymath}
		\begin{cases}
			C_1'(x)y_1(x) + C_2'(x)y_2(x) = 0\\
			C_1'(x)y_1'(x) + C_2'(x)y_2'(x) = q(x)
		\end{cases}
	\end{displaymath}

\subsubsection{Aplicacions}
Sistemes mecànics: tenim un sistema massa-molla. La massa es de $m$ kg i la molla fa $l$ cm de llargada i $s$ de llargada quan es posa la massa. Aplicant la segona llei de Newton tenim $\sum F = mx''$. Tenim la força de la gravetat $F_1 = mg$ i la força de la molla per la llei de Hooke $F_2 = -k(x+s) = -kx-mg$. En equilibri tenim $F_1 + F_2 = 0 \Rightarrow mg - k(x+s) = 0$. Aleshores tenim l'equació diferencial $mx'' = ks$.\\
També podem tindre la força d'amortiment o fricció, que es presuposa que no hi ha si no es diu el contrari, $F_3 = -bx'$. I tambè podem tindre forces externes $F_4 = f(t)$. Aleshores tenim l'equació diferencial $F_1 + F_2 + F_3 + F_4 = 0 \Rightarrow mx'' = -bx' - kx + f(t)$.\\
Si $b = 0$ es diu no amortit. Si $f(t) = 0$ es diu lliure. Si ambdos són $0$ es diu oscilador harmònic simple.\\
El polinomic característic de l'equació diferencial serà $P(\lambda) = m\lambda^2 + k = 0$. Llavors, $x(t) = C_1\sin(\sqrt{\frac{k}{m}}t) + C_2\cos(\sqrt{\frac{k}{m}}t) = A\sin(\sqrt{\frac{k}{m}}t + \phi)$, amb $\phi = \frac{c_1}{c_2},\;A=\sqrt{c_1^2+c_2^2}$.\\

Cas amortit i lliure $b>0,\;k>0,\;m>0$. El polinomic característic serà $P(\lambda) = m\lambda^2 + b\lambda + k = 0$. Llavors tenim tres casos:
\begin{itemize}
	\item $b^2 - 4mk > 0$ llavors les arrels són reals i diferents $\lambda_1$ i $\lambda_2$ i la solució general serà
		\begin{displaymath}
			x(t) = C_1e^{\lambda_1t} + C_2e^{\lambda_2t}
		\end{displaymath}
		Notem que $\lim\limits_{t\to+\infty} x(t) = 0$. Hi ha un maxim (o minim) a $t = \frac{1}{\lambda_2-\lambda_1}\ln{\left(\frac{-C_1\lambda_1}{C_2\lambda_2}\right)}$. Es diu que aquest moviment és sobreamortit.
	\item $b^2 - 4mk = 0$ llavors les arrels són reals i iguals $\lambda_1 = \lambda_2 = \lambda$ i la solució general serà
		\begin{displaymath}
			x(t) = e^{\lambda t}(C_1 + C_2t)
		\end{displaymath}
		Notem que $\lim\limits_{t\to+\infty} x(t) = 0$. Hi ha un maxim (o minim) a $t = -\frac{c_1}{c_2} + \frac{2m}{b}$. Es diu que aquest moviment és críticament amortit.
	\item $b^2 - 4mk < 0$ llavors les arrels són complexes $\lambda_1 = -\frac{b}{2m} + i\sqrt{\frac{4mk-b^2}{2m}} = \alpha + i\beta$ i $\lambda_2 = \alpha - i\beta$ i la solució general serà
		\begin{displaymath}
			x(t) = e^{\alpha t}(C_1\cos(\beta t) + C_2\sin(\beta t)) = Ae^{\alpha t}\sin(\beta t + \phi)\; A = \sqrt{c_1^2+c_2^2}\; \tan{\phi} = \frac{c_1}{c_2}
		\end{displaymath}
		Notem que $\lim\limits_{t\to+\infty} x(t) = 0$. Aquesta sí que oscila, llavors te molts punts critics. Es diu que aquest moviment és subamortit.
\end{itemize}

Cas amortit i forçat $b>0,\;k>0,\;m>0,\;f(t)\neq 0$. L'equació diferencial serà $mx'' + bx' + kx = f(t)$. La solució general serà $x(t) = x_h(t) + x_p(t)$, on $x_h(t)$ és la solució general de l'homogènia i $x_p(t)$ és una solució particular de l'equació diferencial.\\
Notem que ja hem estudiat la homogenia anteriorment, ja que en el cas lliure es homogenia. $x_h(t)$ s'anomena terme transitori i $x_p(t)$ terme estacionari. Això ja que $\lim\limits_{t \to +\infty} x(t) = \lim\limits_{t\to+\infty} x_p(t)$, es dir, al limit el $x_h$ no importa.\\

Cas no amortit i forçat ressonància $b=0,\;k>0,\;m>0,\;f(t) = F_0\sin{\omega_0t}$. L'equació diferencial serà $mx'' + kx = F_0\sin{\omega_0t}$. La solució general serà $x(t) = x_h(t) + x_p(t)$. $x_h(t) =  C_1\sin(\sqrt{\frac{k}{m}}t) + C_2\cos(\sqrt{\frac{k}{m}}t)$. Busquem $x_p(t)$. Anotem $\omega = \sqrt{\frac{k}{m}}$
	\begin{itemize}
		\item $\omega \neq \omega_0$ busquem
		$x_p(t)=A\sin{\left(\omega_0 t\right)} + B\cos{\left(\omega_0 t\right)}$, substituim $x_p(t) = \frac{F_0\sin{\omega_0 t}}{\omega^2-\omega_0^2} \Rightarrow x(t) = C_1\sin{\left(\sqrt{\frac{k}{m}}t\right)} + C_2\cos{\left(\sqrt{\frac{k}{m}}t\right)} + \frac{F_0\sin{\omega_0 t}}{\omega^2-\omega_0^2}$.
		\item $\omega = \omega_0$ busquem
		$x_p(t)=t(A\sin{\omega t} + B\cos{\omega t}) = - \frac{F_0}{2\omega}t\cos{\omega t} \Rightarrow x(t) = C_1\sin{\left(\sqrt{\frac{k}{m}}t\right)} + C_2\cos{\left(\sqrt{\frac{k}{m}}t\right)} - \frac{F_0}{2\omega}t\cos{\omega t}$.
	\end{itemize}
\end{document}
