\documentclass[../main.tex]{subfiles}
\graphicspath{{\subfix{../images/}}}

\begin{document}
\subsection*{Horari}
\begin{itemize}
    \item Dimarts 9-11h.
    \item Divendres 11-13h.
\end{itemize}
\setcounter{subsection}{-1}
\subsection{Introducció}
Les equacions diferenciàls són una eina molt important de modelització.
\begin{definicio}
    Les equacions diferencials son equacions que relacionen una funció (incognita) amb les seves
    derivades.
\end{definicio}
\begin{definicio}
    Si la funció és d'una variable $u: I \subset \mathbb{R} \rightarrow \mathbb{R}\;||\;t \mapsto u(t)$
    es diuen Equacions Diferencials Ordinàries.
\end{definicio}
\begin{definicio}
    Si la funció és de diverses variables $u : \Omega \subset \mathbb{R}^n \to \mathbb{R}$ Es diuen
    Equacions de Derviades Parcials.
\end{definicio}
\subsection{Equacions diferencials de 1r ordre}
\begin{definicio}
    Una equació diferencial ordinária de primer ordre per una funció $y(x)$ és una equació 
    \begin{displaymath}
        F(x,y,y') = 0
    \end{displaymath}
\end{definicio}
\begin{definicio}
    La forma explícita d'una equació diferencial ordinària de 1r ordre es
    \begin{displaymath}
        \frac{dy}{dx} = y' = f\left(x, f\left(x\right)\right)
    \end{displaymath}
\end{definicio}
\begin{definicio}
    La forma explícita d'una equació diferencial ordinària de 1r ordre es diu autònoma si $f$ no
    depèn explicitament de $x$ o sigui, és de la forma $y' = f(y)$
\end{definicio}
\begin{definicio}
    Una solució de la forma explícita d'una equació diferencial ordinària de 1r ordre es una funció
    $y(x)$ diferenciable definida en un interval $I$ tal que per a tot $x \in I$ se satisfà
    l'equació diferencial ordinària de 1r ordre.
\end{definicio}
En general, les solucions d'una EDO de 1r ordre formen una família uniparamètrica de funcions d'un
paràmetre constant. Aquesta experssío s'anomena solució general de la EDO de 1r ordre.
\begin{definicio}
    Una equació diferencial de primer ordre amb una condició inicial s'anomena problema de valor inicial
    \begin{displaymath}
        \begin{cases}
            y' = f(x, y)\\
            y(x_0) = y_0
        \end{cases}
    \end{displaymath}
    La solució d'un problema de valor inicial s'anomena solució particular de l'equació.
\end{definicio}
\begin{definicio}
    Un cas particular de solucions són els equilibris que són les solucions que no depenen de la
    variable independent.
\end{definicio}
Una solució d'equilibri de $y' = f(x, y)$ és una solució de la forma $y(x) = y^*$ (numero). Es
compleix que $y(x) = y^*$ és solució d'equilibri de $y' = f(x, y)$ si i només si $f(x, y^*) = 0$ per
a tot $x$ per al que $f(x, y)$ estigui ben definit.\\
Si l'equació és autonòma $\left(y' = f(y)\right)$ les solucions d'equilibri $y(x) = y$ estàn donades pels zeros
de $f$ i estan definides $\forall x \in \mathbb{R}$.
\begin{teorema}[Picardo-Gindelof]
    Sigui $\mathcal{R}$ una regió rectangular del pla $xy$\\definida per $R = \left\{(x, y) | a \leq x \leq b, c \leq y \leq d\right\}$
    que conté el punt $(x_0, y_0)$.\\
    Suposem que $f$ i que $\frac{\partial f}{\partial y}$ sigui continues a $\mathcal{R}$.\\
    Aleshores existeix una única solució $y(x)$ definida a un interval $I_0 = (x_0-h, x_0+h), h > 0$ contingut a $\left[a, b\right]$ del PVI
    \begin{displaymath}
        \begin{cases}
            y' = f(x,y)\\
            y(x_0) = y_0
        \end{cases}
    \end{displaymath}
    A més a més si denotem la solució de l'anterior sistema per $y(x; x_0, y_0)$ es compleix que $y(x; x_0, y_0)$
    és una funció continua respecte $x_0, y_0$.
    \begin{obs}
        Per assegurar unicitat és suficient amb què $f$ sigui de Lipschitz respecte a la variable $y$
        \begin{obs}
            Que sigui de Lipschitz significa que $\exists L > 0: \left\lvert f(x,y) - f(x,z)\right\rvert < L \left\lvert y-z\right\rvert(c, d)\;\forall(y, z, c, d)$
        \end{obs}
    \end{obs}
\end{teorema}
\begin{teorema}[de Peano]
    Si $f$ és contínua, existeix solució del sistema.
\end{teorema}
\begin{teorema}
    Si $f$ i que $\frac{\partial f}{\partial y}$ són contínues a $\mathcal{R}$ aleshores dues corbes
    solució de $y' = f(x,y)$ diferents no es poden tallar a $\mathbb{R}$.
\end{teorema}
\end{document}