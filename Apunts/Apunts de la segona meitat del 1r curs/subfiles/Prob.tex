\documentclass[../main.tex]{subfiles}
\graphicspath{{\subfix{../images/}}}

\begin{document}
    \begin{definicio}
        Un fenomen o experiment aleatori presenten les següents característiques:
        \begin{itemize}
            \item Abans de realitzar l'experiment no sabem el resultat però sí el conjunt de resultats
            possibles.
            \item En teoria es pot realitzar sota les mateixes condicions infinites vegades.
            \item Es pot assignar probabilitats als resultats.
        \end{itemize}
    \end{definicio}
    \begin{definicio}
        L'espai mostral és el conjunt de possibles resultats de l'experiment aleatori. Es denota per
        la lletra $\Omega$ i els seus elements per $\omega$.
    \end{definicio}
    \begin{definicio}
        Un esdeveniment és una col·lecció de subconjunts de l'espai mostral. Es pot calcular la
        probabilitat d'un esdeveniment. Ha de tenir estructura de $\sigma$-àlgebra.
    \end{definicio}
    \begin{notacio}
        Si $\omega \in \Omega$ és un resultat de l'experimental tal que $\omega \in A$, diem que $A$ s'ha realitzat.
    \end{notacio}
    \begin{definicio}
        Sigui $\mathcal{A}$ una col·lecció de subconjunts d'$\Omega$. $\mathcal{A}$ és una
        $\sigma$-àlgebra si es compleix el següent:
        \begin{enumerate}
            \item $\Omega \in \mathcal{A}$
            \item Si $A \in \mathcal{A}$, aleshores $A^c \in \mathcal{A}$.
            \item Si $A_1, A_2, \dots$ són elements d'$\mathcal{A}$, aleshores $\bigcup\limits_{n=1}^\infty A_n \in \mathcal{A}$
        \end{enumerate}
    \end{definicio}
    \begin{corolari}[propietats d'una $\sigma$-àlgebra]
        Sent $\mathcal{A}$ una $\sigma$-àlgebra
        \begin{itemize}
            \item $\emptyset \in \mathcal{A}$
            \item $A, B \in \mathcal{A} \Rightarrow A \cap B \in \mathcal{A}$
            \item $A, B \in \mathcal{A} \Rightarrow B \setminus A = B\cap A^c \in \mathcal{A}$
        \end{itemize}
    \end{corolari}
    \begin{definicio}[Fórmula de Laplace]
        La probabilitat d'un esdeveniment $\mathcal{A}$ sempre que el conjunt de resultats possibles
        sigui finit i equiprobable, la fórmula de Laplace es pot aplicar.
        \begin{displaymath}
            \mathbb{P}(A) = \frac{\text{Casos probables a } A }{\text{Casos possibles}}
        \end{displaymath}
    \end{definicio}
    Una altra manera de calcular la probabilitat és fent servir una visió freqüentista:
    \begin{displaymath}
        \mathbb{P}(A) = \lim_{n\rightarrow\infty}f_n(A) \hbox{ on } f_n(A):= \frac{\text{nombre de cops que hem obtingut } A}{n}
    \end{displaymath}
    \begin{definicio}[axiomes de Kolmogorov]
        Siguin $\Omega$ un conjunt i $\mathcal{A}$ una $\sigma$-àlgebra sobre $\Omega$. Una
        probabilitat és qualsevol aplicació $\mathbb{P} : \mathcal{A} \longrightarrow [0,1]$ que
        compleix el següent:
        \begin{itemize}
            \item $\mathbb{P}(\Omega) = 1$
            \item Si $\{A_n, n\geq 1\} \subset \mathcal{A}$ són disjunts dos a dos llavors
            \begin{displaymath}
                \mathbb{P} (\bigcup\limits_{n=1}^\infty) = \sum\limits_{n=1}^{\infty}\mathbb{P}(A_n)
            \end{displaymath}
        \end{itemize}
    \end{definicio}
    \begin{definicio}
        Un espai de probabilitat és la terna $(\Omega, \mathcal{A}, \mathbb{P})$.
    \end{definicio}
    \begin{notacio}
        Per a unions disjuntes fem servir $\uplus$.
    \end{notacio}
    \begin{corolari} Propietats dels axiomes de Kolmogorov.
        \begin{enumerate}
            \item $\mathbb{P}(\emptyset) = 0$
            \item $A, B \in \mathcal{A} \Rightarrow \mathbb{P}(A\cup B) = \mathbb{P}(A) + \mathbb{P}(B) - \mathbb{P}(A \cap B)$
            \item $A \subset B \Rightarrow \mathbb{P}(A) \leq \mathbb{P}(B)$
            \item $A \subset B \Rightarrow \mathbb{P}(B\setminus A) = \mathbb{P}(B) - \mathbb{P}(A)$
            \item $\mathbb{P}(A\cup B) \leq \mathbb{P}(A) + \mathbb{P}(B)$
        \end{enumerate}
    \end{corolari}
    \begin{definicio}
        Quan parlem de \textit{odds} de $A$, definim:
        \begin{itemize}
            \item Odds a favor de $A$: $\text{Odds}(A) = \frac{\mathbb{P}(A)}{\mathbb{P}(A^c)}$
            \item Odds en contra de $A$: $\text{Odds}(A^c) = \frac{\mathbb{P}(A^c)}{\mathbb{P}(A)}$
        \end{itemize}
    \end{definicio}
    \begin{exemple}
        $\text{Odds}(A) = \frac{3}{2} \Longleftrightarrow \mathbb{P}(A) = \frac{3}{2}\mathbb{P}(A^c)$ i
        sabem que $\mathbb{P}(A^c) = 1 - \mathbb{P}(A)$, llavors en resoldre tenim $\mathbb{P}(A) = 0.6$
        i $\mathbb{P}(A^c) = 0.4$.
    \end{exemple}
    \begin{notacio}
        $\hbox{}$
        \begin{itemize}
            \item Permutacions de $n$ elements: $\mathbf{P}_n = n!$
            \item Variacions de $n$ elements sense reposició on triem $m\leq n$: $\mathbf{V}_n^m = \frac{n!}{(n-m)!}$
            \item Variacions de $n$ elements amb reposició on triem $m$: $\mathbf{VR}_n^m = n^m$
            \item Combinacions de $n$ elements: $\mathbf{C}_n^m = (\substack{n\\m}) = \frac{n!}{(n-m)!m!} = \frac{\mathbf{V}_n^m}{\mathbf{P}_m}$
        \end{itemize}
    \end{notacio}
\end{document}