\documentclass[../main.tex]{subfiles}
\graphicspath{{\subfix{../images/}}}

\begin{document}
    \begin{definicio}
        Un fenomen o experiment aleatori presenten les següents característiques:
        \begin{itemize}
            \item Abans de realitzar l'experiment no sabem el resultat però sí el conjunt de
            resultats possibles.
            \item En teoria es pot realitzar sota les mateixes condicions infinites vegades.
            \item Es pot assignar probabilitats als resultats.
        \end{itemize}
    \end{definicio}
    \begin{definicio}
        L'espai mostral és el conjunt de possibles resultats de l'experiment aleatori. Es denota per
        la lletra $\Omega$ i els seus elements per $\omega$.
    \end{definicio}
    \begin{definicio}
        Un esdeveniment és una col·lecció de subconjunts de l'espai mostral. Es pot calcular la
        probabilitat d'un esdeveniment. Ha de tenir estructura de $\sigma$-àlgebra.
    \end{definicio}
    \begin{notacio}
        Si $\omega \in \Omega$ és un resultat de l'experimental tal que $\omega \in A$, diem que $A$
        s'ha realitzat.
    \end{notacio}
    \begin{definicio}
        Sigui $\mathcal{A}$ una col·lecció de subconjunts d'$\Omega$. $\mathcal{A}$ és una
        $\sigma$-àlgebra si es compleix el següent:
        \begin{enumerate}
            \item $\Omega \in \mathcal{A}$
            \item Si $A \in \mathcal{A}$, aleshores $A^c \in \mathcal{A}$.
            \item Si $A_1, A_2, \dots$ són elements d'$\mathcal{A}$, aleshores $\bigcup\limits_{n=1}^\infty A_n \in \mathcal{A}$
        \end{enumerate}
    \end{definicio}
    \begin{corolari}[propietats d'una $\sigma$-àlgebra]
        Sent $\mathcal{A}$ una $\sigma$-àlgebra
        \begin{itemize}
            \item $\emptyset \in \mathcal{A}$
            \item $A, B \in \mathcal{A} \Rightarrow A \cap B \in \mathcal{A}$
            \item $A, B \in \mathcal{A} \Rightarrow B \setminus A = B\cap A^c \in \mathcal{A}$
        \end{itemize}
    \end{corolari}
    \begin{definicio}[Fórmula de Laplace]
        La probabilitat d'un esdeveniment $\mathcal{A}$ sempre que el conjunt de resultats possibles
        sigui finit i equiprobable, la fórmula de Laplace es pot aplicar.
        \begin{displaymath}
            \mathbb{P}(A) = \frac{\text{Casos probables a } A }{\text{Casos possibles}}
        \end{displaymath}
    \end{definicio}
    Una altra manera de calcular la probabilitat és fent servir una visió freqüentista:
    \begin{displaymath}
        \mathbb{P}(A) = \lim_{n\rightarrow\infty}f_n(A) \hbox{ on } f_n(A):= \frac{\text{nombre de cops que hem obtingut } A}{n}
    \end{displaymath}
    \begin{definicio}[axiomes de Kolmogorov]
        Siguin $\Omega$ un conjunt i $\mathcal{A}$ una $\sigma$-àlgebra sobre $\Omega$. Una
        probabilitat és qualsevol aplicació $\mathbb{P} : \mathcal{A} \longrightarrow [0,1]$ que
        compleix el següent:
        \begin{itemize}
            \item $\mathbb{P}(\Omega) = 1$
            \item Si $\{A_n, n\geq 1\} \subset \mathcal{A}$ són disjunts dos a dos llavors
            \begin{displaymath}
                \mathbb{P} (\bigcup\limits_{n=1}^\infty) = \sum\limits_{n=1}^{\infty}\mathbb{P}(A_n)
            \end{displaymath}
        \end{itemize}
    \end{definicio}
    \begin{definicio}
        Un espai de probabilitat és la terna $(\Omega, \mathcal{A}, \mathbb{P})$.
    \end{definicio}
    \begin{notacio}
        Per a unions disjuntes fem servir $\uplus$.
    \end{notacio}
    \begin{corolari} Propietats dels axiomes de Kolmogorov.
        \begin{enumerate}
            \item $\mathbb{P}(\emptyset) = 0$
            \item $A, B \in \mathcal{A} \Rightarrow \mathbb{P}(A\cup B) = \mathbb{P}(A) + \mathbb{P}(B) - \mathbb{P}(A \cap B)$
            \item $A \subset B \Rightarrow \mathbb{P}(A) \leq \mathbb{P}(B)$
            \item $A \subset B \Rightarrow \mathbb{P}(B\setminus A) = \mathbb{P}(B) - \mathbb{P}(A)$
            \item $\mathbb{P}(A\cup B) \leq \mathbb{P}(A) + \mathbb{P}(B)$
        \end{enumerate}
    \end{corolari}
    \begin{definicio}
        Quan parlem de \textit{odds} de $A$, definim:
        \begin{itemize}
            \item Odds a favor de $A$: $\text{Odds}(A) = \frac{\mathbb{P}(A)}{\mathbb{P}(A^c)}$
            \item Odds en contra de $A$: $\text{Odds}(A^c) = \frac{\mathbb{P}(A^c)}{\mathbb{P}(A)}$
        \end{itemize}
    \end{definicio}
    \begin{exemple}
        $\text{Odds}(A) = \frac{3}{2} \Longleftrightarrow \mathbb{P}(A) = \frac{3}{2}\mathbb{P}(A^c)$
        i sabem que $\mathbb{P}(A^c) = 1 - \mathbb{P}(A)$, llavors en resoldre tenim $\mathbb{P}(A) = 0.6$
        i $\mathbb{P}(A^c) = 0.4$.
    \end{exemple}
    \begin{notacio}
        $\hbox{}$
        \begin{itemize}
            \item Permutacions de $n$ elements: $\mathbf{P}_n = n!$
            \item Variacions de $n$ elements sense reposició on triem $m\leq n$: $\mathbf{V}_n^m = \frac{n!}{(n-m)!}$
            \item Variacions de $n$ elements amb reposició on triem $m$: $\mathbf{VR}_n^m = n^m$
            \item Combinacions de $n$ elements: $\mathbf{C}_n^m = (\substack{n\\m}) = \frac{n!}{(n-m)!m!} = \frac{\mathbf{V}_n^m}{\mathbf{P}_m}$
        \end{itemize}
    \end{notacio}
    \subsection{Probabilitat condicionada}
    \begin{exemple}
        Llancem dos daus distingibles. Sabem llavors que \\$\Omega = \left\{ \left(1,1\right), \left(1,2\right), \dots  \left(6,6\right) \right\}$,
        $\#\Omega = 36$ simètric. $\mathcal{A} = \mathcal{P}\left(\Omega\right)$, $\mathbb{P}$ és
        tal que $\mathbb{P}\left(\left\{i,j\right\} \right) = \frac{1}{36}$. Sigui $B \equiv $ "la
        puntuació dels daus és la mateixa" $\Rightarrow \mathbb{P}\left(B\right) = \frac{6}{36} = \frac{1}{6} \approx 0.17$
        Abans de mirar el resultat del dau, ens diuen que la suma dels daus és $8$. Com recalculem
        la probabilitat de $B$?
    \end{exemple}
    \begin{notacio}
        Denotem que un esdeveniment condicionat per $A$ de la manera $..|A$.
    \end{notacio}
    \begin{definicio}
        Si $A, B$ són esdeveniments tals que $\mathbb{P}\left(A\right)>0$ definim
        \begin{displaymath}
            \mathbb{P}\left(B|A\right) = \frac{\mathbb{P}\left(A\cap B\right)}{\mathbb{P}\left(A\right)}
        \end{displaymath}
    \end{definicio}
    \begin{exemple}(continuem l'anterior)
        Per tant, $A \equiv$ "la suma $=8$" llavors
        \begin{displaymath}
            \mathbb{P}\left(B|A\right) = \frac{\mathbb{P}\left(A\cap B\right)}{\mathbb{P}\left(A\right)} = \frac{\frac{1}{36}}{\frac{5}{36}} = \frac{1}{5} = 0.2 
        \end{displaymath}
    \end{exemple}
    \begin{obs}Aquestes dues propietats es compleixen:
        \begin{itemize}
            \item $\mathbb{P}\left(B^c|A\right) = 1 - \mathbb{P}\left(B|A\right)$
            \item $\mathbb{P}\left(C_1 \cup C_2 | A\right) = \mathbb{P}\left(C_1|A\right) + \mathbb{P}\left(C_2|A\right) - \mathbb{P}\left(C_1 \cap C_2|A\right)$
        \end{itemize}
        Però, en general, no és cert que $\mathbb{P}\left(B|A^c\right) = 1 - \mathbb{P}\left(B|A\right)$.
    \end{obs}
    \begin{teorema}[Fórmula de les probabilitats compostes]
        Regla del producte
        \begin{displaymath}
            \mathbb{P}\left(A \cap B\right) = \mathbb{P}\left(B|A\right)\mathbb{P}\left(A\right)\text{, sempre que } \mathbb{P}\left(A\right)>0
        \end{displaymath}
    \end{teorema}
    \begin{definicio}
        Donats $A$ i $B$ esdeveniments, direm que són independents si $\mathbb{P}\left(A\cap B\right) = \mathbb{P}\left(A\right)\mathbb{P}\left(B\right)$
    \end{definicio}
    \begin{obs}
        Si $\mathbb{P}\left(A\right) > 0$, $\mathbb{P}\left(A\cap B\right) = \mathbb{P}\left(A\right) \mathbb{P}\left(B\right) \Longleftrightarrow \mathbb{P}\left(B|A\right) = \mathbb{P}\left(B\right)$\\
        Si $\mathbb{P}\left(B\right) > 0$, $\mathbb{P}\left(A\cap B\right) = \mathbb{P}\left(A\right) \mathbb{P}\left(B\right) \Longleftrightarrow \mathbb{P}\left(A|B\right) = \mathbb{P}\left(A\right)$
    \end{obs}
    \begin{proposicio}[Propietats dels esdeveniments independents]
        Són equivalents
        \begin{itemize}
            \item $A$ i $B$ independents
            \item $A$ i $B^c$ independents
            \item $A^c$ i $B$ independents
            \item $A^c$ i $B^c$ independents
        \end{itemize}
    \end{proposicio}
    \begin{definicio}
        Diem que $3$ esdeveniments $A$, $B$, $C$ són independents si compleixen
        \begin{itemize}
            \item Són independents dos a dos, és dir
            \begin{displaymath}
                \begin{cases}
                    \mathbb{P}\left(A\cap B\right) = \mathbb{P}\left(A\right)\mathbb{P}\left(B\right)\\
                    \mathbb{P}\left(B\cap C\right) = \mathbb{P}\left(B\right)\mathbb{P}\left(C\right)\\
                    \mathbb{P}\left(A\cap C\right) = \mathbb{P}\left(A\right)\mathbb{P}\left(C\right)
                \end{cases}
            \end{displaymath}
            \item Compleixen $\mathbb{P}\left(A\cap B\cap C\right) = \mathbb{P}\left(A\right)\mathbb{P}\left(B\right)\mathbb{P}\left(C\right)$
        \end{itemize}
    \end{definicio}
    $\hbox{}$\\
    \begin{teorema}[Formula de las probabilitats totals]
        Considerem \\$B_1, B_2, \dots, B_n \in \mathcal{A}$ que siguin una partició de $\Omega$, és a dir, $\Omega = \biguplus\limits_{i=1}^{n} B_i$,
        tals que $\mathbb{P}\left(B_i\right) > 0$, $\forall i \in \left\{1, 2, \dots, n\right\}$,
        sigui $A \in \mathcal{A}$, aleshores
        \begin{displaymath}
            \mathbb{P}\left(A\right) = \sum\limits_{i=1}^{n} \mathbb{P}\left(A|B_i\right) \mathbb{P}\left(B_i\right)
        \end{displaymath}
    \end{teorema}
    \begin{teorema}[Formula de las probabilitats totals condicionada]
        Considerem \\$B_1, B_2, \dots, B_n \in \mathcal{A}$ que siguin una partició de $\Omega$, és a dir, $\Omega = \biguplus\limits_{i=1}^{n} B_i$,
        tals que $\mathbb{P}\left(B_i\right) > 0$, $\forall i \in \left\{1, 2, \dots, n\right\}$,
        sigui $A \in \mathcal{A}$ i $C \in \mathcal{A}$ amb $\mathbb{P}\left(C\right)>0$, aleshores
        \begin{displaymath}
            \mathbb{P}\left(A|C\right) = \sum\limits_{i=1}^{n} \mathbb{P}\left(A|B_i\cap C\right) \mathbb{P}\left(B_i|C\right)
        \end{displaymath}
    \end{teorema}
    \begin{teorema}[Fórmula de Bayes]
        Considerem \\$B_1, B_2, \dots, B_n \in \mathcal{A}$ que siguin una partició de $\Omega$, és a dir, $\Omega = \biguplus\limits_{i=1}^{n} B_i$,
        tals que $\mathbb{P}\left(B_i\right) > 0$, $\forall i \in \left\{1, 2, \dots, n\right\}$,
        sigui $A \in \mathcal{A}$ amb $\mathbb{P}\left(A\right)>0$, aleshores
        \begin{displaymath}
            \mathbb{P}\left(B_j|A\right) = \frac{\mathbb{P}\left(A|B_j\right)\mathbb{P}\left(B_j\right)}{\mathbb{P}\left(A\right)} = \frac{\mathbb{P}\left(A|B_j\right)\mathbb{P}\left(B_j\right)}{\sum\limits_{i=1}^{n} \mathbb{P}\left(A|B_i\right) \mathbb{P}\left(B_i\right)}
        \end{displaymath}
    \end{teorema}
    \begin{displaymath}
        \underbrace{LR}_\text{Rao de versemblances} = \frac{\text{Odds}\left(A|E\right)}{\text{Odds}\left(A\right)}
    \end{displaymath}
    \begin{displaymath}
        LR \begin{cases}
        \approx 1 & \text{l’evidència E no dona suport a $A$ ni al seu contrari}\\
        > 1 & \text{l’evidència E dona suport a $A$}\\
        < 1 & \text{l’evidència E dona suport al contrari de $A$}
        \end{cases}
    \end{displaymath}
    \subsection{Variables aleatòries}
    \begin{definicio}
        Diem que una aplicació $x: \Omega \to \mathbb{R}$ és una variable aleatòria si, per a tot
        interval $B$ (o semirecta), tenim que
        \begin{displaymath}
            \left\{\omega \in \Omega, X\left(\omega\right) \in \mathcal{A}\right\}
        \end{displaymath}
    \end{definicio}
    \begin{obs}
        La $\mathbb{P}$ \underline{no} intervé en la definició de v.a., però la motivació és poder
        calcular-la.
    \end{obs}
    \begin{notacio}
        $\mathbb{P}\left(\left\{\omega \in \Omega, X\left(\omega\right) \in B\right\} \right) = \mathbb{P}\left(X \in B\right)$  
    \end{notacio}
    \begin{notacio}
        Si $B = \left\{a\right\}$, ho escriurem $\mathbb{P}\left(X=a\right)$. Es farà també amb $<$,
        $\leq$ i els seus equivalents $>$, $\geq$.
    \end{notacio}
    \begin{obs}
        IMPORTANT: Si $\mathcal{A} = \mathcal{P}\left(\Omega\right)$, aleshores tota l'aplicació $X: \Omega \to \mathbb{R}$
        és una v.a. ($\mathcal{P}\left(\Omega\right)$ és el conjunt de totes les particions possibles amb $\Omega$)
    \end{obs}
    \begin{proposicio}
        Siguin $X, Y$ v.a. Sigui $a \in \mathbb{R}$. Aleshores:
        \begin{displaymath}
            X+Y, XY, aX, |X|, X^2
        \end{displaymath}
        són també v.a.\\
        Si $X\left(\omega\right) \neq 0 \hbox{ }\forall \omega \in \Omega \Rightarrow \frac{1}{X}$ és una v.a.
    \end{proposicio}
    \begin{proposicio}
        Tenim $\left\{X_n, n\geq 1\right\}$ successió de v.a.\\
        \begin{displaymath}
            x_n: \Omega \to \mathbb{R}
        \end{displaymath} 
        Suposem que $\forall \omega \in \Omega \hbox{ } \left\{X_n\left(\omega\right) , n\geq 1\right\}$
        (successió de nombres reals) és convergent, llavors
        \begin{definicio}
            \begin{displaymath}
                X: \Omega \to \mathbb{R}
            \end{displaymath}
            \begin{displaymath}
                \omega \longmapsto X\left(\omega\right) = \lim\limits_{n \to \infty} X_n \left(\omega\right) 
            \end{displaymath}
        \end{definicio}
    \end{proposicio}
    \subsubsection{Funció de distribució d'una v.a.}
    \begin{definicio}
        Sigui $X$ una v.a. la \underline{funció de distribució} de $X$ és 
        \begin{displaymath}
            F: \mathbb{R} \to \left[0, 1\right]
        \end{displaymath}
        \begin{displaymath}
            x \longmapsto F\left(x\right)  = \mathbb{P}\left(X \leq x\right) 
        \end{displaymath}
    \end{definicio}
    \begin{obs}
        $F$ ben definida, perquè
        \begin{displaymath}
            \mathbb{P}\left(X\leq x\right) = \mathbb{P}\left(X \in (-\infty, x]\right) \text{, i } B = (-\infty, x] \subset \mathbb{R}
        \end{displaymath}
    \end{obs}
    \begin{proposicio}[Propietats]
        $\hbox{}$
        \begin{enumerate}
            \item $F$ no decreixent: $x < y \Longrightarrow F\left(x\right) \leq F\left(y\right)$
            \item $F$ és contínua per la dreta i té límits per l'esquerra.  
            \item $\lim\limits_{x \to -\infty} F\left(x\right) = 0, \lim\limits_{x \to \infty} F\left(x\right) = 1$
            \item $F$ té, com a màxim, un nombre numerable de discontinuïtats.
            \item $\mathbb{P}\left(s \leq x \leq t\right) = F\left(t\right) - F\left(s\right)$
            \item $\mathbb{P}\left(X < s\right) = F\left(x^-\right)$
            \item $\mathbb{P}\left(X = x\right) = F\left(x\right) - F\left(x^-\right)$ IMPORTANT: $F$
            es discontinua en $x \Longleftrightarrow \mathbb{P}\left(X=x\right)>0$ 
        \end{enumerate}
    \end{proposicio}
    \begin{definicio}
        Una v.a. $X$ és discreta si $\exists S \subset \mathbb{R}$ finit o numerable tal que 
        \begin{displaymath}
            \mathbb{P}\left(X \in S\right) = 1
        \end{displaymath}
        \begin{itemize}
            \item $S$ suport de $X$
            \item Suposem que tot valor a $S$ compleix que $\mathbb{P}\left(X = x\right) > 0$
            \item Suposem també que $S = \left\{x_i, i\in I\right\}, I \subset \mathbb{N}$ 
        \end{itemize}
        \begin{definicio}
            La funció de probabilitat d'una v.a. discreta $X$ amb suport $S$ és
            \begin{displaymath}
                p: S \to \left[0, 1\right]
            \end{displaymath}
            \begin{displaymath}
                x_i \longmapsto p\left(x_i\right)  = \mathbb{P} \left(X = x_i\right)  
            \end{displaymath}
        \end{definicio}
    \end{definicio}
    \begin{notacio}
        També, es pot escriure $p_i = p\left(x_i\right)$
    \end{notacio}
    \begin{obs}
        IMPORTANT: $\forall B \subset \mathbb{R}$ interval, tindrem que $\mathbb{P}\left(X \in B\right) = \sum\limits_{i \in I\atop x_i \in B} p_i$
    \end{obs}
    \begin{obs}
        A partir de la funció de probabilitat es pot calcular tot.
    \end{obs}
\end{document}