\documentclass[../main.tex]{subfiles}
\graphicspath{{\subfix{../images/}}}

\begin{document}
    \begin{definicio}
        Un graf és un objecte combinatori que està format per un parell ordenat de vèrtex i arestes ($G = (V, E)$).
        Una aresta ($E$) està etiquetat per un origen i un destí (o extrems si no estan orientades)
        sent aquests vèrtexs ($V$).
    \end{definicio}
    \begin{definicio}
        Un graf dirigit, o orientat, és un graf on les arestes tenen direcció, com si fos una fletxa.
    \end{definicio}
    \begin{definicio}
        Un graf no dirigit serà un graf on les arestes no tenen direccions. Podem suposar que
        l'aresta és bidireccional.
    \end{definicio}
    \begin{definicio}
        Un graf és planar si es pot unir tots els vèrtexs sense que es creuin les arestes.
        Si s'han de creuar obligatòriament, és un graf no planar.
    \end{definicio}
    \begin{teorema}
        Tot graf no planar té un subgraf $K_{3,3}$ o $P_5$.
    \end{teorema}
\end{document}