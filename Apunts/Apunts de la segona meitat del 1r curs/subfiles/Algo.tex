\documentclass[../main.tex]{subfiles}
\graphicspath{{\subfix{../images/}}}

\begin{document}
    \begin{definicio}
        Un graf és un objecte combinatori que està format per un parell ordenat de vèrtex i arestes ($G = (V, E)$).
        Una aresta ($E$) està etiquetat per un origen i un destí (o extrems si no estan orientades)
        sent aquests vèrtexs ($V$).
    \end{definicio}
    \begin{definicio}
        Un graf dirigit, o orientat, és un graf on les arestes tenen direcció, com si fos una fletxa.
    \end{definicio}
    \begin{definicio}
        Un graf no dirigit serà un graf on les arestes no tenen direccions. Podem suposar que
        l'aresta és bidireccional.
    \end{definicio}
    \begin{definicio}
        Un graf és planar si es pot unir tots els vèrtexs sense que es creuin les arestes.
        Si s'han de creuar obligatòriament, és un graf no planar.
    \end{definicio}
    \begin{teorema}
        Tot graf no planar té un subgraf $K_{3,3}$ o $P_5$.
    \end{teorema}
    \begin{definicio}[Propietats dels grafs]
        $\hbox{ }$
        \begin{enumerate}
            \item L'\textbf{ordre} d'un graf és el nombre de vèrtex
            \item La \textbf{grandària} d'un graf és el nombre d'arestes
            \item La \textbf{valència} d'un vèrtex és el nombre d'arestes entrant o sortint del vèrtex.
            Si surt i es connecta en si mateix compta com dos.
            \item La \textbf{valència d'entrada} és el nombre d'arestes entrant.
            \item La \textbf{valència de sortida} és el nombre d'arestes sortint.
            \item Els vèrtexs amb valència 1 s'anomenen \textbf{fulles}.
            \item Els vèrtexs amb valència més gran que dos es diuen \textbf{branching} (o \textbf{encreuament}).
            \item Un \textbf{camí} és la seqüència de vèrtexs connectats linealment. La llargària d'un camí
            és el nombre de vèrtexs.
            \item Un \textbf{circuit} és un camí tancat, és a dir, un camí que comença i termina al mateix lloc.
        \end{enumerate}
    \end{definicio}
    \begin{notacio}
        Probablement, per denotar un camí si és un bàsic, ens bastarà amb dir d'on ve fins on va, és 
        dir $A\rightarrow B$. Això no obstant, possiblement la millor forma és posar etiquetes a les
        arestes, de tal que $\sigma_1 := A \rightarrow B$. Si volem denotar que comença en $B$ i
        termina en $A$, s'hi diu $\sigma_1^{-1}$. Així, un $A \rightarrow B \rightarrow C$ seria $\sigma_1 \cdot \sigma_2$.
        Aquest $\cdot$ no és commutatiu.
    \end{notacio}
    \begin{definicio}
        S'hi diu que un circuit és reduït si no té $\sigma_i \cdot \sigma_i^{-1}$.
    \end{definicio}
    \begin{definicio}
        Un graf no dirigit és \textbf{connex} si hi ha un camí des de tot vèrtex qualsevol fins a tot 
        altre.
    \end{definicio}
    \begin{definicio}
        Un component connex és un subgraf connex i maximal.
    \end{definicio}
    \begin{definicio}
        Un graf dirigit és \textbf{connex} si hi ha un camí des de tot vèrtex qualsevol fins a tot 
        altre.
    \end{definicio}
    \begin{proposicio}
        Per a un graf $(V, E)$ no orientat les afirmacions següents són equivalents:
        \begin{enumerate}
            \item $(V, E)$ és un graf connex.
            \item $\forall v_o\in V$, existeix un camí d'arestes de $v_o$ a $v$, $\forall v \in V$.
            \item $\exists v_o \in V$ tal que existeix un camí d'arestes de $v_o$ a $v$.
        \end{enumerate}
    \end{proposicio}
    \begin{notacio}
        Fem servir $\circ$ per multiplicar camins. 
    \end{notacio}
    \begin{definicio}
        Un graf dirigit és \textbf{feblement connex} si hi ha un camí des de tot vèrtex qualsevol fins a tot 
        altre sense fer servir l'orientació, és a dir, seria connex en cas que fos no orientat.
    \end{definicio}
    \begin{definicio}
        Un graf dirigit és \textbf{semi connex} si al escullir qualssevol dos vèrtexs del graf hi
        existeix un camí connectant-los, sigui d'un sentit o d'altre.
    \end{definicio}
    Podem representar un graf no orientat amb una matriu de adjaciencia, també la matriu de
    incidencia, on cada vèrtex és una fila i una aresta és una columna, problema si hi ha una amb
    el mateix origen i destí.\\
    Nosaltres farem servir llistes de adjaciencia.
    \begin{definicio}
        Un \textbf{arbre} és un graf sense circuits reduïts.
    \end{definicio}
    \begin{proposicio}[Tecnicament es un lema]
        Equivalentment, un arbre és un graf on tota parella de vèrtexs estan connectats per un únic
        camí reduït.
    \end{proposicio}
    \begin{exercici}[Demostració de l'anterior] $\hbox{}$
        \begin{itemize}
            \item[$\Rightarrow)$] Suposem que $\exists v, w$ vèrtexs de $G$ que es poden unir per dos (o més) camins reduïts
            $\sigma$ i $\tau$. Com que $\tau\neq\sigma \hbox{ } \exists j : a_j \neq b_j$. $j_0 = \min\limits_j\{a_j\neq b_j\}$
            i $m$ com el mínim $m\geq j_0$ tal que $\exists n\geq j_0 : V_m = W_n$. Llavors tenim el circuit $a_{j_0} \circ a_{j_0+1} \circ \dots \circ a_{m} \circ b_{n}^{-1} \circ \dots \circ b_{j_0}^{-1}$.
            I per tant, no és un arbre.
            \item[$\Leftarrow)$] Suposem que no és un arbre, això implica que té un circuit reduït
            no trivial. Anomenem $\sigma$ un camí que va de $V$ a $V$. Llavors tenim dos camins,
            $\sigma$ i $\emptyset$. Això, trivialment, no és equivalent.
        \end{itemize}
    \end{exercici}
    \begin{corolari}[Propietats dels arbres]$\hbox{}$
        \begin{itemize}
            \item Si s'afegeix qualsevol aresta, deixa de ser un arbre.
            \item Eliminar qualsevol aresta fa que l'arbre es desconnecti.
            \item Un arbre de $n$ vèrtex té exactament $n-1$ arestes.
        \end{itemize}
    \end{corolari}
    \begin{definicio}
        Si $G$ és un graf, definim la característica d'Euler-Poincaré d'un graf com:
        \begin{displaymath}
            \chi\left(G\right) = \underbrace{\left\lvert V \right\rvert}_\text{nombre de vèrtexs} - \underbrace{\left\lvert E \right\rvert}_\text{nombre d'arestes}
        \end{displaymath}
    \end{definicio}
    \begin{teorema}
        Si $G$ és un graf connex, són equivalents:
        \begin{itemize}
            \item[a)] $G$ és un arbre.
            \item[b)] $\forall v, w$ vèrtexs de $G$, existeix un únic camí reduït que els uneix.
            \item[c)] $\chi(G) = 1$. 
        \end{itemize}
    \end{teorema}
    \begin{definicio}
        Si $G$ és un graf, $T \subseteq G$ és un arbre maximal si:
        \begin{itemize}
            \item $T$ és un arbre
            \item $T \subseteq G$ es maximal si considerem $\overline{G}$ un subgraf de $G$ tal que
            contingui $T$ i sigui diferent de $T$ llavors $\overline{G}$ \underline{no} és un arbre.
        \end{itemize}
    \end{definicio}
    \begin{proposicio}[Tecnicament un lema]
        Sigui $G$ un graf connex
        \begin{itemize}
            \item[a)] Tot arbre $T \subseteq G$ està contingut en un arbre maximal.
            \item[b)] $T \subseteq G$ és un arbre maximal $\Leftrightarrow$
            \begin{itemize}
                \item[$\bullet$] és un arbre
                \item[$\bullet$] té tots els vèrtexs de $G$
            \end{itemize}
        \end{itemize}
    \end{proposicio}
    \begin{exercici}[Demostració de l'anterior lema]
        $\hbox{}$
        \begin{itemize}
            \item[a)] Fem tots els possibles subgrafs (un nombre finit) i ens quedem amb els arbres
            més grans.
            \item[b)] $G$ és un graf connex i $T \subseteq G$ és un arbre maximal i que no conté
            tots els vèrtexs, volem una contradicció. Sigui $V$ un vèrtex de l'arbre, i $W$ un
            vèrtex que no sigui de l'arbre, com $G$ és connex hi ha un camí que uneix $V$ i $W$.
            Sigui $j_0 = \max\limits_{j}\{V_k \in T \hbox{ }\forall k \leq j\}$ $j_0 < l$ ($W$ no és
            part de l'arbre). $V_{j_0} \in T$, $V_{j_0+1} \notin T$. Considerem $\overline{T} = T \cup \{a\} \cup \{V_{j_0}+1\}$,
            com $\overline{T}$ és un arbre més gran que $T$, per tant $T$ no era maximal.
        \end{itemize}
    \end{exercici}
    \begin{proposicio}[Tecnicament un lema]
        Si $T$ és un arbre, llavors $\exists v \in T$ que sigui fulla. 
    \end{proposicio}
    \begin{exercici}[Demostració de l'anterior lema]
        Suposem que tots els vèrtexs de $T$ tenen valència $\geq 2$. Sigui $v_0 \in T$, triem una
        aresta $a_1$ que surti de $v_0$ i arribi a $v_1$. Continuem el camí, perquè té valència $\geq 2$,
        sempre fent que $a_2 \neq a_1$, en continuar aquest camí, triem un camí reduït amb tants
        vèrtexs com es vulguin, per exemple un amb $\left\lvert V\right\rvert +1$ vèrtexs. Llavors
        n'hi haurà un de repetit $V_i = V_j$ amb $i < j$. És un circuit reduït, i per tant $T$ no és
        un arbre.
    \end{exercici}
    Recordem que també cal demostrar que si $G$ és un graf connex llavors $G$ és un arbre $\iff \chi\left(G\right) = 1$
    \begin{exercici}[Demostració de a) $\iff$ c) de l'ultim teorema possat]
        $\hbox{}$
        \begin{itemize}
            \item[$\Rightarrow)$] Suposem que $G$ és un arbre, demostrem per inducció que $\left\lvert V\right\rvert$
            \begin{itemize}
                \item[$\bullet$] $\left\lvert V \right\rvert = 1$, llavors $\chi\left(G\right) = 1-0 = 0$
                i $G$ és un arbre.\\
                Suposem que això es continua complint amb $n$ vèrtexs.
                \item[$\bullet$] $\left\lvert V \right\rvert = n + 1$. Aplicant el lema, si $v_0$ un
                vèrtex de $G$ amb València $1$, per tant hi ha una única aresta que arriba a $v_0$ i $a_0$.\\
                Sigui $H \subset G$ un subgraf que té tots els vèrtexs i arestes de $G$ excepte $a_0$
                i $v_0$, llavors $\left\lvert V_H\right\rvert = n$ i $\left\lvert E_H\right\rvert = n-1$.
                Com se segueix complint, sabem que $\chi\left(H\right) = 1$, i per allò $\chi\left(G\right) = 1$
            \end{itemize}
            \item[$\Leftarrow)$] Sigui $G$ un graf connex amb $\chi\left(G\right) = 1$, volem veure
            que $G$ és un arbre.\\
            Sigui $T\subseteq G$ un arbre maximal. Sigui $n = \left\lvert V_G\right\rvert$ i que $\left\lvert E_G\right\rvert = n-1$
            Llavors, $T$ té $n$ vèrtexs i com $\chi\left(T\right) = 1 \Rightarrow G = T$ i per allò,
            $G$ és un arbre.
        \end{itemize}
    \end{exercici}
\end{document}