\documentclass[../main.tex]{subfiles}
\graphicspath{{\subfix{../images/}}}

\begin{document}
    \begin{definicio}
        Un graf és un objecte combinatori que està format per un parell ordenat de vèrtex i arestes ($G = (V, E)$).
        Una aresta ($E$) està etiquetat per un origen i un destí (o extrems si no estan orientades)
        sent aquests vèrtexs ($V$).
    \end{definicio}
    \begin{definicio}
        Un graf dirigit, o orientat, és un graf on les arestes tenen direcció, com si fos una fletxa.
    \end{definicio}
    \begin{definicio}
        Un graf no dirigit serà un graf on les arestes no tenen direccions. Podem suposar que
        l'aresta és bidireccional.
    \end{definicio}
    \begin{definicio}
        Un graf és planar si es pot unir tots els vèrtexs sense que es creuin les arestes.
        Si s'han de creuar obligatòriament, és un graf no planar.
    \end{definicio}
    \begin{teorema}
        Tot graf no planar té un subgraf $K_{3,3}$ o $P_5$.
    \end{teorema}
    \begin{definicio}[Propietats dels grafs]
        $\hbox{ }$
        \begin{enumerate}
            \item L'\textbf{ordre} d'un graf és el nombre de vèrtex
            \item La \textbf{grandària} d'un graf és el nombre d'arestes
            \item La \textbf{valència} d'un vèrtex és el nombre d'arestes entrant o sortint del vèrtex.
            Si surt i es connecta en si mateix compta com dos.
            \item La \textbf{valència d'entrada} és el nombre d'arestes entrant.
            \item La \textbf{valència de sortida} és el nombre d'arestes sortint.
            \item Els vèrtexs amb valència 1 s'anomenen \textbf{fulles}.
            \item Els vèrtexs amb valència més gran que dos es diuen \textbf{branching} (o \textbf{encreuament}).
            \item Un \textbf{camí} és la seqüència de vèrtexs connectats linealment. La llargària d'un camí
            és el nombre de vèrtexs.
            \item Un \textbf{circuit} és un camí tancat, és a dir, un camí que comença i termina al mateix lloc.
        \end{enumerate}
    \end{definicio}
    \begin{definicio}
        Un graf no dirigit és \textbf{connex} si hi ha un camí des de tot vèrtex qualsevol fins a tot 
        altre.
    \end{definicio}
    \begin{definicio}
        Un component connex és un subgraf connex i maximal.
    \end{definicio}
    \begin{definicio}
        Un graf dirigit és \textbf{connex} si hi ha un camí des de tot vèrtex qualsevol fins a tot 
        altre.
    \end{definicio}
    \begin{proposicio}
        Per a un graf $(V, E)$ no orientat les afirmacions següents són equivalents:
        \begin{enumerate}
            \item $(V, E)$ és un graf connex.
            \item $\forall v_o\in V$, existeix un camí d'arestes de $v_o$ a $v$, $\forall v \in V$.
            \item $\exists v_o \in V$ tal que existeix un camí d'arestes de $v_o$ a $v$.
        \end{enumerate}
    \end{proposicio}
    \begin{notacio}
        Fem servir $\circ$ per multiplicar camins. 
    \end{notacio}
    \begin{definicio}
        Un graf dirigit és \textbf{feblement connex} si hi ha un camí des de tot vèrtex qualsevol fins a tot 
        altre sense fer servir l'orientació, és a dir, seria connex en cas que fos no orientat.
    \end{definicio}
    \begin{definicio}
        Un graf dirigit és \textbf{semi connex} si al escullir qualssevol dos vèrtexs del graf hi
        existeix un camí connectant-los, sigui d'un sentit o d'altre.
    \end{definicio}
    Podem representar un graf no orientat amb una matriu de adjaciencia, també la matriu de
    incidencia, on cada vèrtex és una fila i una aresta és una columna, problema si hi ha una amb
    el mateix origen i destí.\\
    Nosaltres farem servir llistes de adjaciencia.
\end{document}