\documentclass[../main.tex]{subfiles}
\graphicspath{{\subfix{../images/}}}

\begin{document}
    \begin{definicio}
        $\mathbb{R}^n = \{\left( x_1, x_2, \dots, x_n \right): x_1, \dots, x_n \in \mathbb{R}\}$
    \end{definicio}
    \begin{definicio}
        Siguin $\left( x_1, x_2, \dots, x_n \right) \in \mathbb{R}^n$ i $\left( y_1, y_2, \dots, y_n \right) \in \mathbb{R}^n$,
        definim $\underbrace{\left\langle \left( x_1, x_2, \dots, x_n \right), \left( y_1, y_2, \dots, y_n \right) \right\rangle }_\text{ (producte escalar) } = x_1y_1 + x_2y_2 + \dots + x_ny_n$
        com a producte escalar.
    \end{definicio}
    \begin{definicio}
        $\left\lVert \left( x_1, x_2, \dots, x_n \right) \right\rVert = +\sqrt{\left\langle x, x \right\rangle }$ on $x = \left( x_1, x_2, \dots, x_n \right)$. Això és
        la distància del punt $x$ a $\left( 0, 0, \dots, 0 \right)$.  
    \end{definicio}
    Propietats de la norma:
    \begin{enumerate}
        \item $\left\lVert x\right\rVert \geq 0$ per tot $x \in \mathbb{R}^n$
        \item $\left\lVert \lambda x \right\rVert = \left\lvert \lambda \right\rvert \left\lVert x \right\rVert$ per tot $x \in \mathbb{R}^n$ i $\lambda \in \mathbb{R}$
        \item Desigualtat triangular: $\left\lVert x+y \right\rVert \leq \left\lVert x \right\rVert + \left\lVert y \right\rVert$ per tot $x, y \in \mathbb{R}^n$
    \end{enumerate}
    Desigualtat de Cauchy-Schwartz: $\left\langle x,y \right\rangle  \leq \left\lVert x\right\rVert\left\lVert y\right\rVert$ per tot $x, y \in \mathbb{R}^n$. Això ho
    acceptem.
    \begin{obs}
        Observem que $-1 \leq \frac{\left\langle x,y \right\rangle }{\left\lVert x \right\rVert\left\lVert y \right\rVert} \leq 1$ i definim l'angle entre els vectors $x$ i $y$
        com l'angle $\alpha$ tal que $\cos{\alpha} = \frac{\left\langle x,y \right\rangle }{\left\lVert x \right\rVert\left\lVert y\right\rVert}$,
        és a dir, $\left\langle x,y \right\rangle  = \left\lVert x \right\rVert\left\lVert y \right\rVert\cos{\alpha}$.
    \end{obs}
    \begin{exemple}
        Trobem els valors de $\mathbb{R}^3 \perp (-1, -2, 1)$. Busquem $\left( x_1, x_2, x_3 \right) \in \mathbb{R}^n$
        tal que $\left\langle \left( x_1, x_2, x_3 \right), (-1, -2, 1) \right\rangle  = 0 \Leftrightarrow -x_1-2x_2+x_3=0$.
    \end{exemple}
    \subsection{Boles a \texorpdfstring{$\mathbb{R}^n$}{diverses dimensions}}
    \underline{Si $n=2$} la bola de centre $\left( x_0, y_0 \right)$ i radi $R$ és $\{\left( x,y \right) \in \mathbb{R}^2: \left\lVert\left( x,y \right)-\left( x_0, y_0 \right)\right\rVert < R\} =$
    (disc) $= \{\left( x,y \right)\in \mathbb{R}^2: (x-x_0)^2+(y-y_0)^2<R^2\}$. Això és una bola oberta, denotada per
    \begin{notacio}
        Fem servir $\mathfrak{B}_R\left( x_0, y_0, \dots \right)$ la bola oberta ($< R$) i $\overline{\mathfrak{B}_R}\left( x_0, y_0, \dots \right)$
        la tancada ($\leq R$). Es farà servir més la bola oberta.
    \end{notacio}
    \begin{definicio}
        Sigui $A \subset \mathbb{R}^n$, definim $\mathring{A} = \{\vec{x} \in \mathbb{R}^n : \exists R > 0 | \mathfrak{B}_R(\vec{x}) \subset A\}$
    \end{definicio}
    \begin{exemple}
        $\hbox{ }$
        \begin{enumerate}
            \item $A = \{\left( x,y \right):x \geq 0\}$, llavors $\mathring{A} = \{\left( x,y \right):x > 0\}$
            \item $A = \{\left( x,y,z \right):-a \leq x \leq a, -b \leq y \leq b, -c \leq z \leq c\}$, llavors
            $\mathring{A} = \{\left( x,y,z \right):-a < x < a, -b < y < b, -c < z < c\}$
        \end{enumerate}
    \end{exemple}
    \begin{definicio}
        Un conjunt $A \subset \mathbb{R}^n$ es diu obert si $A = \mathring{A}$, és a dir, si tot
        punt del conjunt $A$ és també un punt d'$\mathring{A}$.
    \end{definicio}
    \begin{definicio}
        Sigui $A \subset \mathbb{R}^n$, definim $\overline{A} = \{ x \in \mathbb{R}^n : \forall R>0, \mathfrak{B}_R\left( x \right) \cap A \neq \emptyset\}$.
        Diem que $\overline{A}$ és l'adherència d'$A$.
    \end{definicio}
    \begin{exemple}
        $A = \{\left( x,y \right) \in \mathbb{R}^2: yx \left\langle 1,x \right\rangle  0\}$. $\overline{A} = \{\left( 0, y \right): y\in \mathbb{R}\} \cup \{\left( x, y \right): yx \leq 1, x > 0\}$
    \end{exemple}
    \begin{definicio}
        Un conjunt ($A \subset \mathbb{R}^n$) és tancat si $A = \overline{A}$.
    \end{definicio}
    \begin{proposicio}
        $A$ és obert $\Longleftrightarrow A^c$ és tancat. 
    \end{proposicio}
    \begin{definicio}
        La frontera d'un conjunt, $Fr\left( A \right) = \overline{A} \setminus \mathring{A}$.
    \end{definicio}
    \begin{definicio}
        $A \subset \mathbb{R}^n$ es diu acotat si $A$ està contingut dins d'una bola.
    \end{definicio}
    \begin{definicio}
        $A \subset \mathbb{R}^n$ es diu compacte si $A$ és tancat i acotat.
    \end{definicio}
    \begin{exercici}(4 de la llista 1) 
        Per als conjunts següents, determineu l'interior, l'adherència i la frontera. Decidiu si
        són oberts, tancats, acotats o compactes.
        \begin{enumerate}
            \item[a)] $A = \left\{\left(x, y\right) \in \mathbb{R}^2 : x, y < 0\right\}$\\
            $\mathring{A} = A \Rightarrow$ obert.\\
            $\overline{A} = \left\{\left(x, y\right) \in \mathbb{R}^2 : x, y \leq 0\right\}$\\
            $Fr\left(A\right)=\left\{\left(x, y\right) \in \mathbb{R}^2 : x, y = 0\right\}$\\
            No és acotat ni tancat, per això no és compacte.
            \item[b)] $B$ és la unió de totes les rectes de pendent nul que tallen l'eix $OY$ en un enter.\\
            $\mathring{B} = \emptyset$.\\
            $\overline{B} = B \Rightarrow$ tancat.\\
            $Fr\left(B\right) = B$
            No és acotat, per això no és compacte.
            \item[c)] $C = \left\{\left(x, y, z\right) \in \mathbb{R}^3 : x^2 + y^2 + z^2 \leq 1, z \geq 0, y \geq 0\right\}$.\\
            $\mathring{C} = \left\{\left(x, y, z\right) \in \mathbb{R}^3 : x^2 + y^2 + z^2 < 1, z > 0, y > 0\right\}$.
            $\overline{C} = C \Rightarrow$ tancat.\\
            $Fr\left(C\right) = \left\{\left(x, y, z\right) \in \mathbb{R}^3 : x^2 + y^2 + z^2 = 1, z = 0, y = 0\right\}$
            És compacte, ja que és acotat i tancat.
            \item[d)] $D$ és la intersecció dels discos $D_1 := \left\{\left(x, y\right) \in \mathbb{R}^2 : x^2+y^2 < 1\right\}$\\
            i $D_2 := \left\{\left(x, y\right) \in \mathbb{R}^2 : (x-1)^2+y^2 < 1\right\}$ $\left(D = D_1 \cap D_2\right)$.\\ 
            $\mathring{D} = \mathring{D_1} \cap \mathring{D_2}$.\\
            $\overline{D} = \overline{D_1} \cap \overline{D_2}$
            $Fr\left(D\right)  = \left(Fr{D_1} \cap \mathring{D_2}\right)\cup\left(Fr{D_2} \cap \mathring{D_1}\right)$
        \end{enumerate}
    \end{exercici}
    \begin{exercici}[5 de la llista 1]
        Proveu que la intersecció de dos oberts és un obert. Feu el mateix per a la unió.
        \begin{center}
            \begin{tikzpicture}
                \begin{scope}
                    \clip (1,1) circle (2em);
                    \fill[blu!60, thick, dashed] (2.25,1) circle (2em);
                \end{scope}
                \draw[black,thick,dashed] (1,1) circle (2em) node {P};
                \draw[black,thick,dashed] (2.25,1) circle (2em) node {Q};
            \end{tikzpicture}
        \end{center}
        Hem de provar que $P\cap Q$ és obert. $\forall x \in P\cap Q \hbox{ }\exists B_x \subset P\cap Q$,
        on $B_x$ és una bola centrada en el punt pertanyent a la intersecció.\\
        Com $P$ és obert, tot punt de $P$ té una bola pertanyent, passa el mateix en $Q$.\\
        $x \in B_x^P \subset P$, $x \in B_x^Q \subset Q$, $x \in B_x^P \cap B_x^Q \subset P\cap Q$.
        Llavors, com $P\cap Q$ té les propietats d'ambdues, és obert.
    \end{exercici}
    \begin{exercici}[6 de la llista 1]
        Proveu que la intersecció de dos tancats és un tancat. Feu el mateix per a la unió.
        \begin{center}
            \begin{tikzpicture}
                \begin{scope}
                    \clip (1,1) circle (2em);
                    \fill[blu!60, thick] (2.25,1) circle (2em);
                \end{scope}
                \draw[black,thick] (1,1) circle (2em) node {P};
                \draw[black,thick] (2.25,1) circle (2em) node {Q};
            \end{tikzpicture}
        \end{center}
        Sabem llavors que $P^c, Q^c$ són llavors oberts. Per l'exercici anterior, sabem que $P^c\cup Q^c$
        és obert. I, finalment, $\left(P^c\cup Q^c\right)^c = P\cup Q$ és tancada. 
    \end{exercici}
    \begin{exercici}[7 de la llista 1] Poseu un exemple d'una successió d'oberts $V_1, \dots$ de manera que $\bigcap V_j$ no sigui un obert.
        \begin{displaymath}
            \bigcap\limits_{n \in \mathbb{N}} \left(-1-\frac{1}{n}, 1+\frac{1}{n}\right) = [-1, 1]
        \end{displaymath}
        \begin{obs}
            Una unió de tancats no necessàriament és tancada.
        \end{obs}
    \end{exercici}
    \subsection{Límits de funcions i continuïtat}
    \begin{definicio}
        Sigui $A \subset \mathbb{R}^n$ i $f: A\rightarrow \mathbb{R}$, definim graf$\left( f \right) = \{(x, f\left( x \right)) \in \mathbb{R}^{n+1}: x\in A\}$.
    \end{definicio}
    \begin{definicio}
        El conjunt de nivell de la funció $f$ és $\{x \in A: f\left( x \right) = c\}$.
    \end{definicio}

    \subsubsection{Límit d'una funció a un punt}
    Siguin $A \subset \mathbb{R}^n$, $f: A \rightarrow \mathbb{R}$ i $x_0 \in \mathbb{R}^n$.
    \begin{definicio}
        Diem que $\lim\limits_{x\rightarrow x_0} f\left( x \right) = L$ si per tot $\varepsilon > 0$, existeix $\delta > 0$
        tal que $|f\left( x \right)-L| < \varepsilon$ si $0 < \left\lVert x-x_0\right\rVert < \delta$.
    \end{definicio}
    \begin{corolari}(Propietats dels límits)
        Suposem $\lim\limits_{x\rightarrow x_0} f\left( x \right) = L_1$, $\lim\limits_{x\rightarrow x_0} g\left( x \right) = L_2$
        \begin{enumerate}
            \item $\lim\limits_{x\rightarrow x_0} (f\left( x \right) + g\left( x \right)) = L_1 + L_2$
            \item $\lim\limits_{x\rightarrow x_0} (f\left( x \right) \times g\left( x \right)) = L_1 \times L_2$
            \item Si $L_2 \neq 0$, llavors $\lim\limits_{x\rightarrow x_0} (\frac{f\left( x \right)}{g\left( x \right)}) = \frac{L_1}{L_2}$ 
        \end{enumerate}
    \end{corolari}
    \begin{exemple}
        $\lim\limits_{\left( x,y \right) \rightarrow \left( 0, 0 \right)}(x^2+y^2)^\alpha \sin{\frac{1}{x^2+y^2}}$ on $\alpha \in \mathbb{R}$.
        \\Observem que $\lim\limits_{\left( x,y \right) \rightarrow \left( 0, 0 \right)}(x^2+y^2)^\alpha$ = $\begin{cases}
            0 \text{ si } \alpha > 0\\
            +\infty \text{ si } \alpha < 0\\
            1 \text{ si } \alpha = 0
        \end{cases}$
        \begin{itemize}
            \item Si $\alpha > 0$, observem $|(x^2+y^2)^\alpha \sin{\frac{1}{x^2+y^2}}| \leq (x^2+y^2) \longrightarrow 0$
            \item Si $\alpha < 0$  veiem que el límit no existeix.\\
            Trobem $\substack{\left( x_n, y_n \right) \rightarrow \left( 0, 0 \right)\\\left( z_n, w_n \right) \rightarrow \left( 0, 0 \right)}$ quan $n \rightarrow \infty$ per fer \begin{displaymath}
                \lim\limits_{\left( x_n,y_n \right) \rightarrow \left( 0, 0 \right)}(x_n^2+y_n^2)^\alpha \sin{\frac{1}{x_n^2+y_n^2}} = \lim\limits_{\left( z_n,w_n \right) \rightarrow \left( 0, 0 \right)}(z_n^2+w_n^2)^\alpha \sin{\frac{1}{z_n^2+w_n^2}}
            \end{displaymath}       
            Triem $\left( x_n, y_n \right)$ tal que $x_n^2 + y_n^2 = \frac{1}{n\pi}$ i triem $\left( z_n, w_n \right)$ tal que $z_n^2+w_n^2 = \frac{1}{\pi/2 + 2\pi n}$.\\
            Per exemple $x_n = \frac{1}{\sqrt{n\pi}}$, $y_n = 0$, $z_n = 0$ i $w_n = \frac{1}{\sqrt{\pi/2+2\pi n}}$.\\
            Això provocarà que el primer límit doni $0$ i el segon $\infty$, que, òbviament, no són
            iguals.
            \item Si $\alpha = 0$, exercici pel lector. No existeix.
        \end{itemize}
    \end{exemple}
    \subsubsection{Límit seguint rectes i límit a un punt}
    \begin{proposicio}
        Suposem $\lim\limits_{\left( x,y \right) \rightarrow \left( 0,0 \right)} f\left( x,y \right) = L$, llavors $\lim\limits_{x \rightarrow 0} f\left( x, mx \right) = L \hbox{ }\forall m \in \mathbb{R}$.
    \end{proposicio}
    Utilitzarem aquest fet de la següent forma:
    \begin{corolari}
        Si $\lim\limits_{x\rightarrow0} f\left( x, mx \right)$ depèn de $m$, aleshores $\lim\limits_{x\rightarrow0}f\left( x,y \right)$ no existeix.
    \end{corolari}
    \begin{exemple}
        $\lim\limits_{\left( x,y \right) \rightarrow\left( 0,0 \right)} \frac{x^2-y^2}{x^2+y^2} = \frac{0}{0} \not\exists$\\
        Si fem $y = mx$, tenim $\lim\limits_{x \rightarrow 0} \frac{x^2-\left( mx \right)^2}{x^2+\left( mx \right)^2} = \lim\limits_{x \rightarrow 0} \frac{1-m^2}{1+m^2}$.
        Com depèn de $m$, no existeix el límit.
    \end{exemple}
    \begin{obs}
        Que el límit no depengui de $m$ no implica que el límit existeixi.
    \end{obs}
    \begin{exemple}
        $\lim\limits_{\left( x,y \right) \rightarrow\left( 0,0 \right)} \frac{x^2y}{x^4+y^2}$$\not\exists$, però si $y = mx$,
        tenim el seguent: $\lim\limits_{x\rightarrow0} \frac{mx^3}{x^4+m^2x^2} = \lim\limits_{x\rightarrow0} \frac{mx}{x^2+m^2} = 0$.\\
        Ara, imaginem que $y = x^2$, llavors resulta $\lim\limits_{x\rightarrow0} \frac{x^4}{2x^4} = \frac{1}{2}$
    \end{exemple}
    \begin{exemple}
        $\lim\limits_{\left( x,y \right) \rightarrow\left( 0,0 \right)} \frac{xy^2}{x^2+y^2} = \frac{0}{0}$, fem $y = mx$:\\
        $\lim\limits_{x \rightarrow0} \frac{x\left( mx \right)^2}{x^2+\left( mx \right)^2} = \lim\limits_{x \rightarrow0} \frac{m^2x}{1+m^2} = 0 \leftarrow$
        això ha sigut una perdua de temps, no podem deduir res.\\
        Veiem que, en efecte, $\lim\limits_{\left( x,y \right) \rightarrow\left( 0,0 \right)} \frac{xy^2}{x^2+y^2} = 0$:\\
        si fem $\frac{|x|y^2}{x^2+y^2}$, podem veure que $\frac{y^2}{x^2+y^2} \leq 1$, llavors $\frac{|x|y^2}{x^2+y^2} \leq |x|1 \longrightarrow 0$.
        Per tant, el límit equival a $0$.
    \end{exemple}
    \begin{exercici}[11 de la llista 2] $\hbox{}$
        \begin{itemize}
            \item[e)] $\lim\limits_{\left(x,y\right)\rightarrow\left(0,0\right)} \frac{1-\cos{x}}{x^2+y^2}$
            Si ho fem "a lo bruto" ens donem compte que el límit seria una indeterminació del tipus $\frac{0}{0}$.\\
            Triem $y = mx$ i tenim llavors $\lim\limits_{x \rightarrow 0} \frac{1-\cos{x}}{x^2+m^2x^2}$.
            Després d'aplicar L'Hôpital ens donem compte que $\frac{1}{2\left(1+m^2\right)}$. Com depèn
            de $m$, el límit no existeix.
            \item[f)] $\lim\limits_{\left(x,y,z\right)\rightarrow\left(0,0,0\right)} \frac{xyz}{x^2+y^2+z^2}$
            Si ho fem "a lo bruto" ens donem compte que el límit seria una indeterminació del tipus $\frac{0}{0}$.\\
            No obstant, com son tres variables no podem fer de manera simple L'Hôpital. Podem fer
            servir una altra idea: $\left\lvert\frac{xyz}{x^2+y^2+z^2} \right\rvert = \frac{\left\lvert x \right\rvert \left\lvert y \right\rvert \left\lvert z \right\rvert}{x^2+y^2+z^2}$
            Llavors, savem que podem fer servir $\frac{\left\lvert x \right\rvert \left\lvert y \right\rvert \left\lvert z \right\rvert}{x^2+y^2+z^2} \leq \frac{\frac{1}{2}\left(x^2+y^2\right)\left\lvert z \right\rvert }{x^2+y^2+z^2} \leq \frac{1}{2}\left\lvert z\right\rvert$
            i com $z$ tendeix a $0$, sabem que el límit tendeix a $0$.
            \item[g)] $\lim\limits_{\left(x,y\right)\rightarrow\left(1,-1\right)} \frac{xy+x-y-1}{x^2+y^2-2x+2y+2}$
            Si ho fem "a lo bruto" ens donem compte que el límit seria una indeterminació del tipus $\frac{0}{0}$.\\
            Llavors, fem $y = m(x-1)-1$, tenim, després de molt simplificar, el seguent $\lim\limits_{x\rightarrow1}\frac{m\left(x-1\right)^2}{\left(x-1\right)^2+m^2\left(x-1\right)^2}$.
            Si simplifiquem una mica més, ens donem compte que depèn de $m$. Som feliços per que
            sabem que el límit no exieteix i podem continuar amb la nostra vida.
        \end{itemize}
    \end{exercici}
    \begin{definicio}
        Una funció $f: \mathbb{R}^n \rightarrow \mathbb{R}$ és continua a $x_0 \in \mathbb{R}^n$ si $\lim\limits_{x \rightarrow x_0} f\left( x \right) = f\left( x_0 \right)$
    \end{definicio}
    \begin{exercici}[14 de la llista 2]
        Sigui
        \begin{displaymath}
            f\left(x,y\right) = \begin{cases}
                \frac{x^2y}{x^4+y^2} & \text{si } \left(x,y\right) \neq \left(0,0\right)\\
                \alpha & \text{si } \left(x,y\right) = \left(0,0\right)
            \end{cases} 
        \end{displaymath}
        Comproveu que $f$ no es continua a l'origen.\\
        Sabem que el límit, si existeix, serà igual a $\alpha$. Fem $y = mx$ per veure si existeix.
        Tenim llavors $\lim\limits_{x\rightarrow 0} \frac{mx}{x^2+m^2} = 0$, com no depén de $m$,
        potser existeix.\\
        Y si fem $y=mx^2$? Llavors depén de $m$, ja que $\lim\limits_{x\rightarrow 0} \frac{mx^4}{x^4+m^2x^4} = \frac{m}{1+m^2}$.
        Per tant, $\alpha$ no existeix i $f$ no és continua a $\left(0,0\right)$. 
    \end{exercici}
    \begin{corolari}[Propietats de la continuitat]
        Tenim $f$ continua a $x_0$ i $g$ continua a $x_0$.
        \begin{enumerate}
            \item $f+g$ és continua a $x_0$
            \item $f\times g$ és continua a $x_0$
            \item $\frac{f}{g}$ és continua a $x_0$ si $g\left( x_0 \right) \neq 0$
        \end{enumerate}
    \end{corolari}
    \begin{teorema}[Weistrass (Te la meto por detrás)]
        Sigui $K \subset \mathbb{R}^n$ compacte i sigui $f: K \rightarrow \mathbb{R}$ continua a $K$.
        Aleshores $f$ te un màxim i un minim a $K$, és a dir, existeix $x_\text{min}$ i $x_\text{max} \in K$
        tals que $f(x_\text{min}) \leq f\left( x \right) \leq f(x_\text{max}) \hbox{ } \forall x \in K$.
    \end{teorema}
    \begin{obs}
        Es fundamental que $K$ sigui compacte. $f\left( x,y \right) = x$ no té máxim a $\mathfrak{B}_R\left( 0,1 \right). \leftarrow$ no
        es compacte.
    \end{obs}
    \subsection{Calcul diferencial}
    \begin{definicio}
        Sigui $f$ una funció definida a $\mathbb{R}^n$ i sigui $x_0=\left\{x_{0,1}, x_{0,2}, \dots, x_{0,n}\right\}$, 
        llavors $\frac{\partial f}{\partial x_i}\left(x_0\right) = \lim\limits_{h\rightarrow0} \frac{f\left(x_{0,1}, \dots, x_{0,i}+h, \dots, x_{0,n}\right)-f(x_0)}{h}$
        (és a dir, mantenim les variables fixes menys $x_i$).
    \end{definicio}
    \begin{obs}
        Hi ha funcions amb derivades parcials a tot punt peró no continues.
    \end{obs}
    Això, i la perdua de la visió geometrica de la derivada (amb una variable, genera la recta
    tangent. La derivada parcial no la genera.) ens fa veure que la derivada parcial no es gaire
    atractiva.\\
    La soluciò és la diferencial
    \begin{notacio}
        Si $f$ és una funció definida a $\mathbb{R}^n$, $\underbrace{\nabla f}_\text{gradient} = \left(\frac{\partial f}{\partial x_1}, \dots, \frac{\partial f}{\partial x_n} \right)$
    \end{notacio}
    \begin{definicio}
        Sigui $f$ una funció definida a $\mathbb{R}^n$ i $x_0 \in \mathbb{R}^n$ diem que $f$ és
        diferenciable a $x_0$ si $\frac{\partial f}{\partial x_i}(x_0)$ existeixen per tot $i = 1, 2, \dots, n$
        i a més $\lim\limits_{x \to x_0} \frac{f\left(x\right) - f\left(x_0\right)-\left\langle \nabla f\left(x_0\right), x-x_0\right\rangle }{\left\lVert x-x_0\right\rVert } = 0$
    \end{definicio}
    \begin{proposicio}
        Figura tangent a la gràfica de $f\left(x\right)$ al punt $x_0$ es
        \begin{displaymath}
            z = f\left(x_0\right) + \left\langle \nabla f \left(x_0\right), \left(x,y\right)-x_0\right\rangle 
        \end{displaymath} 
    \end{proposicio}
    \begin{obs}
        \begin{itemize}
            \item El vector normal a la figura tangent d'una grafica al punt $x_0$ és el $\nabla f\left(x_0\right)$ i $-1$.
            \item Si $f$ té derivades parcials i totes les derivades parcials son continues al punt $x_0$
            llavors $f$ és diferenciable al punt $x_0$.
        \end{itemize}
    \end{obs}
    \begin{proposicio}
        Si $f$ és diferenciable a $x_0 \rightarrow f$ és continua a $x_0$.
    \end{proposicio}
    \begin{definicio}
        Sigui $f$ una funció definida a $\mathbb{R}^n$ i sigui $x_0 \in \mathbb{R}^n$,
        sigui $\vec{e} \in \mathbb{R}^n$ amb $\left\lVert \vec{e} \right\rVert = 1$. Definim
        \begin{displaymath}
            \mathcal{D}_{\vec{e}}f\left(x_0\right) = \lim\limits_{t\to0} \frac{f\left(x_0+\vec{e} t\right)-f\left(x_0\right)}{t}
        \end{displaymath}
        Això s'anomena derivada direccional. 
    \end{definicio}
    \begin{proposicio}
        \begin{displaymath}
            \mathcal{D}_ef\left(x_0\right) = \left\langle \nabla f \left(x_0\right), \vec{e}\right\rangle 
        \end{displaymath}
    \end{proposicio}
    \begin{corolari}
        Per tot $\vec{e} \in \mathbb{R}^n$ amb $\left\lVert \vec{e} \right\rVert = 1$, tenim $-\left\lVert \nabla f(x_0)\right\rVert \leq \mathcal{D}_{\vec{e}}f\left(x_0\right) \leq \left\lVert \nabla f(x_0)\right\rVert$ 
    \end{corolari}
    \begin{definicio}
        Sigui $f: \mathbb{R}^n \to \mathbb{R}^m$, $\left(x_1, x_2, \dots, x_n\right) \to \left(f_1, f_2, \dots, f_m\right)$.
        $f$ és diu diferenciable a $x_0 \in \mathbb{R}^n$ si $\left(f_1, f_2, \dots, f_n\right)$  és diferenciable a $x_0$.\\
        En aquest cas, escrivim
        \begin{displaymath}
            \underbrace{\mathcal{D}f\left(x_0\right)}_\text{matriu diferencial} =
            \begin{pmatrix}
                \frac{\partial f_1}{\partial x_1} & \frac{\partial f_1}{\partial x_2} & \dots & \frac{\partial f_1}{\partial x_n}\\
                \frac{\partial f_2}{\partial x_1} & \frac{\partial f_2}{\partial x_2} & \dots & \frac{\partial f_2}{\partial x_n}\\
                \vdots                            & \vdots                            & \ddots& \vdots\\
                \frac{\partial f_m}{\partial x_1} & \frac{\partial f_m}{\partial x_2} & \dots & \frac{\partial f_m}{\partial x_n}\\
            \end{pmatrix}
        \end{displaymath}
    \end{definicio}
    \begin{definicio}
        Sigui $\mathbb{R}^n \xlongrightarrow{f} \mathbb{R}^m \xlongrightarrow{g} \mathbb{R}^k$, la
        regla de la cadena és
        \begin{displaymath}
            \mathcal{D}\left(g \circ f\right)(x_0) = \mathcal{D}g\left(f\left(x_0\right) \right) \mathcal{D}f\left(x_0\right)
        \end{displaymath}
    \end{definicio}
    \begin{definicio}
        Una corba a $\mathbb{R}^n$ és una aplicació $\gamma = [a,b] \to \mathbb{R}^n$ 
    \end{definicio}
    \begin{notacio}
        Per $\gamma$, al fer la derivada, farem servir $\dot{\gamma}$
    \end{notacio}
    \begin{definicio}
        La longitud d'una corba $\gamma = [a,b] \to \mathbb{R}^n$ és
        \begin{displaymath}
            \int_a^b \left\lVert \dot{\gamma}\left(t\right)\right\rVert  dt
        \end{displaymath}
    \end{definicio}
    \begin{definicio}
        Una superficie a $\mathbb{R}^3$ és un conjunt de forma $\left\{\left(x,y,z\right) \in \mathbb{R}^3 : f\left(x,y,z\right) = c\right\}$
        on $f: \mathbb{R}^3 \to \mathbb{R}$ i $c \in \mathbb{R}$.
    \end{definicio}
    \begin{obs}
        $\nabla f\left(x,y,z\right)$ és el vector normal a la superficie $f\left(x,y,z\right) = c$. 
    \end{obs}
\end{document}