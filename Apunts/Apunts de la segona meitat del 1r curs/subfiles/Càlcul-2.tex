\documentclass[../main.tex]{subfiles}
\graphicspath{{\subfix{../images/}}}

\begin{document}
    \begin{definicio}
        $\mathbb{R}^n = \{(x_1, x_2, \dots, x_n): x_1, \dots, x_n \in \mathbb{R}\}$
    \end{definicio}
    \begin{definicio}
        Siguin $(x_1, x_2, \dots, x_n) \in \mathbb{R}^n$ i $(y_1, y_2, \dots, y_n) \in \mathbb{R}^n$,
        definim $\underbrace{<(x_1, x_2, \dots, x_n), (y_1, y_2, \dots, y_n)>}_\text{(producte escalar)} = x_1y_1 + x_2y_2 + \dots + x_ny_n$
        com a producte escalar.
    \end{definicio}
    \begin{definicio}
        $||(x_1, x_2, \dots, x_n)|| = +\sqrt{<x, x>}$ on $x = (x_1, x_2, \dots, x_n)$. Això és
        la distància del punt $x$ a $(0, 0, \dots, 0)$.  
    \end{definicio}
    Propietats de la norma:
    \begin{enumerate}
        \item $||x|| \geq 0$ per tot $x \in \mathbb{R}^n$
        \item $||\lambda x|| = |\lambda| ||x||$ per tot $x \in \mathbb{R}^n$ i $\lambda \in \mathbb{R}$
        \item Desigualtat triangular: $||x+y|| \leq ||x|| + ||y||$ per tot $x, y \in \mathbb{R}^n$
    \end{enumerate}
    Desigualtat de Cauchy-Schwartz: $<x,y> \leq ||x||||y||$ per tot $x, y \in \mathbb{R}^n$. Això ho
    acceptem.
    \begin{obs}
        Observem que $-1 \leq \frac{<x,y>}{||x||||y||} \leq 1$ i definim l'angle entre els vectors $x$ i $y$
        com l'angle $\alpha$ tal que $\cos{\alpha} = \frac{<x,y>}{||x||||y||}$, és a dir, $<x, y> = ||x||||y||\cos{\alpha}$.
    \end{obs}
    \begin{exemple}
        Trobem els valors de $\mathbb{R}^3 \perp (-1, -2, 1)$. Busquem $(x_1, x_2, x_3) \in \mathbb{R}^n$
        tal que $<(x_1, x_2, x_3), (-1, -2, 1)> = 0 \Leftrightarrow -x_1-2x_2+x_3=0$.
    \end{exemple}
    \subsection{Boles a \texorpdfstring{$\mathbb{R}^n$}{Diverses dimensions}}
    \underline{Si $n=2$} la bola de centre $(x_0, y_0)$ i radi $R$ és $\{(x,y) \in \mathbb{R}^2: ||(x,y)-(x_0, y_0)|| < R\} =$
    (disc) $= \{(x,y)\in \mathbb{R}^2: (x-x_0)^2+(y-y_0)^2<R^2\}$. Això és una bola oberta, denotada per
    \begin{notacio}
        Fem servir $\mathfrak{B}_R(x_0, y_0, \dots)$ la bola oberta ($< R$) i $\overline{\mathfrak{B}_R}(x_0, y_0, \dots)$
        la tancada ($\leq R$). Es farà servir més la bola oberta.
    \end{notacio}
    \begin{definicio}
        Sigui $A \subset \mathbb{R}^n$, definim $\mathring{A} = \{\vec{x} \in \mathbb{R}^n : \exists R > 0 | \mathfrak{B}_R(\vec{x}) \subset A\}$
    \end{definicio}
    \begin{exemple}
        $\hbox{ }$
        \begin{enumerate}
            \item $A = \{(x,y):x \geq 0\}$, llavors $\mathring{A} = \{(x,y):x > 0\}$
            \item $A = \{(x,y,z):-a \leq x \leq a, -b \leq y \leq b, -c \leq z \leq c\}$, llavors
            $\mathring{A} = \{(x,y,z):-a < x < a, -b < y < b, -c < z < c\}$
        \end{enumerate}
    \end{exemple}
    \begin{definicio}
        Un conjunt $A \subset \mathbb{R}^n$ es diu obert si $A = \mathring{A}$, és a dir, si tot
        punt del conjunt $A$ és també un punt d'$\mathring{A}$.
    \end{definicio}
    \begin{definicio}
        Sigui $A \subset \mathbb{R}^n$, definim $\overline{A} = \{ x \in \mathbb{R}^n : \forall R>0, B_R(x) \cap A \neq \emptyset\}$.
        Diem que $\overline{A}$ és l'adherència d'$A$.
    \end{definicio}
    \begin{exemple}
        $A = \{(x,y) \in \mathbb{R}^2: yx < 1, x > 0\}$. $\overline{A} = \{(0, y): y\in \mathbb{R}\} \cup \{(x, y): yx \leq 1, x > 0\}$
    \end{exemple}
    \begin{definicio}
        Un conjunt ($A \subset \mathbb{R}^n$) és tancat si $A = \overline{A}$.
    \end{definicio}
    \begin{proposicio}
        $A$ és obert $\Longleftrightarrow A^c$ és tancat. 
    \end{proposicio}
    \begin{definicio}
        La frontera d'un conjunt, $B_R(A) = \overline{A} \setminus \mathring{A}$.
    \end{definicio}
    \begin{definicio}
        $A \subset \mathbb{R}^n$ es diu acotat si $A$ està contingut dins d'una bola.
    \end{definicio}
    \begin{definicio}
        $A \subset \mathbb{R}^n$ es diu compacte si $A$ és tancat i acotat.
    \end{definicio}
    \subsection{Límits de funcions i continuïtat}
    \begin{definicio}
        Sigui $A \subset \mathbb{R}^n$ i $f: A\rightarrow \mathbb{R}$, definim graf$(f) = \{(x, f(x)) \in \mathbb{R}^{n+1}: x\in A\}$.
    \end{definicio}
    \begin{definicio}
        El conjunt de nivell de la funció $f$ és $\{x \in A: f(x) = c\}$.
    \end{definicio}
    \subsubsection{Límit d'una funció a un punt}
    Siguin $A \subset \mathbb{R}^n$, $f: A \rightarrow \mathbb{R}$ i $x_0 \in \mathbb{R}^n$.
    \begin{definicio}
        Diem que $\lim\limits_{x\rightarrow x_0} f(x) = L$ si per tot $\varepsilon > 0$, existeix $\delta > 0$
        tal que $|f(x)-L| < \varepsilon$ si $0 < ||x-x_0|| < \delta$.
    \end{definicio}
    \begin{corolari}(Propietats dels límits)
        Suposem $\lim\limits_{x\rightarrow x_0} f(x) = L_1$, $\lim\limits_{x\rightarrow x_0} g(x) = L_2$
        \begin{enumerate}
            \item $\lim\limits_{x\rightarrow x_0} (f(x) + g(x)) = L_1 + L_2$
            \item $\lim\limits_{x\rightarrow x_0} (f(x) \times g(x)) = L_1 \times L_2$
            \item Si $L_2 \neq 0$, llavors $\lim\limits_{x\rightarrow x_0} (\frac{f(x)}{g(x)}) = \frac{L_1}{L_2}$ 
        \end{enumerate}
    \end{corolari}
    \begin{exemple}
        $\lim\limits_{(x,y) \rightarrow (0, 0)}(x^2+y^2)^\alpha \sin{\frac{1}{x^2+y^2}}$ on $\alpha \in \mathbb{R}$.
        \\Observem que $\lim\limits_{(x,y) \rightarrow (0, 0)}(x^2+y^2)^\alpha$ = $\begin{cases}
            0 \text{ si } \alpha > 0\\
            +\infty \text{ si } \alpha < 0\\
            1 \text{ si } \alpha = 0
        \end{cases}$
        \begin{itemize}
            \item Si $\alpha > 0$, observem $|(x^2+y^2)^\alpha \sin{\frac{1}{x^2+y^2}}| \leq (x^2+y^2) \longrightarrow 0$
            \item Si $\alpha < 0$  veiem que el límit no existeix.\\
            Trobem $\substack{(x_n, y_n) \rightarrow (0, 0)\\(z_n, w_n) \rightarrow (0, 0)}$ quan $n \rightarrow \infty$ per fer \begin{displaymath}
                \lim\limits_{(x_n,y_n) \rightarrow (0, 0)}(x_n^2+y_n^2)^\alpha \sin{\frac{1}{x_n^2+y_n^2}} = \lim\limits_{(z_n,w_n) \rightarrow (0, 0)}(z_n^2+w_n^2)^\alpha \sin{\frac{1}{z_n^2+w_n^2}}
            \end{displaymath}       
            Triem $(x_n, y_n)$ tal que $x_n^2 + y_n^2 = \frac{1}{n\pi}$ i triem $(z_n, w_n)$ tal que $z_n^2+w_n^2 = \frac{1}{\pi/2 + 2\pi n}$.\\
            Per exemple $x_n = \frac{1}{\sqrt{n\pi}}$, $y_n = 0$, $z_n = 0$ i $w_n = \frac{1}{\sqrt{\pi/2+2\pi n}}$.\\
            Això provocarà que el primer límit doni $0$ i el segon $\infty$, que, òbviament, no són
            iguals.
            \item Si $\alpha = 0$, exercici pel lector. No existeix.
        \end{itemize}
    \end{exemple}
    \subsubsection{Límit seguint rectes i límit a un punt}
    \begin{proposicio}
        Suposem $\lim\limits_{(x,y) \rightarrow (0,0)} f(x,y) = L$, llavors $\lim\limits_{x \rightarrow 0} f(x, mx) = L \hbox{ }\forall m \in \mathbb{R}$.
    \end{proposicio}
    Utilitzarem aquest fet de la següent forma:
    \begin{corolari}
        Si $\lim\limits_{x\rightarrow0} f(x, mx)$ depèn de $m$, aleshores $\lim\limits_{x\rightarrow0}f(x,y)$ no existeix.
    \end{corolari}
    \begin{exemple}
        $\lim\limits_{(x,y) \rightarrow(0,0)} \frac{x^2-y^2}{x^2+y^2} = \frac{0}{0} \not\exists$\\
        Si fem $y = mx$, tenim $\lim\limits_{x \rightarrow 0} \frac{x^2-(mx)^2}{x^2+(mx)^2} = \lim\limits_{x \rightarrow 0} \frac{1-m^2}{1+m^2}$.
        Com depèn de $m$, no existeix el límit.
    \end{exemple}
    \begin{obs}
        Que el límit no depengui de $m$ no implica que el límit existeixi.
    \end{obs}
    \begin{exemple}
        $\lim\limits_{(x,y) \rightarrow(0,0)} \frac{x^2y}{x^4+y^2}$$\not\exists$, però si $y = mx$,
        tenim el seguent: $\lim\limits_{x\rightarrow0} \frac{mx^3}{x^4+m^2x^2} = \lim\limits_{x\rightarrow0} \frac{mx}{x^2+m^2} = 0$.\\
        Ara, imaginem que $y = x^2$, llavors resulta $\lim\limits_{x\rightarrow0} \frac{x^4}{2x^4} = \frac{1}{2}$
    \end{exemple}
    \begin{exemple}
        $\lim\limits_{(x,y) \rightarrow(0,0)} \frac{xy^2}{x^2+y^2} = \frac{0}{0}$, fem $y = mx$:\\
        $\lim\limits_{x \rightarrow0} \frac{x(mx)^2}{x^2+(mx)^2} = \lim\limits_{x \rightarrow0} \frac{m^2x}{1+m^2} = 0 \leftarrow$
        això ha sigut una perdua de temps, no podem deduir res.\\
        Veiem que, en efecte, $\lim\limits_{(x,y) \rightarrow(0,0)} \frac{xy^2}{x^2+y^2} = 0$:\\
        si fem $\frac{|x|y^2}{x^2+y^2}$, podem veure que $\frac{y^2}{x^2+y^2} \leq 1$, llavors $\frac{|x|y^2}{x^2+y^2} \leq |x|1 \longrightarrow 0$.
        Per tant, el límit equival a $0$.
    \end{exemple}
    \begin{definicio}
        Una funció $f: \mathbb{R}^n \rightarrow \mathbb{R}$ és continua a $x_0 \in \mathbb{R}^n$ si $\lim\limits_{x \rightarrow x_0} f(x) = f(x_0)$
    \end{definicio}
    \begin{corolari}[Propietats de la continuitat]
        Tenim $f$ continua a $x_0$ i $g$ continua a $x_0$.
        \begin{enumerate}
            \item $f+g$ és continua a $x_0$
            \item $f\times g$ és continua a $x_0$
            \item $\frac{f}{g}$ és continua a $x_0$ si $g(x_0) \neq 0$
        \end{enumerate}
    \end{corolari}
    \begin{teorema}[Weistrass (Te la meto por detrás)]
        Sigui $K \subset \mathbb{R}^n$ compacte i sigui $f: K \rightarrow \mathbb{R}$ continua a $K$.
        Aleshores $f$ te un màxim i un minim a $K$, és a dir, existeix $x_\text{min}$ i $x_\text{max} \in K$
        tals que $f(x_\text{min}) \leq f(x) \leq f(x_\text{max}) \hbox{ } \forall x \in K$.
    \end{teorema}
    \begin{obs}
        Es fundamental que $K$ sigui compacte. $f(x,y) = x$ no té máxim a $B_R(0,1). \leftarrow$ no
        es compacte.
    \end{obs}
\end{document}