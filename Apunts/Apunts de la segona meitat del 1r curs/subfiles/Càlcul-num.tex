\documentclass[../main.tex]{subfiles}
\graphicspath{{\subfix{../images/}}}

\begin{document}
    Hi ha 3 tipus d'errors (4 si em comptes a mi):
    \begin{enumerate}
        \item Errors en les dades d'entrada
        \item Errors a les operacions
        \item Errors de truncament
    \end{enumerate}
    Aquí es tractaran principalment els dos primers.
    \begin{teorema}[Representació en punt flotant en base $b$]
        Per $b \in \mathbb{N}$, $b \geq 2$. Tot $x \in \mathbb{R}, x\neq 0$ pot ser representat de
        la següent forma:
        \begin{displaymath}
            x = s(\sum\limits_{i=1}^{\infty}\alpha_ib^{-i}) b^q
        \end{displaymath}
        amb $s \in \{-1, 1\}, q \in \mathbb{Z}$ i $\alpha_i \in \{0, 1, \dots, b-1\}$. A més, la
        representació anterior és única si $\alpha_1 \neq 0$ i els $\alpha_i$ no són tots $b-1$ 
        d'una posició en endavant.
    \end{teorema}
    \begin{definicio}[Representació en punt flotant]
        És la versió finita de la representació. En aquesta representació, tot nombre $x$ consta de 
        \begin{itemize}
            \item el signe, $s$
            \item la mantissa, que només consta d'un nombre finit de dígits, $\delta_1, \delta_2, \dots, \delta_t$
            expressats en base $b$, i
            \item l'exponent, $q$, que està limitat a un rang prefixat, $q_\text{min} \leq q \leq q_\text{max}$
        \end{itemize}
    \end{definicio}
    \begin{notacio}
        Si $x$ és el valor exacte, $\tilde{x}$ és el valor aproximat.
    \end{notacio}
    \begin{definicio}
        L'error absolut és $\Delta x = x - \tilde{x}$. L'error relatiu és $\frac{\Delta x}{x} = \frac{x-\tilde{x}}{x} = 1 - \frac{\tilde{x}}{x}$
    \end{definicio}
    \begin{definicio}
        Definim la fórmula de propagació d'error com $|\Delta f(x_o)| \lesssim h|f'(x_o)|$.
    \end{definicio}
    \begin{definicio}
        Amb dues variables, $|f(x+h, y+k)| \lesssim |\frac{\partial f(x,y)}{\partial x}||h| + |\frac{\partial f(x,y)}{\partial y}||k|$ (dicho en clase: ¿por qué no se cancelan los $\partial$?)
    \end{definicio}
    \begin{exercici}
        Calcular de forma exacta i l'error absolut de $(2\pm 0.01)(3\pm 0.2)^2$.\\
        De manera exacta, tenim $F(x,y) = xy^2$
        \begin{displaymath}
            F(2, 3) = 2\times3^2 = 18
        \end{displaymath}
        L'error absolut és el següent
        \begin{displaymath}
            |\Delta F(x,y)| \lesssim |\frac{\partial F}{\partial x}||h| +  |\frac{\partial F}{\partial y}||k| = |y^2||h| + |2xy||k|
        \end{displaymath}
        Si substituïm $x = 2, y = 3, h = 0.01$ i $k = 0.2$
        \begin{displaymath}
            |\Delta F(x,y)| \lesssim |9||0.01| + |12||0.2| = 0.09+2.4 = 2.49
        \end{displaymath}
    \end{exercici}
    \begin{exercici}
        Calcular de forma exacta i l'error absolut de $(2\pm 0.01)e^{-1\pm 0.2}$.\\
        De manera exacta, tenim $F(x,y) = xe^y$
        \begin{displaymath}
            F(2, -1) = 2e^{-1}
        \end{displaymath}
        L'error absolut és el següent
        \begin{displaymath}
            |\Delta F(x,y)| \lesssim |\frac{\partial F}{\partial x}||h| +  |\frac{\partial F}{\partial y}||k| = |e^y||h| + |xe^y||k|
        \end{displaymath}
        Si substituïm $x = 2, y = -1, h = 0.01$ i $k = 0.2$
        \begin{displaymath}
            |\Delta F(x,y)| \lesssim |e^{-1}||0.01| + |2e^{-1}||0.2|
        \end{displaymath}
    \end{exercici}
\end{document}