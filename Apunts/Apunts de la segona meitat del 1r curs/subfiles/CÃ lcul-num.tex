\documentclass[../main.tex]{subfiles}
\graphicspath{{\subfix{../images/}}}

\begin{document}
    Hi ha 3 tipus d'errors (4 si em comptes a mi):
    \begin{enumerate}
        \item Errors en les dades d'entrada
        \item Errors a les operacions
        \item Errors de truncament
    \end{enumerate}
    Aquí es tractaran principalment els dos primers.
    \begin{teorema}[Representació en punt flotant en base $b$]
        Per $b \in \mathbb{N}$, $b \geq 2$. Tot $x \in \mathbb{R}, x\neq 0$ pot ser representat de
        la següent forma:
        \begin{displaymath}
            x = s(\sum\limits_{i=1}^{\infty}\alpha_ib^{-i}) b^q
        \end{displaymath}
        amb $s \in \{-1, 1\}, q \in \mathbb{Z}$ i $\alpha_i \in \{0, 1, \dots, b-1\}$. A més, la
        representació anterior és única si $\alpha_1 \neq 0$ i els $\alpha_i$ no són tots $b-1$ 
        d'una posició en endavant.
    \end{teorema}
    \begin{definicio}[Representació en punt flotant]
        És la versió finita de la representació. En aquesta representació, tot nombre $x$ consta de 
        \begin{itemize}
            \item el signe, $s$
            \item la mantissa, que només consta d'un nombre finit de dígits, $\delta_1, \delta_2, \dots, \delta_t$
            expressats en base $b$, i
            \item l'exponent, $q$, que està limitat a un rang prefixat, $q_\text{min} \leq q \leq q_\text{max}$
        \end{itemize}
    \end{definicio}
    \begin{notacio}
        Si $x$ és el valor exacte, $\tilde{x}$ és el valor aproximat.
    \end{notacio}
    \begin{definicio}
        L'error absolut és $\Delta x = x - \tilde{x}$. L'error relatiu és $\frac{\Delta x}{x} = \frac{x-\tilde{x}}{x} = 1 - \frac{\tilde{x}}{x}$
    \end{definicio}
    \begin{definicio}
        Definim la fórmula de propagació d'error com $\left\lvert \Delta f(x_o) \right\rvert \lesssim h|f'(x_o)|$.
    \end{definicio}
    \begin{definicio}
        Amb dues variables, $\left\lvert f(x+h, y+k) \right\rvert \lesssim \left\lvert \frac{\partial f\left( x,y \right)}{\partial x} \right\rvert\left\lvert h \right\rvert + \left\lvert \frac{\partial f\left( x,y \right)}{\partial y} \right\rvert\left\lvert k \right\rvert$
        (dicho en clase: ¿por qué no se cancelan los $\partial$?)
    \end{definicio}
    \begin{definicio}
        Per derivades succesives, $\left\lvert f(\tilde{x}) \right\rvert \lesssim \sum\limits_{i=1}^n\left\lvert \frac{\partial f\left( x \right)}{\partial x_i} \right\rvert\left\lvert \tilde{x_i}-x_i \right\rvert$
    \end{definicio}
    \begin{exercici}
        Calcular de forma exacta i l'error absolut de $(2\pm 0.01)(3\pm 0.2)^2$.\\
        De manera exacta, tenim $F\left( x,y \right) = xy^2$
        \begin{displaymath}
            F\left( 2, 3 \right) = 2\times3^2 = 18
        \end{displaymath}
        L'error absolut és el següent
        \begin{displaymath}
            \left\lvert \Delta F\left( x,y \right) \right\rvert \lesssim \left\lvert \frac{\partial F}{\partial x} \right\rvert\left\lvert h \right\rvert +  \left\lvert \frac{\partial F}{\partial y} \right\rvert\left\lvert k \right\rvert = \left\lvert y^2 \right\rvert\left\lvert h \right\rvert + \left\lvert 2xy \right\rvert\left\lvert k \right\rvert
        \end{displaymath}
        Si substituïm $x = 2, y = 3, h = 0.01$ i $k = 0.2$
        \begin{displaymath}
            \left\lvert \Delta F\left( x,y \right) \right\rvert \lesssim \left\lvert 9 \right\rvert\left\lvert 0.01 \right\rvert + \left\lvert 12 \right\rvert\left\lvert 0.2 \right\rvert = 0.09+2.4 = 2.49
        \end{displaymath}
    \end{exercici}
    \begin{exercici}
        Calcular de forma exacta i l'error absolut de $(2\pm 0.01)e^{-1\pm 0.2}$.\\
        De manera exacta, tenim $F\left( x,y \right) = xe^y$
        \begin{displaymath}
            F(2, -1) = 2e^{-1}
        \end{displaymath}
        L'error absolut és el següent
        \begin{displaymath}
            \left\lvert \Delta F\left( x,y \right) \right\rvert \lesssim \left\lvert \frac{\partial F}{\partial x} \right\rvert\left\lvert h \right\rvert +  \left\lvert \frac{\partial F}{\partial y} \right\rvert\left\lvert k \right\rvert = \left\lvert e^y \right\rvert\left\lvert h \right\rvert + \left\lvert xe^y \right\rvert\left\lvert k \right\rvert
        \end{displaymath}
        Si substituïm $x = 2, y = -1, h = 0.01$ i $k = 0.2$
        \begin{displaymath}
            \left\lvert \Delta F\left( x,y \right) \right\rvert \lesssim \left\lvert e^{-1} \right\rvert\left\lvert 0.01 \right\rvert + \left\lvert 2e^{-1} \right\rvert\left\lvert 0.2 \right\rvert
        \end{displaymath}
    \end{exercici}
    \begin{notacio}
        Denotem un nombre $x$ en punt flotant com $\text{fl} \left(x\right)$
    \end{notacio}
    \begin{notacio}
        Quan fem una operació amb punt flotant, ho posem de la següent forma per denotar que
        existeix error:
        \begin{itemize}
            \item $x \oplus y = \text{fl}\left( \text{fl}\left( x \right) + \text{fl}\left( y \right)\right)$
            \item $x \ominus y = \text{fl}\left( \text{fl}\left( x \right) - \text{fl}\left( y \right)\right)$
            \item $x \otimes y = \text{fl}\left( \text{fl}\left( x \right) \cdot \text{fl}\left( y \right)\right)$
            \item $x \oslash y = \text{fl}\left( \text{fl}\left( x \right) / \text{fl}\left( y \right)\right)$
        \end{itemize}
    \end{notacio}
    \begin{obs}
        Posats a donar aproximacions de l'error relatiu, posem més.
    \end{obs}
    \begin{proposicio}[tecnicament és un Lema]
        $\Delta \log{\left\lvert f\left(x\right) \right\rvert} \approx$ l'error relatiu.
    \end{proposicio}
    \begin{proposicio}[Error absolut i relatiu de les operacions elementals]
        $\hbox{ }$
        \begin{itemize}
            \item Error absolut de la suma o resta: Error absolut de $x \pm$ error absolut de $y$.
            \item Error relatiu de la multiplicació: Error relatiu de $x +$ error relatiu de $y$. 
            \item Error relatiu de la divisió: Error relatiu de $x -$ error relatiu de $y$.
        \end{itemize}
    \end{proposicio}
    \begin{definicio}
        Direm que un algorisme és numèricament inestable quan petites pertorbacions dels resultats
        inicials produeixen una gran diferència en el resultat final. Això el farà inservible des
        del punt de vista numèric, donat que amplificarà de manera sistemàtica els errors de les
        dades inicials.
    \end{definicio}
    \subsection{Zeros de funcions}
    El millor mètode per trobar un zero d'una funció és el mètode de bisecció, però és molt lent.\\
    Hi ha molts, millor mirar el diaporama.\\
    \begin{exercici}
        Aplicar el mètode de newton per calcular $\sqrt[k]{c}$ on $c \in \mathbb{R}^+$ i $k \geq 2$:\\
        \begin{displaymath}
            x = \sqrt[k]{c} \Rightarrow 0 = x^k-c \Rightarrow f\left(x\right) = x^k-c
        \end{displaymath}
        Apliquem el mètode de Newton:
        \begin{displaymath}
            x_{n+1} = x_n - \frac{f\left(x_n\right)}{f'\left(x_n\right)}
        \end{displaymath}
        \begin{displaymath}
            x_{n+1} = x_n - \frac{x_n^3-c}{kx_n^{k-1}}
        \end{displaymath}
        Si comencem amb $x_0 = 3$:
        \begin{itemize}
            \item[$x_1$]$= 3.\underline{03}703703$
            \item[$x_2$]$= 3.\underline{0365}8903$
            \item[$x_3$]$= 3.\underline{03658897}$
            \item[$x_4$]$= 3.\underline{03658897}$
        \end{itemize}
    \end{exercici}
    \begin{exercici}
        Calculeu, usant quatre dígits de precisió i els mètodes d'eliminació gaussiana, la solució
        dels sistema
        \begin{displaymath}
            \left(
                \begin{array}{c c c | c}
                    1 & -1 & 3 & 2\\
                    3 & -3 & 1 & -1\\
                    1 & 1  & 0 & 3
                \end{array}
            \right)
            \sim
            \left(
                \begin{array}{c c c | c}
                    1 & -1 & 3 & 2\\
                    0 & 0 & -8 & -7\\
                    0 & -2  & -3 & 1
                \end{array}
            \right)
            \sim
            \left(
                \begin{array}{c c c | c}
                    1 & -1 & 3 & 2\\
                    0 & 1  & \frac{3}{2} & -\frac{1}{2}\\
                    0 & 0 & 1 & \frac{7}{8}
                \end{array}
            \right)
        \end{displaymath}
        \begin{displaymath}
            \Rightarrow \begin{cases}
                x_1 =& \frac{19}{16}\\
                x_2 =& \frac{29}{16}\\
                x_3 =& \frac{7}{9}
            \end{cases}
        \end{displaymath}
    \end{exercici}
    \begin{exercici}
        Obtenir la factoració LU sense pivotatge de
        \begin{displaymath}
            A = 
            \left(
                \begin{array}{c c c}
                    1 & 2 & 3\\
                    4 & 5 & 6\\
                    7 & 8 & 1
                \end{array}
            \right)
        \end{displaymath}
        \begin{displaymath}
            \begin{split}
                A &=\left(
                    \begin{array}{c c c}
                        1 & 2 & 3\\
                        4 & 5 & 6\\
                        7 & 8 & 1
                    \end{array}
                \right)
                =
                \left(
                    \begin{array}{c c c}
                        1 & 0 & 0\\
                        -4 & 1 & 0\\
                        0 & 0 & 1
                    \end{array}
                \right)
                \left(
                    \begin{array}{c c c}
                        1 & 2 & 3\\
                        0 & -3 & -6\\
                        7 & 8 & 1
                    \end{array}
                \right)\\
                &=
                \left(
                    \begin{array}{c c c}
                        1 & 0 & 0\\
                        -4 & 1 & 0\\
                        0 & 0 & 1
                    \end{array}
                \right)
                \left(
                    \begin{array}{c c c}
                        1 & 0 & 0\\
                        0 & 1 & 0\\
                        -7 & 0 & 1
                    \end{array}
                \right)
                \left(
                    \begin{array}{c c c}
                        1 & 2 & 3\\
                        0 & -3 & -6\\
                        0 & -6 & -20
                    \end{array}
                \right)\\
                &=
                \left(
                    \begin{array}{c c c}
                        1 & 0 & 0\\
                        -4 & 1 & 0\\
                        0 & 0 & 1
                    \end{array}
                \right)
                \left(
                    \begin{array}{c c c}
                        1 & 0 & 0\\
                        0 & 1 & 0\\
                        -7 & 0 & 1
                    \end{array}
                \right)
                \left(
                    \begin{array}{c c c}
                        1 & 0 & 0\\
                        0 & 1 & 0\\
                        0 & -2 & 1
                    \end{array}
                \right)
                \left(
                    \begin{array}{c c c}
                        1 & 2 & 3\\
                        0 & -3 & -6\\
                        0 & 0 & -8
                    \end{array}
                \right)
            \end{split}
        \end{displaymath}
        Llavors $U = \left(
            \begin{array}{c c c}
                1 & 2 & 3\\
                0 & -3 & -6\\
                0 & 0 & -8
            \end{array}
        \right)$. Ara, calculem el $L$:
        \begin{displaymath}
            \begin{split}
                L &= \left[\left(
                \begin{array}{c c c}
                    1 & 0 & 0\\
                    -4 & 1 & 0\\
                    0 & 0 & 1
                \end{array}
            \right)
            \left(
                \begin{array}{c c c}
                    1 & 0 & 0\\
                    0 & 1 & 0\\
                    -7 & 0 & 1
                \end{array}
            \right)
            \left(
                \begin{array}{c c c}
                    1 & 0 & 0\\
                    0 & 1 & 0\\
                    0 & -2 & 1
                \end{array}
            \right)\right]^{-1}\\
            &=
            \left(
                \begin{array}{c c c}
                    1 & 0 & 0\\
                    0 & 1 & 0\\
                    0 & -2 & 1
                \end{array}
            \right)^{-1}
            \left(
                \begin{array}{c c c}
                    1 & 0 & 0\\
                    0 & 1 & 0\\
                    -7 & 0 & 1
                \end{array}
            \right)^{-1}
            \left(
                \begin{array}{c c c}
                    1 & 0 & 0\\
                    -4 & 1 & 0\\
                    0 & 0 & 1
                \end{array}
            \right)^{-1}\\
            &=
            \left(
                \begin{array}{c c c}
                    1 & 0 & 0\\
                    0 & 1 & 0\\
                    0 & 2 & 1
                \end{array}
            \right)
            \left(
                \begin{array}{c c c}
                    1 & 0 & 0\\
                    0 & 1 & 0\\
                    7 & 0 & 1
                \end{array}
            \right)
            \left(
                \begin{array}{c c c}
                    1 & 0 & 0\\
                    4 & 1 & 0\\
                    0 & 0 & 1
                \end{array}
            \right)\\
            &=
            \left(
                \begin{array}{c c c}
                    1 & 0 & 0\\
                    4 & 1 & 0\\
                    7 & 2 & 1
                \end{array}
            \right)
            \end{split}
        \end{displaymath}
    \end{exercici}
    \begin{exercici}[30]
        \begin{displaymath}
            A\left(a,b\right) = \left(
                \begin{array}{c c c c}
                    a & b & 0 & 0\\
                    b & a & b & 0\\
                    0 & b & a & b\\
                    0 & 0 & b & a
                \end{array}
            \right)
        \end{displaymath}
        amb $a, b \in \mathbb{R},\; a \neq 0$, determineu la regió dels punts del pla $(a,b)$ per
        als que el mètode de Gauss-Seidel convergeix al resoldre el sistema $A\left(a,b\right)x = c$.
        En el cas $A\left(3, 1\right)$ determineu el nombre de passos que caldria fer per a resoldre
        el sistema amb $c = \left(2,2,8,-5\right)^T$ pel mètode de Gauss-Seidel amb $x^{(0)} = \left(1,0,0,0\right)^T$
        i amb un error $\left\lVert x^{(k)}-x\right\rVert _\infty < 10^{-3}$.\\\\
        \begin{displaymath}
            \left\lvert
                \begin{array}{c c c c}
                    \lambda a & b & 0 & 0\\
                    \lambda b & \lambda a & b & 0\\
                    0 & \lambda b & \lambda a & b\\
                    0 & 0 & \lambda b & \lambda a
                \end{array}
            \right\rvert = 0 = \lambda^2\left(\lambda^2 a^4 - 3\lambda a^2 b^2 + b^4\right)
        \end{displaymath}
        \begin{displaymath}
            \Rightarrow \lambda = \begin{cases}
                0\\
                \frac{3a^2 b^2 \pm \sqrt{9a^4 b^4 - 4a^4b^4}}{2a^4} = \frac{(\pm \sqrt{5}-3)b^2}{2a^2}
            \end{cases}
        \end{displaymath}
        \begin{displaymath}
            R_{GS} = \max{\left\lvert\lambda_i\right\rvert} = \frac{\pm \sqrt{5} - 3}{2a^2} < 1\Rightarrow a > b \sqrt{\frac{\sqrt{5}+1}{2}}
        \end{displaymath}
        Esta malament però dona igual. El procediment està be pero hi ha alguna cosa malament.
    \end{exercici}
    \begin{exercici}[Problema 40]
        $\left\lvert Error\right\rvert = \frac{h^4}{180}(b-a)\left\lvert f^{4)}\right\rvert$
    \end{exercici}
\end{document}