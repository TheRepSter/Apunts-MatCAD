\documentclass[../main.tex]{subfiles}
\graphicspath{{\subfix{../images/}}}

\begin{document}
    Hi ha 3 tipus d'errors (4 si em comptes a mi):
    \begin{enumerate}
        \item Errors en les dades d'entrada
        \item Errors a les operacions
        \item Errors de truncament
    \end{enumerate}
    Aquí es tractaran principalment els dos primers.
    \begin{teorema}[Representació en punt flotant en base $b$]
        Per $b \in \mathbb{N}$, $b \geq 2$. Tot $x \in \mathbb{R}, x\neq 0$ pot ser representat de
        la següent forma:
        \begin{displaymath}
            x = s(\sum\limits_{i=1}^{\infty}\alpha_ib^{-i}) b^q
        \end{displaymath}
        amb $s \in \{-1, 1\}, q \in \mathbb{Z}$ i $\alpha_i \in \{0, 1, \dots, b-1\}$. A més, la
        representació anterior és única si $\alpha_1 \neq 0$ i els $\alpha_i$ no són tots $b-1$ 
        d'una posició en endavant.
    \end{teorema}
    \begin{definicio}[Representació en punt flotant]
        És la versió finita de la representació. En aquesta representació, tot nombre $x$ consta de 
        \begin{itemize}
            \item el signe, $s$
            \item la mantissa, que només consta d'un nombre finit de dígits, $\delta_1, \delta_2, \dots, \delta_t$
            expressats en base $b$, i
            \item l'exponent, $q$, que està limitat a un rang prefixat, $q_\text{min} \leq q \leq q_\text{max}$
        \end{itemize}
    \end{definicio}
    \begin{notacio}
        Si $x$ és el valor exacte, $\tilde{x}$ és el valor aproximat.
    \end{notacio}
    \begin{definicio}
        L'error absolut és $\Delta x = x - \tilde{x}$. L'error relatiu és $\frac{\Delta x}{x} = \frac{x-\tilde{x}}{x} = 1 - \frac{\tilde{x}}{x}$
    \end{definicio}
    \begin{definicio}
        Definim la fórmula de propagació d'error com $\left\lvert \Delta f(x_o) \right\rvert \lesssim h|f'(x_o)|$.
    \end{definicio}
    \begin{definicio}
        Amb dues variables, $\left\lvert f(x+h, y+k) \right\rvert \lesssim \left\lvert \frac{\partial f\left( x,y \right)}{\partial x} \right\rvert\left\lvert h \right\rvert + \left\lvert \frac{\partial f\left( x,y \right)}{\partial y} \right\rvert\left\lvert k \right\rvert$ (dicho en clase: ¿por qué no se cancelan los $\partial$?)
    \end{definicio}
    \begin{exercici}
        Calcular de forma exacta i l'error absolut de $(2\pm 0.01)(3\pm 0.2)^2$.\\
        De manera exacta, tenim $F\left( x,y \right) = xy^2$
        \begin{displaymath}
            F\left( 2, 3 \right) = 2\times3^2 = 18
        \end{displaymath}
        L'error absolut és el següent
        \begin{displaymath}
            \left\lvert \Delta F\left( x,y \right) \right\rvert \lesssim \left\lvert \frac{\partial F}{\partial x} \right\rvert\left\lvert h \right\rvert +  \left\lvert \frac{\partial F}{\partial y} \right\rvert\left\lvert k \right\rvert = \left\lvert y^2 \right\rvert\left\lvert h \right\rvert + \left\lvert 2xy \right\rvert\left\lvert k \right\rvert
        \end{displaymath}
        Si substituïm $x = 2, y = 3, h = 0.01$ i $k = 0.2$
        \begin{displaymath}
            \left\lvert \Delta F\left( x,y \right) \right\rvert \lesssim \left\lvert 9 \right\rvert\left\lvert 0.01 \right\rvert + \left\lvert 12 \right\rvert\left\lvert 0.2 \right\rvert = 0.09+2.4 = 2.49
        \end{displaymath}
    \end{exercici}
    \begin{exercici}
        Calcular de forma exacta i l'error absolut de $(2\pm 0.01)e^{-1\pm 0.2}$.\\
        De manera exacta, tenim $F\left( x,y \right) = xe^y$
        \begin{displaymath}
            F(2, -1) = 2e^{-1}
        \end{displaymath}
        L'error absolut és el següent
        \begin{displaymath}
            \left\lvert \Delta F\left( x,y \right) \right\rvert \lesssim \left\lvert \frac{\partial F}{\partial x} \right\rvert\left\lvert h \right\rvert +  \left\lvert \frac{\partial F}{\partial y} \right\rvert\left\lvert k \right\rvert = \left\lvert e^y \right\rvert\left\lvert h \right\rvert + \left\lvert xe^y \right\rvert\left\lvert k \right\rvert
        \end{displaymath}
        Si substituïm $x = 2, y = -1, h = 0.01$ i $k = 0.2$
        \begin{displaymath}
            \left\lvert \Delta F\left( x,y \right) \right\rvert \lesssim \left\lvert e^{-1} \right\rvert\left\lvert 0.01 \right\rvert + \left\lvert 2e^{-1} \right\rvert\left\lvert 0.2 \right\rvert
        \end{displaymath}
    \end{exercici}
    \begin{notacio}
        Denotem un nombre $x$ en punt flotant com $\text{fl} \left(x\right)$
    \end{notacio}
    \begin{notacio}
        Quan fem una operació amb punt flotant, ho posem de la següent forma per denotar que
        existeix error:
        \begin{itemize}
            \item $x \oplus y = \text{fl}\left( \text{fl}\left( x \right) + \text{fl}\left( y \right)\right)$
            \item $x \ominus y = \text{fl}\left( \text{fl}\left( x \right) - \text{fl}\left( y \right)\right)$
            \item $x \otimes y = \text{fl}\left( \text{fl}\left( x \right) \cdot \text{fl}\left( y \right)\right)$
            \item $x \oslash y = \text{fl}\left( \text{fl}\left( x \right) / \text{fl}\left( y \right)\right)$
        \end{itemize}
    \end{notacio}
\end{document}